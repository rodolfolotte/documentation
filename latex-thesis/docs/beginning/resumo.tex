\begin{resumo}
Urban environments are regions in which spectral and spatial variability are extremely high, with a huge range of shapes and sizes, they also demand high resolution images for applications involving their study. These environments can grow over time, applications related to their large-scale monitoring tend to rely on autonomous intelligent systems that, along with high-resolution images, can help and even predict everyday situations. In addition to the detection of these features, 3D representations of these environments have also been object of study to assist in the investigation of the environmental quality of very dense areas, occupational socioeconomic patterns, the construction of urban landscape models, building demolitions or flood simulations for evacuation plans and strategic delimitation, among countless others. The main objective of this study was to explore the advantages of such technologies, in order to present an automatic methodology for the detection and reconstruction of urban elements, and also to understand the difficulties that still surround the automatic mapping of these environments. Specifically we aimed: (i) To develop a routine of automatic classification of facade features in 2D domain, using a Convolutional Neural Network (CNN); (ii) Using the same images, obtain the facade geometry using Structure-from-Motion (SfM) and Multi-View Stereo (MVS) techniques; (iii) Evaluate the performance of the CNN for different urban scenarios and architectural styles; (iv) Evaluate the performance of the CNN in a real application in Brazil, whose architecture differs from the datasets used in the neural model training; and (v) Classify the 3D model of the extracted facade using images segmented in 2D domain by the Ray-Tracing (RT) technique. In order to atempt that, the methodology was splited into 2D analysis (detection) and 3D (reconstruction). So in the first, a supervised CNN is used to segment terrestrial optical images of facades into six classes: roof, window, wall, door, balcony and shops. At the same time, the facade is reconstructed using the SfM/MVS technique, obtaining the geometry of the scene. Finally, the results of segmentation in both domains, 2D and 3D, are then merged by the Ray-Tracing technique, finally obtaining the 3D model classified. It is demonstrated that the proposed methodology is robust toward complex scenarios. The inferences made with the CNN reached up to 93\% accuracy, and 90\% F1-score for most of the datasets used. For scenarios not used for training, the neural model reached lower accuracy indexes, justified by the high differentiation of architectural styles. However, the use of deep neural models gives chances for new configurations and use with other deep architectures to improve results, especially for unsupervised models. Finally, the work demonstrated the autonomous capacity of a CNN against the complexity of urban environments, in order to diversify between different styles of facades. Although there are improvements to be made regarding 3D classification, the methodology is consistent and allowed to combine state-of-the-art methods in the detection and reconstruction of urban elements, as well as providing support for new studies and projections on even more distinct scenarios.

%A particular object can reveal much about the urban context, consequently helps to determine the application of environmental policies and urbanization: buildings. During mid-2000s, a common premise to identify their characteristics from remote sensing data was to delineate its roof-edges (footprint), which most of the time, reveals its planimetric location and could be then used to get simple building extrusions, where the volumetry is generated by its footprint and height. 

%A particular object can reveal much about the urban context, consequently help determine the application of environmental policies and urbanization: buildings. During the mid-2000s, a common premise for identifying them in aerial imagery was the delineation of their roof (\emph{footprint}), which often reveals their planimetric location. With the advent of laser sensors years later, the same footprints were used together with height information to obtain simple 3D models of these buildings.

%Once acquired, 3D models could easily be used to investigate the environmental quality of very dense areas, socio-economic patterns of occupation, to build urban landscape models, assessment on heat island effects, buildings demolitions or floods simulations to draw up strategic action plans. However, with no additional information, the 3D models generated from building footprints could only provide details regarding building topology, not much than that. With the bulk of social-media, and availability of faster and more powerfull computers, the concept of smart cities has broken the barrier of entertainment and rise new scentific questions. What could provide rich information about buildings and its facades? How could we automatic extract them? Even more, what if classified facade features had also a related geometry? This questions has surrounded the new advances in technologies regarding close-range acquirements and 3D mapping. A new era of sensors and plataform has guiding the research to a highly detailed mapping, and this work purpose to explore the advantages of these technologies not only to present an automatic methodology for detecting urban features, but also to understand the difficulties that still surround the complexity of such environments. 

%The task of mapping cities by autonomous operators was usually carried out by aerial optical images due to its scale and resolution, but new scientific questions have arisen and this has led research into a new era of high detailed data extraction. 

%For many years, using artificial neural models to solve complex problems such as automatic image classification was commonplace, owing much of their popularity to their ability to adapt in complex situations with minimal need of human intervention. In spite of that, their popularity declined in the mid-2000s, mostly due to the complex and time-consuming nature of their methods and workflows. However, new neural network architectures have brought back interest in their application as autonomous classifiers, especially in image processing.

\palavraschave{
	\palavrachave{3D urban mapping}
	\palavrachave{facade features}
	\palavrachave{deep-learning}
	\palavrachave{convolutional neural network}
	\palavrachave{structure-from-motion}
}

\end{resumo}
