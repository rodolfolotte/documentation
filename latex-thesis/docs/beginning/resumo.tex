\begin{resumo}
Urban environments are regions in which spectral and spatial variability are extremely high, with a huge range of shapes and sizes they also demand high resolution images for applications involving their study. Due to the fact that these environments can grow even more over time, applications related to their large-scale monitoring tend to rely on autonomous intelligent systems that, along with high-resolution images, can help and even predict everyday situations. In addition to the intelligent detection of these features, 3D representations of these environments have also been object of study to assist in the investigation of the environmental quality of very dense areas, occupational socioeconomic patterns, the construction of urban landscape models, building demolitions or flood simulations for evacuation plans and strategic delimitation, among countless others. For these aspects, the main objective of this study was to explore the advantages of such technologies, in order to present not only an automatic methodology for the detection and reconstruction of urban elements, but also to understand the difficulties that still surround the automatic mapping of these environments. Specifically we aimed: (i) To develop a routine of automatic classification of facade features in 2D domain, using a Convolutional Neural Network (CNN). (ii) Using the same images, obtain the facade geometry using Structure-from-Motion (SfM) and Multi-View Stereo (MVS) techniques. (iii) Evaluate the performance of the neural model for different urban scenarios and architectural styles. (iv) Evaluate the performance of the neural model in a real application in Brazil, whose architecture differs from the datasets used in the neural model training. (v) Classify the 3D model of the extracted facade using images segmented in 2D domain by the Ray-Tracing (RT) technique. In order to atempt that, the methodology was splited into 2D analysis (detection) and 3D (reconstruction). So in the first, a supervised CNN is used to segment terrestrial optical images of facades into six classes: roof, window, wall, door, balcony and shops. At the same time, the facade is reconstructed using the SfM/MVS technique, obtaining the geometry of the scene. Finally, the results of segmentation in both domains, 2D and 3D, are then merged by the Ray-Tracing technique, finally obtaining the 3D model classified. It is demonstrated that the proposed methodology is robust toward complex scenarios. The inferences made with the CNN neural model reached up to 93\% accuracy, and 90\% F1-score for most of the datasets used. For unknown scenarios, the neural model reached lower accuracy indexes, justified by the high differentiation of architectural styles. However, the use of deep neural models gives chances for new configurations and use with other deep architectures to improve results, especially for unsupervised models. Finally, the work demonstrated the autonomous capacity of a Convolutional Neural Network against the complexity of urban environments, in order to diversify between different styles of facades. Although there are improvements to be made regarding 3D classification, the methodology is consistent and allowed to combine state-of-the-art methods in the detection and reconstruction of urban elements, as well as providing support for new studies and projections on even more distinct scenarios.
\palavraschave{
	\palavrachave{3D urban mapping}
	\palavrachave{facade features}
	\palavrachave{deep-learning}
	\palavrachave{convolutional neural network}
	\palavrachave{structure-from-motion}
}

\end{resumo}
