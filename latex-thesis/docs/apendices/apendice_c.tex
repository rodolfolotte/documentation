\chapter{APPENDIX B - 3D MAPPING IN BRAZIL}\label{apendiceC}    
  \section{Poll about the use of 3D maps in Brazil}\label{apendiceC-poll}
  The questionary was made by email, in some cases, was not possible to reach some of the capitals either by email or phone. The detailed answers from Table \ref{poll-answers} can be seen in Table \ref{poll-answers2}.
  \begin{table}[!H]
    \renewcommand{\arraystretch}{1.6}
    \caption{Poll answers from Table \ref{poll-answers} in details.}
    \tiny \centering
    \rowcolors{2}{white}{gray!25}
    \begin{tabular}{lcp{11.5cm}}    
        %     \begin{tabular}L{1cm}L{0.4cm}L{4.6cm}L{2.3cm}L{4cm}}  
        \toprule
        \textbf{Capital} & \textbf{State}  & \textbf{Answer} \\ 
        \toprule
        Fortaleza & CE & \textbf{a)} The City Hall of Fortaleza is acquiring the 3D mapping by laser profiling. Currently, department does not use this data for studies.\par \textbf{d)} Planning is performed without any 3D map.\par \textbf{e)} Studies using the Z component.\\% & ouvidoria.seuma@fortaleza.ce.gov.br\\
        
        Vit�ria & ES & \textbf{a)} No.\par \textbf{d)} By 2D thematic maps, with additional database information, such as occupation, use, height, feedback, distribution of activities, commerce and urban services, public equipment, socioeconomic characteristics, among others.\par \textbf{e)} For example, Vit�ria has a map on the protection of the natural heritage of the municipality and the landscape. A 3D map would aid in reconciling territory occupation policies with the protection policy of this heritage, as well as advanced studies such as volumetric tests, among others. It can be an important tool in the studies of protection of the environmental patrimony, mangroves, hills and conservation units, allowing a reading of the constructed object of the city and its relation with the environmentally protected areas.\\% & clivia.leite@correio1.vitoria.es.gov.br\\
        
        Porto Alegre & RS & \textbf{a)} Yes, Digital Surface Model (DSM) and Digital Terrain Model (DTM).\par \textbf{b)} At the moment, in the identification of areas susceptible to geological, geotechnical and hydrological risks; in the estimation of the height of the buildings, for tax purposes. GIS system in Web platform, for internal queries.\par \textbf{e)} Support for decision making, confronting the planned environment with the built environment, impact and shading simulations, volumetric simulations. Volumetric studies.\\% & rodrigoml@smurb.prefpoa.com.br\\
        
        S�o Paulo & SP & \textbf{a)} Yes. DSM and DTM. In 2017, a new campaign using LiDAR sensors was performed and new 3D models will be produced.\par \textbf{b)} Workings such as the Strategic Master Plan, Zoning, Drainage Plan, among others. \par \textbf{c)} The DTM is used as a reference for other surveys, including planning and execution of some current aerial imaging stages. However, in projects and infrastructure, the effective use of these 3D digital technologies is still in the process of being disseminated to the technical staff.\par \textbf{d)} Although the 3D models are available, 2D is still predominantly used in urban planning activities.\par \textbf{e)} They incorporate technical features that are extremely useful for urban management.\\% & amlaurenza@prefeitura.sp.gov.br\\
        
        Belo Horizonte & MG & \textbf{a)} Yes. The execution of 3D modeling is carried out on demand in projects or urban plans that require analysis of modifications in the urban landscape.\par \textbf{b)} The 3D maps help in the visualization of impacts, in the increase of the constructive density, in the generation of obstructions of sightings, in the relations between equipment and urban infrastructures, among others.\par \textbf{c)} Volumetry studies are carried out for specific projects, such as regions of Urban Operations, Plans in Areas of Social Interest, Plans of Cultural Regions.\par \textbf{d)} Planning is already done with the eventual use of 3D modeling of the terrain and buildings.\par \textbf{e)} The 3D maps help in diagnosing the situation before and after the implementation of projects to evaluate the urban impacts of density and volumetry in parameters established in the plans.\\% & guilherme.vargas@pbh.gov.br\\
        
        Rio de Janeiro & RJ & \textbf{a)} The buildings are represented in two dimensions with possibility of elevation in 3D. Calculated density calculations for proposed zoning alterations, urban parameters and studies of impacts on the landscape.\par \textbf{b)} 3D simulations allow for technical discussion, improvement of proposals and better communication with managers, city councilors and the general population.\par \textbf{c)} Most of urban law plans made in the last five years have included 3D maps on strategic areas.\par \textbf{d)} In addition to the 3D studies, the urban planning uses information produced by both federal and state bodies such as IBGE, ISPE and also municipal, produced by the municipal authorities.\par \textbf{e)} Associating the relief with the buildings, it contributes to studies of insolation, heat islands, aimed at the natural and cultural landscape, simulations and urban parameters such as building heights and their distances to the other buildings and impacts on the environment.\\% & valeriahazan.pcrj@gmail.com\\
        
        Curitiba & PR & \textbf{a)} Yes. Partial maps, that is, not of the whole city, but of parts of it.\par \textbf{b)} Providing visualization of occupancy scenarios (for example, using all basic building potential or with extra potential acquisition); in studies of insolation and microclimate. Answered in the previous question.\par \textbf{e)} In addition to the current applications, these maps could be used, for example, in Neighborhood Impact Studies and project detailing.\\% & geoprocessamento@ippuc.org.br\\
        
        Recife & PE & \textbf{a)} Yes, the Urban Planning Dept. has georeferenced databased with vector and raster data from LiDAR campaings. Another tool that it is used is Google Earth Pro, which has the 3D model of the entire municipality of Recife.\par \textbf{b)} A 3D model allows us to manipulate, take measurements, visualize any interventions already built, and also simulate transformations from anywhere.\par \textbf{c)} Any analysis of morphology and occupation of the territory. These activities include analyzes of impact projects, simulation of changes in legislation and simulations, and studies of interventions. \\%& icps@recife.pe.gov.br\\
        \bottomrule
    \end{tabular}    
    \label{poll-answers2}
    \FONTE{Author's production.}
  \end{table}
    
  The contact of each professional responsible for the answers, can be found in Table \ref{contacts}.  
  \begin{table}[!ht]
  \renewcommand{\arraystretch}{1.6}
  \caption{Responsibles for the 3D mapping poll answers, sent to the Brazilian capitals infraestructure department (Table \ref{poll-answers2}, in Section \ref{3d-urban-brazil}).}
  \scriptsize \centering
  \rowcolors{2}{white}{gray!25}
    \begin{tabular}{L{2.2cm}L{1cm}L{4.8cm}L{4.7cm}}  
      \toprule
      \textbf{Capital} & \textbf{State}  & \textbf{Name} & \textbf{Email} \\ 
      \toprule
      Fortaleza & CE & Ouvidoria & ouvidoria.seuma@fortaleza.ce.gov.br\\      
      Vit�ria & ES & Clivia Leite Mendon�a & clivia.leite@correio1.vitoria.es.gov.br\\      
      Porto Alegre & RS & Rodrigo Marsillac Linn & rodrigoml@smurb.prefpoa.com.br\\      
      S�o Paulo & SP & Ana Maria A. Laurenza and Silvio C. L. Ribeiro & amlaurenza@prefeitura.sp.gov.br\\      
      Belo Horizonte & MG & Guilherme Pereira de Vargas & guilherme.vargas@pbh.gov.br\\            
      Rio de Janeiro & RJ & Val�ria Hazan & valeriahazan.pcrj@gmail.com\\      
      Curitiba & PR & Oscar Ricardo M. Schmeiske and Alessandro Dias & geoprocessamento@ippuc.org.br\\      
      Recife & PE & Tiago Henrique & icps@recife.pe.gov.br\\
      \bottomrule
    \end{tabular}
    \label{contacts}
    \FONTE{Author's production.}
  \end{table}
