\chapter{CONCLUSION}\label{chapter5}
\section{General conclusions}
Increasingly, the research regarding the facade feature extraction from complex structures, under a dynamic and hard-to-work environment (crowded cities) represent a new branch of research, with perspectives to the areas of technology, such as the concept of smart cities, as well as the areas of Cartography, toward to more detailed maps and semantized systems. In this study, an overview of the most common techniques was presented, as well as an introduction of instruments and ways of observing structural information through remote sensing data. Besides, was also presented a methodology to detect facade features by the use of a CNN, incorporating this detection to its respective geometry through the application of a SfM pipeline and ray-tracing analysis. 

The experiments are mainly focused in aspects such those aforementioned techniques and their computational capability in detecting facade features, regardless of architectural style, location, scale, orientation or color variation. All the images used in the training procedures underwent no preprocessing whatsoever, keeping the study area as close as possible as to what would be a common-user dataset (photos taken from the streets). 

In this sense, the edges of the acquired delineated features show the robustness of the CNN technique in segmenting any kind of material, in any level of brightness (shadow and occluded areas), orientation, or presence of pedestrians and cars. Considering that the values achieved for the individual datasets were above 90\%, it is concluded that the CNNs can provide good results for image segmentation in many situations. However, being a supervised architecture, the network has to pass through a huge training set, with no guarantees of good inputs, in order to get reliable inferences. When applied over unknown data, such as the experiment on the SJC data, it was noted that the neural network failed, except in regions where the facade features share similar characteristics, though such occasions were rare.

\subsection{Brazilian architectural styles and unreachable areas}
In Section \ref{map-brazil}, it is shown some of the difficulties encountered specifically in Brazil. In addition, in Chapter \ref{chapter4}, experiments were carried out on the extraction and reconstruction of features on Brazilian facades. As mentioned before, factors such as local culture, number of inhabitants and economy are determinant in local geometry issues, especially in poor regions whose civil construction is often made irregularly, over equally irregular and difficult access areas. Not only that, but the factors can also determine, for example, the geometry of city-centers, such as the presence of skyscrapers (e.g. S�o Paulo, Brazil; Changhai, China; and New York, United States) or commercial areas with buildings with a maximum of four floors (e.g. London, Germany, and France). Thus, studies involving the measurement of these factors could contribute as input for the development of 3D reconstruction techniques as presented in this document.

\section{Source-code: Usability, Licenses and Extension}\label{code-usability-and-extension}
References to the codes and experiments carried out in this study can be found in the Table \ref{tools-used}, in Appendix \ref{apendiceA}. Since the beginning of this work, libraries such as CGAL, PCL, VisualSfM, COLMAP, OpenCV, among others, have been used for tests and experiments throughout development. Of course, some of these libraries and software were discarded despite having a key role in refining the final methodology. Therefore, listing the technologies used, their sources, and other characteristics, is a way of guiding other works and contributing with related researches.

\subsection{Difficulties}
This work was entirely carried out with the aid of Open-Source tools, for the most part, under MIT and GPL licenses\footnote{Licenses of a public domain that could be extended and, in case of GPL, must be shared.}, which includes the non-commercial extension and free-to-use code. In the table \ref{tools-used}, in the Appendix \ref{apendiceA}, the tools used and their relation with the respective methodological step are presented.

In addition to the results archived in this study, every implementation was carried out in order to be expandable, interoperable, understandable, cohesive, public and easily accessible. Therefore, all methodological steps are available for consultation and public use at the GitHub\texttrademark~\cite{github}, where the links and respective explanations are available in Table \ref{tools-developed}. All the datasets used for this research were provided with any cost. In the same way, private tools that were essential for the development of the research were difficult to access or too expensive to acquire, most of the technologies used in this study were free, for example, the tools mentioned in Section \ref{code-usability-and-extension}.

Due to the technology approached in this study covering the state-of-the-art of classification and urban mapping solutions, the supply of training and courses is scarce. Likewise, in Brazil, there was a great difficulty in learning how to process LiDAR data during the first phases of this study. As we progressed, the operations involving SfM/MVS became more consistent, mainly, with the involvement and cooperation of multiple institutions (academic, military - national and international). For example, the DSG, in the revision of text regarding Brazilian standards and specifications for 3D geographic mapping in Brazil. The Department of Photogrammetry (IfP), at the University of Stuttgart, Germany, by offering disciplines of Computer Vision and Pattern Recognition. The Federal University of Paran� (UFPR), by their methodic assistance regarding the use of LiDAR in urban areas, among others. 

\section{Future prospects}
Identifying facade features under a great variety and arrangement make up these tasks still a scientific challenge whose tendency is to expand. The technologies to observe cities, such as sophisticated sensors, reconstruction and classification techniques, evolve as the numerous architectural styles change according to local culture and way of life. Moreover, it is essential to think that the multiplicity of architectural styles is not the only problem. Studies, such as those carried out at the MIT Center for Art, Science and Technology (CAST), Massachusetts Institute of Technology (MIT)\footnote{Video available at https://www.youtube.com/watch?v=vRfNbhyPPKs. Accessed \today.} and at Eidgen�ssische Technische Hochschule (ETH) Z�rich \cite{adriaenssens2016}, show, for example, that materials used in construction might become dynamic and therefore do not present a single static structure of a building. Urban occupation tends to evolve, which also demands that mapping techniques must both to follow the current architectural structures, as well as their eminent evolution.

As future prospects, to explore aspects such as the use of non-supervised models, separate tasks such as pre-classification of architectural styles, and mix different DL techniques to deal with specific scenarios, such as the chaotic arrangement of urban elements. Even it is the first study case, the methodology presented is highly dependent of the quality and number of images for training. The power of generalization in a neural network occurs as soon as the training set is large enough, as well as their resolutions. Besides, once this study has shown the robustness of CNN over complicated situations, we believe that efforts directed towards post-processing techniques could make the final 3D labeled model even more accurate.
