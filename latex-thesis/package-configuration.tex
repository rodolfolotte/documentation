\makeatletter
\def\BState{\State\hskip-\ALG@thistlm}
\makeatother

\makeatletter
\patchcmd{\@verbatim}
  {\verbatim@font}
  {\verbatim@font\scriptsize}
  {}{}
\makeatother

\newcommand*\makelabelcases[1]{\textit{Q.$#1$}:}
\SetEnumitemKey{question}
  {itemsep=0pt,align=left,leftmargin=\parindent,
   itemindent=!,before=\let\makelabel\makelabelcases}

\definecolor{myCyan}{RGB}{128, 255, 255}
\definecolor{myPurple}{RGB}{128, 0, 255}   
   
   
\makeatletter
\newread\pin@file
\newcounter{pinlineno}
\newcommand\pin@accu{}
\newcommand\pin@ext{pintmp}
% inputs #3, selecting only lines #1 to #2 (inclusive)
\newcommand*\partialinput [3] {%
  \IfFileExists{#3}{%
    \openin\pin@file #3
    % skip lines 1 to #1 (exclusive)
    \setcounter{pinlineno}{1}
    \@whilenum\value{pinlineno}<#1 \do{%
      \read\pin@file to\pin@line
      \stepcounter{pinlineno}%
    }
    % prepare reading lines #1 to #2 inclusive
    \addtocounter{pinlineno}{-1}
    \let\pin@accu\empty
    \begingroup
    \endlinechar\newlinechar
    \@whilenum\value{pinlineno}<#2 \do{%
      % use safe catcodes provided by e-TeX's \readline
      \readline\pin@file to\pin@line
      \edef\pin@accu{\pin@accu\pin@line}%
      \stepcounter{pinlineno}%
    }
    \closein\pin@file
    \expandafter\endgroup
    \scantokens\expandafter{\pin@accu}%
  }{%
    \errmessage{File `#3' doesn't exist!}%
  }%
}
\makeatother

\DeclareCaptionFont{white}{\color{white}}
\DeclareCaptionFormat{listing}{\colorbox[cmyk]{0.43, 0.35, 0.35,0.01}{\parbox{\textwidth}{\hspace{15pt}#1#2#3}}}
\captionsetup[lstlisting]{format=listing,labelfont=white,textfont=white, singlelinecheck=false, margin=0pt, font={bf,footnotesize}}

% \hypersetup{
%     colorlinks = true,
%     citecolor = {blue},
%     linkcolor = {blue},
%     linkbordercolor = {white}    
% }

\hypersetup{
    colorlinks,
    linkcolor={blue},
    citecolor={blue!50!black},
    urlcolor={blue!80!black}
}

\lstdefinestyle{bash}
{ 
    language=Bash,
    numbers=none,     
    keywordstyle=\color{green}\sf,    
    basicstyle=\scriptsize\ttfamily,
    rulecolor=\color{white},
    backgroundcolor=\color{gray},
    commentstyle=\color{gray},
    captionpos=b,    
    breaklines=true,    
    stringstyle=\color{red!60},
    morecomment=[l][\color{magenta}]{\#},
    identifierstyle=\color{green},  
    tabsize=1,
    title=\lstname,
    columns=flexible
}

\lstdefinestyle{java}
{ 
    language=Java,
    numbers=left, numberstyle=\tiny, numbersep=3pt,
    keywordstyle=\color{blue}\sf,  
    basicstyle=\scriptsize\ttfamily,
    rulecolor=\color{darkgray},
    backgroundcolor=\color{gray!10},    
    commentstyle=\color{gray},
    captionpos=b,    
    breaklines=true,    
    stringstyle=\color{red!60},
    morecomment=[l][\color{magenta}]{\#},
    identifierstyle=\color{green},  
    tabsize=2,
    keepspaces=true,  
    title=\lstname,
    columns=flexible
}

\lstdefinestyle{cpp}
{ 
    language=C,
    numbers=left, numberstyle=\tiny, numbersep=3pt,
    keywordstyle=\color{blue}\sf,  
    basicstyle=\scriptsize\ttfamily,
    rulecolor=\color{darkgray},
    backgroundcolor=\color{gray!10},    
    commentstyle=\color{gray},
    captionpos=b,    
    breaklines=true,    
    stringstyle=\color{red!60},
    morecomment=[l][\color{magenta}]{\#},
    identifierstyle=\color{green},  
    tabsize=2,
    keepspaces=true,  
    title=\lstname,
    columns=flexible
}

\lstset{escapechar=@,style=cpp}

\newcommand{\data}{/data/phd}
\newcommand{\documents}{/home/lotte/Documents}
\newcommand{\dropbox}{/home/lotte/Dropbox}
\newcommand{\figspath}{/home/lotte/design/phd/thesis} 
\newcommand{\bibpath}{bib/thesis-2} 

\newcommand{\horrule}[1]{\rule{\linewidth}{#1}}
\newcommand{\q}[1]{``#1''}
\newcolumntype{C}[1]{>{\centering\let\newline\\\arraybackslash}m{#1}}
\newcolumntype{L}[1]{>{\let\newline\\\arraybackslash}m{#1}}

\setlist[description]{leftmargin=\parindent,labelindent=\parindent}

\newcommand{\spheading}[2][10em]{% \spheading[<width>]{<stuff>}
  \rotatebox{90}{\parbox{#1}{\centering #2}}}
    
\newcommand*\daywidth{6cm}
\newcommand*\hourheight{1.2em}

% \newcommand*{\source}[2]{%
%   \caption[{#1}]{%
%     #1%
%     \\\hspace{\linewidth}%
%     \textbf{Source:} #2%
%   }%
% }
