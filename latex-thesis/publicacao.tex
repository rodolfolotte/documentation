\documentclass[
%SemVinculoColorido,
%SemFormatacaoCapitulo,
%SemFolhaAprovacao,
%SemImagens,
%CitacaoNumerica, %% o padr�o � cita��o tipo autor-data
%PublicacaoDissOuTese, %% (� tamb�m o "default") com ficha catal. e folha de aprova��o em branco. Caso tenha lista de s�mbolos e lista de siglas e abreviaturas retirar os coment�rios dos arquivos siglas.tex e abreviaturasesiglas.tex. Retirar tamb�m os coment�rios indicados nesse arquivo, nos includes
%PublicacaoArtigoOuRelatorio, %% texto sequencial, sem quebra de p�ginas nem folhas em branco
%PublicacaoProposta, 
%PublicacaoQualificacao, %% feito por rodolfo lotte (lotte@dsr.inpe.br)
%PublicacaoLivro, %% com cap�tulos
%PublicacaoLivro,SemFormatacaoCapitulo, %% sem cap�tulos
%english, portuguese %% para os documentos em Portugu�s com abstract.tex em Ingl�s
portuguese,english %% para os documentos em Ingl�s com abstract.tex em Portugu�s
,LogoINPE	%% comentar essa linha para fazer aparecer o logo do Governo
,CCBYNC	% as op��es de licen�a s�o: CCBY, CCBYSA, CCBYND, CCBYNC, CCBYNCSA, CCBYNCND e INPECopyright
]{tdiinpe}
%]{../../../../../iconet.com.br/banon/2008/03.25.01.19/doc/tdiinpe}

% PARA EXIBIR EM ARIAL TIRAR O COMENT�RIO DAS DUAS LINHAS SEGUINTES
%\renewcommand{\rmdefault}{phv} % Arial
%\renewcommand{\sfdefault}{phv} % Arial
% PARA PUBLICA��ES EM INGL�S:
% renomear o arquivo: abnt-alf.bst para abnt-alfportuguese.bst
% renomear o arquivo: abnt-alfenglish.bst para abnt-alf.bst

%\watermark{Review No. 1.0} 

\usepackage{rotating}
\usepackage{enumitem}
\usepackage{longtable}
\usepackage{hhline}
\usepackage{hyperref}
\usepackage{caption3}
\usepackage{array}
\usepackage[table]{xcolor}
\usepackage{textcomp}
\usepackage{graphicx}
\usepackage{lscape}
\usepackage{multirow}
\usepackage{dcolumn,longtable}
\usepackage{booktabs}
\usepackage{xcolor}
\usepackage{listings}
\usepackage{courier}
\usepackage{algorithm}
\usepackage[noend]{algpseudocode}
\usepackage{colortbl}
\usepackage{marvosym}
\usepackage{fontawesome}
\usepackage{fancyvrb}
\usepackage[pdftex]{graphicx}	
\usepackage{graphicx,subfigure}
\usepackage{url}
\usepackage{etoolbox}
\usepackage{calc}
\usepackage{lmodern}


\makeatletter
\def\BState{\State\hskip-\ALG@thistlm}
\makeatother

\makeatletter
\patchcmd{\@verbatim}
  {\verbatim@font}
  {\verbatim@font\scriptsize}
  {}{}
\makeatother

\newcommand*\makelabelcases[1]{\textit{Q.$#1$}:}
\SetEnumitemKey{question}
  {itemsep=0pt,align=left,leftmargin=\parindent,
   itemindent=!,before=\let\makelabel\makelabelcases}

\definecolor{myCyan}{RGB}{128, 255, 255}
\definecolor{myPurple}{RGB}{128, 0, 255}   
   
   
\makeatletter
\newread\pin@file
\newcounter{pinlineno}
\newcommand\pin@accu{}
\newcommand\pin@ext{pintmp}
% inputs #3, selecting only lines #1 to #2 (inclusive)
\newcommand*\partialinput [3] {%
  \IfFileExists{#3}{%
    \openin\pin@file #3
    % skip lines 1 to #1 (exclusive)
    \setcounter{pinlineno}{1}
    \@whilenum\value{pinlineno}<#1 \do{%
      \read\pin@file to\pin@line
      \stepcounter{pinlineno}%
    }
    % prepare reading lines #1 to #2 inclusive
    \addtocounter{pinlineno}{-1}
    \let\pin@accu\empty
    \begingroup
    \endlinechar\newlinechar
    \@whilenum\value{pinlineno}<#2 \do{%
      % use safe catcodes provided by e-TeX's \readline
      \readline\pin@file to\pin@line
      \edef\pin@accu{\pin@accu\pin@line}%
      \stepcounter{pinlineno}%
    }
    \closein\pin@file
    \expandafter\endgroup
    \scantokens\expandafter{\pin@accu}%
  }{%
    \errmessage{File `#3' doesn't exist!}%
  }%
}
\makeatother

\DeclareCaptionFont{white}{\color{white}}
\DeclareCaptionFormat{listing}{\colorbox[cmyk]{0.43, 0.35, 0.35,0.01}{\parbox{\textwidth}{\hspace{15pt}#1#2#3}}}
\captionsetup[lstlisting]{format=listing,labelfont=white,textfont=white, singlelinecheck=false, margin=0pt, font={bf,footnotesize}}

\hypersetup{
    colorlinks,
    linkcolor={blue},
    citecolor={blue!50!black},
    urlcolor={blue!80!black}
}

\lstdefinestyle{bash}
{ 
    language=Bash,
    numbers=none,     
    keywordstyle=\color{green}\sf,    
    basicstyle=\scriptsize\ttfamily,
    rulecolor=\color{white},
    backgroundcolor=\color{gray},
    commentstyle=\color{gray},
    captionpos=b,    
    breaklines=true,    
    stringstyle=\color{red!60},
    morecomment=[l][\color{magenta}]{\#},
    identifierstyle=\color{green},  
    tabsize=1,
    title=\lstname,
    columns=flexible
}

\lstdefinestyle{java}
{ 
    language=Java,
    numbers=left, numberstyle=\tiny, numbersep=3pt,
    keywordstyle=\color{blue}\sf,  
    basicstyle=\scriptsize\ttfamily,
    rulecolor=\color{darkgray},
    backgroundcolor=\color{gray!10},    
    commentstyle=\color{gray},
    captionpos=b,    
    breaklines=true,    
    stringstyle=\color{red!60},
    morecomment=[l][\color{magenta}]{\#},
    identifierstyle=\color{green},  
    tabsize=2,
    keepspaces=true,  
    title=\lstname,
    columns=flexible
}

\lstdefinestyle{cpp}
{ 
    language=C,
    numbers=left, numberstyle=\tiny, numbersep=3pt,
    keywordstyle=\color{blue}\sf,  
    basicstyle=\scriptsize\ttfamily,
    rulecolor=\color{darkgray},
    backgroundcolor=\color{gray!10},    
    commentstyle=\color{gray},
    captionpos=b,    
    breaklines=true,    
    stringstyle=\color{red!60},
    morecomment=[l][\color{magenta}]{\#},
    identifierstyle=\color{green},  
    tabsize=2,
    keepspaces=true,  
    title=\lstname,
    columns=flexible
}

\lstset{escapechar=@,style=cpp}

\newcommand{\data}{/data/phd}
\newcommand{\documents}{/home/lotte/Documents}
\newcommand{\dropbox}{/home/lotte/Dropbox}
\newcommand{\figspath}{\dropbox/design/phd/thesis} 
\newcommand{\bibpath}{\dropbox/phd/bib/thesis-2} 

\newcommand{\horrule}[1]{\rule{\linewidth}{#1}}
\newcommand{\q}[1]{``#1''}
\newcolumntype{C}[1]{>{\centering\let\newline\\\arraybackslash}m{#1}}
\newcolumntype{L}[1]{>{\let\newline\\\arraybackslash}m{#1}}

\setlist[description]{leftmargin=\parindent,labelindent=\parindent}

\newcommand{\spheading}[2][10em]{
  \rotatebox{90}{\parbox{#1}{\centering #2}}}
    
\newcommand*\daywidth{6cm}
\newcommand*\hourheight{1.2em}

%%%%%%%%%%%%%%%%%%%CAPA%%%%%%%%%%%%%%%%%%%%%%%%%%%%%%%%
%\serieinpe{INPE-NNNNN-TDI/NNNN} %% n�o mais usado

\titulo{Extra��o de estradas em imagens \emph{SAR} aerotransportadas: Uma abordagem baseada em Modelo de Contorno Ativo com o uso de semeador 
semiautom�tico}
\title{Roads center-axis extraction in airborne SAR images: An approach based in Active Contour Model with the use of semi-automatic seeding} %% 
\author{Rodolfo Georjute Lotte} %% coloque o nome do(s) autor(es)
\descriccao{Disserta��o de Mestrado do Curso de P�s-Gradua��o em Computa��o Aplicada, orientada pelo Dr. Sidnei Jo�o Siqueira Sant'Anna, Dra. Cl�udia Maria de Almeida, 
Dra. Corina da Costa Freitas e Dr. Jos� Demi�sio Sim�es da Silva}%, aprovada em dd de m�s por extenso de aaaa.}
\repositorio{aa/bb/cc/dd} %% reposit�rio onde est�o depositado este documento - na omiss�o, ser�o preenchido pelo SID
\tipoDaPublicacao{TDI}	%% tipo da publica��o (NTC, RPQ, PRP, MAN, PUD, TDI, TAE e PRE) na aus�ncia do n�mero de s�rie INPE, caso contr�rio deixar vazio
\IBI{xx/yy} %% IBI (exemplo: J8LNKAN8PW/36CT2G2) quando existir, caso contr�rio o nome do reposit�rio onde est�o depositado o documento

\date{2012}%ano da publica��o

%%%%%%%%%%%%%%%%%%%%%%%%%%VERSO DA CAPA%%%%%%%%%%%%%%%%%%%%%%%%%%%%%%%%%%%%%%%%%%%%%%%
\tituloverso{\vspace{-0.9cm}\textbf{\PublicadoPor:}}
\descriccaoverso{Instituto Nacional de Pesquisas Espaciais - INPE\\
Gabinete do Diretor (GB)\\
Servi�o de Informa��o e Documenta��o (SID)\\
Caixa Postal 515 - CEP 12.245-970\\
S�o Jos� dos Campos - SP - Brasil\\
Tel.:(012) 3945-6923/6921\\
Fax: (012) 3945-6919\\
E-mail: {\url{pubtc@sid.inpe.br}}
}

\descriccaoversoA{\textbf{\ConselhoDeEditoracao:}\\
\textbf{\Presidente:}\\
Dr. Gerald Jean Francis Banon - Coordena��o Observa��o da Terra (OBT)\\
\textbf{\Membros:}\\
Dra� Inez Staciarini Batista - Coordena��o Ci�ncias Espaciais e Atmosf�ricas (CEA)\\
Dra� Maria do Carmo de Andrade Nono - Conselho de P�s-Gradua��o \\
Dra� Regina C�lia dos Santos Alval� - Centro de Ci�ncia do Sistema Terrestre (CST)\\
Marciana Leite Ribeiro - Servi�o de Informa��o e Documenta��o (SID)\\
Dr. Ralf Gielow - Centro de Previs�o de Tempo e Estudos Clim�ticos (CPT)\\
Dr. Wilson Yamaguti - Coordena��o Engenharia e Tecnologia Espacial (ETE)\\
Dr. Hor�cio Hideki Yanasse - Centro de Tecnologias Especiais (CTE)\\
\textbf{\BibliotecaDigital:}\\
Dr. Gerald Jean Francis Banon - Coordena��o de Observa��o da Terra (OBT)\\
Marciana Leite Ribeiro - Servi�o de Informa��o e Documenta��o (SID)\\
%Jefferson Andrade Ancelmo - Servi�o de Informa��o e Documenta��o (SID)\\
%Simone A. Del-Ducca Barbedo - Servi�o de Informa��o e Documenta��o (SID)\\
%Deicy Farabello - Centro de Previs�o de Tempo  e Estudos Clim�ticos (CPT)\\
\textbf{\RevisaoNormalizacaoDocumentaria:}\\
Marciana Leite Ribeiro - Servi�o de Informa��o e Documenta��o (SID) \\
%Maril�cia Santos Melo Cid - Servi�o de Informa��o e Documenta��o (SID)\\
Yolanda Ribeiro da Silva Souza - Servi�o de Informa��o e Documenta��o (SID)\\
\textbf{\EditoracaoEletronica:}\\
Viv�ca Sant'Ana Lemos - Servi�o de Informa��o e Documenta��o (SID)\\
}
%%%%%%%%%%%%%%%%%%%FOLHA DE ROSTO

%%%%%%%%%%%%%%%FICHA CATALOGR�FICA
%% N�O PREENCHER - SER� PREENCHIDO PELO SID

\cutterFICHAC{Cutter}
\autorUltimoNomeFICHAC{Lotte, Rodolfo Georjute} %% exemplo: Fuckner, Marcus Andr�
\autorAbreviadoFICHAC {} %% N�o usado - deixar vazio
\tituloFICHAC{Extra��o de estradas em imagens \emph{SAR} aerotransportadas: Uma abordagem baseada em Modelo de Contorno Ativo com o uso de semeador 
semiautom�tico}
\instituicaosigla{INPE}
\instituicaocidade{S�o Jos� dos Campos}
\paginasFICHAC{\pageref{numeroDeP�ginasDoPretexto} + \pageref{LastPage}} %% n�mero total de p�ginas
%\serieinpe{INPE-00000-TDI/0000} %% n�o mais usado
\palavraschaveFICHAC{1.~\emph{Snakes}. 2.~extra��o de estradas 3.~semea��o semiautom�tica. 4.~imagem de radar. I.~\mbox{T��tulo}.} %% recomenda-se pelo menos 5 palavras-chaves - \mbox{} � para evitar hifeniza��o 
\numeroCDUFICHAC{000.000} %% n�mero do CDU 

% Nota da ficha (para TD)
\tipoTD{Disserta��o} % Disserta��o ou Tese
\cursoFA{Mestrado em Computa��o Aplicada}
\instituicaoDefesa{Instituto Nacional de Pesquisas Espaciais}
\anoDefesa{2012} % ano de defesa 
\nomeAtributoOrientadorFICHAC{Orientador}	% pode ser: Orientador, Orientadora ou Orientadores
\valorAtributoOrientadorFICHAC{Corina da Costa Freitas} % nome(s) completo(s)

%%%%%%%%%%%%%%%FOLHA DE APROVAÇAO PELA BANCA EXAMINADORA
\tituloFA{\textbf{ATEN��O! A FOLHA DE APROVA��O SER� INCLU�DA POSTERIORMENTE.}}
%\cursoFA{\textbf{}}
\candidatoOUcandidataFA{}
\dataAprovacaoFA{}
\membroA{}{}{}
\membroB{}{}{}
\membroC{}{}{}
\membroD{}{}{}
\membroE{}{}{}
\membroF{}{}{}
\membroG{}{}{}
\ifpdf

%%%%%%%%%%%%%%N�VEL DE COMPRESS�O {0 -- 9}
\pdfcompresslevel 9
\fi
%%% define em 80% a largura das figuras %%%
\newlength{\mylenfig} 
\setlength{\mylenfig}{0.8\textwidth}
%%%%%%%%%%%%%%%%%%%%%%%%%%%%%%%%%%%%%%%%%%%

%%%%%%%%%%%%%%COMANDOS PESSOAIS
\newcommand{\vetor}[1]{\mathit{\mathbf{#1}}}
 

\makeindex 

\begin{document}
    
    \maketitle
    
    % -- BEGINNING ------------------------------------------------------------------------------
    %%%%%%%%%%%%%%%%%%%%%%%%%%%%%%%%%%%%%%%%%%%%%%%%%%%%%%%%%%%%%%%%%%%%%%%%%%%%%%%%
% Ep�grafe %% opcional

\begin{epigrafe} %% insira sua ep�grafe abaixo; estilo livre

\hypertarget{estilo:epigrafe}{} %% uso para este Guia
 
\textit{\large``Imprima o que for verdade. Escreva de forma clara. Defenda-o at� o �ltimo suspiro''.}

\vspace{1cm}

\hspace{4cm} \emph{\textsc{Johann Wolfgang von Goethe}}\\\hspace{4cm} em \textsl{``Fausto''}, 1808

\end{epigrafe}

    \begin{dedicatoria}

\hypertarget{estilo:dedicatoria}{}

%\newcommand{\mytext}{A meus pais \textbf{Raul} e \textbf{Z�lia}, e � minha irm� \textbf{Cl�udia}}
\newcommand{\mytext}{To my parents Raul and Z�lia, and my sister, Cl�udia}

%\begin{comment}
%%% sugest�o de estilo
%\ifcalligra %% fonte calligra presente nas vers�es mais novas do MiKTeX (>= 2.4)
%  \calligra\Large \mytext %% exemplo usando estilo de fonte caligr�fica, caso haja
%\else
%	\itshape\Large \mytext 
%\fi
%\end{comment}

	\ifcalligra %% fonte calligra presente nas vers�es mais novas do MiKTeX (>= 2.4)
	  \calligra\Large \mytext %% exemplo usando estilo de fonte caligr�fica, caso haja
	\else
		\itshape\Large \mytext 
	\fi

\end{dedicatoria}

    \begin{agradecimentos}
  We create an imaginary show inside our heads, a show that involves more frustrations than the hope of reaching our goal. They are daily struggles, daily failures, so many blows that at first we thought we had lost all this time, but actually, it has been taking you to the highest point, to the point of seeing that small failures were all part of the same gear, the conquest. The energy that always moved me was the disbelief in my work, the lack of support, the disloyalty from part of this department, and the fear of not being able to conclude. But that gear was also \textquotedblleft lubricated\textquotedblright~so many times by faith, family, innumerable friends, places, travels, and so many other events that made things a little less painful. And it is for these people that I toast and dedicate this work today. Thank you to look with me this brand new horizon.    
  
  During my PhD, I had the chance to experience the guidance, friendship, support, criticism, and love of incredible people. Everyone who shared this energy played key roles in my evolution and what I am today. Above all, God and Nossa Senhora da Aparecida. Faith and science, the most famous paradox. Lord did not give me the thesis or health, but faith on you will always be the fuel of my determination and hope.
  
  A special thanks to the great sponsors of this work, my parents, Raul and Z�lia, and my sister, Claudia. I can not imagine this work, or anything else, being done without your cooperation or presence. Thanks for everything.
  
  To the great INPE, which has hosted me since the master's degree in Computer Applied (2010), to its inspiring infrastructure and environment. To the eternal friends of the postgraduate class in Remote Sensing of 2014, with affection to my friends Aline Jacon, Ana Pessoa, Bruna Pechini, Cesare Neto, David Fran�a, Denis Mariano, Henrique Cassol, Jaidson Becker, Jo�o Felipe , Jo�o Bosco, Ranieli dos Anjos, Sacha Siani and all others who were with me. My friends from Hannover, in special Alexander Schunert, Lukas Schack and Joachim Niemeyer. To my dear friends and advisors Dr. Luiz Arag�o and Dr. Yosio Shimabukuro, for having taken this work already underway, for believing, for cooperation and friendship until the end. Dr. Elisabete Moraes, my academic mother and great friend. Thank you for embracing the cause so many times, for understanding and for struggling with me. This work is a special dedication to you.
    
  To the dear friends of the Institute of Photogrammetry (IfP), University of Stuttgart, Germany: Gottfried Mandlburger, Luka Jurjevic, Mateusz Karpina, Patrick Tutzauer, Stefan Schmohl, Dominik Laupheimer, Martina Kroma, Lavinia Runceanu, Chia-Hsiang, Ke Gong, Prof. Dr. Michael Cramer and, in particular, Prof. Dr. Uwe S�rgel and Prof. Dr. Norbert Haala, for believing in my work, for the support, reception and follow-up of all my work during the time I was in Stuttgart.
    
  My co-work friends, at Brazilian Space Weather Monitoring Program (EMBRACE/INPE): Dr. Cristiano Wrasse, Dr. Clezio De Nardin, Fau�z and D�bora, for the enormous collaboration, infrastructure and support in this journey. Undoubtedly, the valuable lessons and experience as an Analyst and Developer in this department have brought great professional projections and a new insight into practical aspects in problem solving.
   
  Finally, to the Coordination for the Improvement of Higher Education Personnel (CAPES) and the National Council for Scientific and Technological Development (CNPq), thank you for the financial support in my first year of doctorate, and for the great opportunity and teachings I had at the University of Stuttgart (IfP-Stuttgart) - grant PDSE, Process No. 88881.132115/2016-01. This great opportunity was undoubtedly a very important piece in the conclusion of this work. The Army Geographic Service Directorate (DSG), for their support for cartographic specifications regarding 3D mapping in Brazil. To the friend George Longhitano, G-Drones, for the cooperation in acquiring and providing the data for our experiments.
    
  Nobody inside this institute was a motivation for me more than myself. No one will motivate you more than your own determination and courage. This was the biggest win of my life, which will be small compared to my brand new challenges.
  
\end{agradecimentos}


         
    \begin{resumo}
Urban environments are regions in which spectral and spatial variability are extremely high, with a huge range of shapes and sizes they also demand high resolution images for applications involving their study. Due to the fact that these environments can grow even more over time, applications related to their large-scale monitoring tend to rely on autonomous intelligent systems that, along with high-resolution images, can help and even predict everyday situations. In addition to the intelligent detection of these features, 3D representations of these environments have also been object of study to assist in the investigation of the environmental quality of very dense areas, occupational socioeconomic patterns, the construction of urban landscape models, building demolitions or flood simulations for evacuation plans and strategic delimitation, among countless others. For these aspects, the main objective of this study was to explore the advantages of such technologies, in order to present not only an automatic methodology for the detection and reconstruction of urban elements, but also to understand the difficulties that still surround the automatic mapping of these environments. Specifically we aimed: (i) To develop a routine of automatic classification of facade features in 2D domain, using a Convolutional Neural Network (CNN). (ii) Using the same images, obtain the facade geometry using Structure-from-Motion (SfM) and Multi-View Stereo (MVS) techniques. (iii) Evaluate the performance of the neural model for different urban scenarios and architectural styles. (iv) Evaluate the performance of the neural model in a real application in Brazil, whose architecture differs from the datasets used in the neural model training. (v) Classify the 3D model of the extracted facade using images segmented in 2D domain by the Ray-Tracing (RT) technique. In order to atempt that, the methodology was splited into 2D analysis (detection) and 3D (reconstruction). So in the first, a supervised CNN is used to segment terrestrial optical images of facades into six classes: roof, window, wall, door, balcony and shops. At the same time, the facade is reconstructed using the SfM/MVS technique, obtaining the geometry of the scene. Finally, the results of segmentation in both domains, 2D and 3D, are then merged by the Ray-Tracing technique, finally obtaining the 3D model classified. It is demonstrated that the proposed methodology is robust toward complex scenarios. The inferences made with the CNN neural model reached up to 93\% accuracy, and 90\% F1-score for most of the datasets used. For unknown scenarios, the neural model reached lower accuracy indexes, justified by the high differentiation of architectural styles. However, the use of deep neural models gives chances for new configurations and use with other deep architectures to improve results, especially for unsupervised models. Finally, the work demonstrated the autonomous capacity of a Convolutional Neural Network against the complexity of urban environments, in order to diversify between different styles of facades. Although there are improvements to be made regarding 3D classification, the methodology is consistent and allowed to combine state-of-the-art methods in the detection and reconstruction of urban elements, as well as providing support for new studies and projections on even more distinct scenarios.
\palavraschave{
	\palavrachave{3D urban mapping}
	\palavrachave{facade features}
	\palavrachave{deep-learning}
	\palavrachave{convolutional neural network}
	\palavrachave{structure-from-motion}
}

\end{resumo}

    %%%%%%%%%%%%%%%%%%%%%%%%%%%%%%%%%%%%%%%%%%%%%%%%%%%%%%%%%%%%%%%%%%%%%%%%%%%%%%%%
% ABSTRACT
\begin{abstract}
\hypertarget{estilo:abstract}{} %% uso para este Guia
      The research involving computational methods for the extraction of roads has intensified in the last two decades. The procedure is usually performed through the analysis of images acquired by optical sensors or radar (Radio Detection and Ranging). The great advantage in the usage of radar images is the possibility of survering areas often covered by clouds, since the imaging by active sensors is independent from atmospheric conditions in the region of interest. The cartographic mapping based on these images is often done manually, requiring considerable time and effort from the interpreter. There are currently many studies involving the extraction of roads by means of automatic or semi-automatic approaches, however, each of them has different solutions for different problems, making this task a scientific issue still open. Among the advantages of using radar images, one can mention the acquistion of images regardless of atmospheric and ilumination conditions, besides the possibility of surveying regions where the terrain is hidden by the vegetation canopy, among others. This work comprises the semi-automatic generation of seed points located close to the features of interest by means of Self-Organizing Maps (SOM). We then employ an active contour method (Snakes) for the extraction of the roads centre-axis in a synthetic aperture radar (SAR) airborne image. The obtained results were evaluated as to their quality with respect to perfection, correction, and redundancy.
\end{abstract}



    \includeListaFiguras
    \includeListaTabelas 
    \begin{abreviaturasesiglas}
2D &--& Bi-Dimensional\\
3D &--& Three-Dimensional\\
AC &--& Auto-Context\\
ACon &--& Active Contour\\
AE &--& Autoencoder\\
ANAC &--& National Civil Aviation Agency\\
ANN &--& Artificial Neural Network\\
BA &--& Bundle Adjustment\\
CASA &--& Civil Aviation Safety Authority\\
CNN &--& Convolutional Neural Network\\
COLLADA &--& COLLAborative Design Activity\\
CONCAR &--& National Commission of Cartography\\
CPU &--& Central Processing Unit\\
CVMS &--& Clustering Views for Multi-View Stereo\\
DL &--& Deep Learning\\
DOBSS &--& Distinctive Order Based Self-Similarity\\
DSG &--& Geographic Service Directorate\\
DSM &--& Digital Surface Model\\
DTM &--& Digital Terrain Model\\
EASA &--& European Aviation Safety Agency\\
FP &--& False Positive\\
FN &--& False Negative\\
FC &--& Fully Connected\\
FCN &--& Fully Connected Network\\
GAP &--& Gradient Accumulation Profile\\
GPU &--& Graphical Processing Unit\\
HOG &--& Histogram of Oriented Gradient\\
IaaS &--& Infrastructure as a Service\\
IM &--& Image-Matching\\
KML &--& Keyhole Markup Language\\
Laser &--& Light Amplification by Stimulated Emission of Radiation\\
LiDAR &--& Light Detection And Ranging\\
LoD &--& Level of Detail\\
MBA &--& Multicore Bundle Adjustment\\
MGCM &--& Multi-Photo Geometrically Constrained Matching\\
ML &--& Machine Learning\\
MLP &--& Multi-Layer Perceptron\\ 
MRF &--& Markov Random Field\\
MS &--& Mean-Shift\\
MVS &--& Multi-View Stereo\\
NC &--& Normalized Cuts\\
NCC &--& Normalized Cross Correlation\\
OGC &--& Open Geospatial Consortium\\
PMVS &--& Path-based Multi-View Stereo\\
PC &--& Point Cloud \\
PSI &--& Persistent Scatterer Interferometry \\
Radar &--& Radio Detection And Ranging\\
ReLU &--& Rectified Linear Unit\\
RR &--& Recurrent Network\\
RT &--& Ray-Tracing\\
RVR &--& Reducing View Redundancy\\
SAR &--& Synthetic Aperture Radar\\
SfM &--& Structure from Motion\\
SGM &--& Semi-Global Matching\\
SIFT &--& Scale-Invariant Feature Transform\\
SJC &--& S�o Jos� dos Campos\\
SLAM &--& Simultaneous Localization and Mapping\\
TP &--& True Positive\\
TN &--& True Negative\\
UAV &--& Unmanned Aerial Vehicle\\
\end{abreviaturasesiglas}

    \begin{simbolos}
$f$ &--& Estimated camera focal length\\
$f_x$ &--& Estimated camera focal length coordinate in $x$\\
$_y$ &--& Estimated camera focal length coordinate in $y$\\    
$c_x$ &--& Principal point offset coordinate in $x$\\
$c_y$ &--& Principal point offset coordinate in $y$\\    
$b$ &--& Estimated skew coefficient\\    
$s_x$ &--& Pixel size coordinate in $x$\\
$s_y$ &--& Pixel size coordinate in $y$\\    
$u$ &--& Pixel coordinate axis $u$ at image coordinate system\\
$v$ &--& Pixel coordinate axis $v$ at image coordinate system\\    
$K$ &--& Camera intrinsic parameters\\    
$R$ &--& Rotation matrix\\
$\omega$ &--& $R_X$\\       
$\phi$ &--& $R_Y$\\
$\kappa$ &--& $R_Z$\\
$T$ &--& Translation matrix\\        
$P$ &--& Camera projection matrix\\
$C$ &--& Center of projection (camera location)\\     
$\zeta$ &--& Radial distortion coefficient\\
$\xi$ &--& Tangential distortion coefficient\\
$\Theta$ &--& Tangential and radial distortion contribution    
\end{simbolos}

    
    \includeSumario
    \inicioIntroducao
    % -------------------------------------------------------------------------------------------
    
    % -- CHAPTERS -------------------------------------------------------------------------------
    \chapter{INTRODUCTION}\label{introducao}
A 3-dimensional (3D) representation of cities became a common term in the last decade \cite{demir2012}. What was once considered an alternative for visualization and entertainment, has become a powerful instrument of urban planning \cite{kolbe2009, stoter2011}. The technology is now well-known in most of the countries on the European continent, such as Switzerland \cite{swisstopo2010}, England \cite{accucities, vertex}, and Germany \cite{virtualcities, aringer2014, kruger2012, dollner2006}, also being commercially popular in North America, where many leading companies and precursor institutions reside. However, the semantic 3D mapping with features and applicability that go beyond the visual scope, is still considered a novelty in many other countries. In Brazil, according to the present study (see Section \ref{map-brazil}), the use of volumetric information as a resource for strategic and management planning is reduced to few cities (see Table \ref{poll-answers}).

%\cite{batty2000} categorize into twelve different areas that are already benefited by the use of 3D models in their urban planning.
%Map buildings details is considered an important task in Cartography \cite{thill2011}. 
According to a recent survey \cite{biljecki2015a}, approximately thirty real applications with the use of 3D urban models have been reported, ranging from environmental simulations, support of planning, cost reduction in modeling and decision making \cite{yang2016, truong2014}. Understanding the principles that establish the organization of such environment, as well as its dynamism, requires a structural analysis between its objects and geometry \cite{lafarge2015}. Therefore, reproducing the maximum of its geometry and volume allows studies such as the estimation of solar irradiance on rooftops \cite{biljecki2015b}, as well as the determination of occluded areas \cite{eicker2015, jochem2009}, in analyzing hotspots for surveillance cameras \cite{yaagoubi2015}, Wi-Fi coverage \cite{lee2015}, in the urbanization and planning of green areas \cite{tooke2011, ahmad2003}, in evacuation plans in case of disasters \cite{kwan2005}, among others. 
% Mapear detalhes de edifica��es � considerado de extrema import�ncia em prop�sitos cartogr�ficos, al�m de auxiliar no planejamento estrat�gico das cidades \cite{thill2011}. Segundo levantamento recente \cite{biljecki2015a}, aproximadamente trinta aplica��es reais envolvendo o uso de modelos 3D urbanos t�m sido reportadas, desde simula��es ambientais � suporte ao planejamento e tomada de decis�es. \citeonline{batty2000} categorizam em doze diferentes �reas que j� se beneficiam com uso de modelos 3D. Entender os princ�pios que estabelece a organiza��o desse ambiente, bem como sua din�mica, requer an�lise das rela��es estruturais entre seus objetos e formas \cite{lafarge2015}. Reproduzir, portanto, o m�ximo de sua geometria e volume computacionalmente, permite estudos como a estimativa de irradia��o solar em telhados \cite{biljecki2015b}, bem como na determina��o de �reas oclusas \cite{eicker2015, jochem2009}, da mesma forma para an�lises envolvendo a visibilidade para instala��o de c�meras \cite{yaagoubi2015}, na urbaniza��o e planejamento de �reas verdes \cite{tooke2011, ahmad2003},  no estudo da cobertura de sinais Wi-Fi \cite{lee2015}, em planos de evacua��o em casos de desastres \cite{kwan2005}, entre outros que completam a gama de alternativas.

Representing cities digitally exactly as they look like in the real world, was considered, for many years, mostly an entertainment application, rather than Cartography. With the appearance of LiDAR (Light Detection and Ranging) \cite{vosselman2001} and the Structure-from-Motion (SfM) and Multi-View Stereo (MVS) workflows \cite{snavely2006}, brought to real the structural urban mapping. Even though the data was extremely accurate, the campaigns in mid-2010 were mostly made by airplanes, which fostered large-scale 3D reconstructions, in which buildings can be accurately represented with their rooftops, occupation, area, height or volume characteristics \cite{salehi2017}. With this remarkable stage, today, new branches of research try not to faithfully represent the scene, but to add knowledge to it, increasingly toward semantic cities\footnote{The term \textquotedblleft semantic\textquotedblright, in this case, consists in the interpretation of a certain unknown information in something readable. Once this set of unknown infomation is interpreted, it could be confront to another set of interpreted information, studying their relationship and, then, make any kind of decision.}, where the nature of object is known and the relationship among them could be investigated.

%Na �ltima d�cada, houve um amadurecimento quanto a pesquisa para a reconstru��o 3D de cidades em larga escala, tal que o ponto crucial restringe-se em adquirir representa��es 3D de edif�cios e suas caracter�sticas de cobertura (telhado), ocupa��o, �rea, altura ou volume \cite{salehi2017, REF}. A medida em que h� um crescimento tecnol�gico de sensores, naturalmente, h� tamb�m mudan�as positivas na qualidade das cenas imageadas. Por conseguinte, novos ramos na pesquisa s�o criados e explorados de forma a estabelecer novos dom�nios e compreens�o do espa�o urbano \cite{demir2012, REF}. Recentemente, e, portanto, madura, uma nova demanda das �reas de Fotogrametria e Sensoriamento Remoto est� reconduzindo a pesquisa � an�lises mais aprofundadas desses objetos urbanos, em que usufruem das ent�o mencionados levantamentos �pticos, para analisar detalhes de suas fachadas e, ent�o, adquirir automaticamente mapas mais detalhados.

%During data collection, two main aspects are essential: scale and details. A data acquisition has been mainly performed by laser scanners (Light Amplification by Simulated Emission of Radiation) or Optical images with high spatial resolution (e.g. \footnote{Resolution are functions of flight altitude, sensors, topography, among others} $\geq$ 300 pts/m$^2$, and $\leq$ 0.5 cm Ground Sample Distance (GSD), respectively), on board of long and close-range platforms \cite{haala2010}. The vertical accuracy provide by laser scanners made them a popular instrument for topography purposes \cite{rottensteiner2005, blais2004}. However, a laser scanner or commonly referred as LiDAR (Ligth Detection And Ranging), still have high costs, especially for facade analysis, where close-range imaging is frequently required.
%Durante o levantamento de informa��es estruturais, dois principais aspectos s�o considerados: escala e detalhamento. A aquisi��o de dados vem sendo realizado por meio de sensores laser (\emph{Light Amplification by Simulated Emission of Radiation}) ou �ptico com elevada resolu��o espacial (p.e.\footnote{Valores de resolu��o s�o fun��es da altitude de voo, sensores, topografia, entre outros. Podendo se alterar conforme plano de voo utilizado.} $\geq$ 300 pts/m$^2$, e $\leq$ 0.5 cm GSD\footnote{Do ingl�s \emph{Ground Sample Distance}. Representa a resolu��o da imagem, a rela��o entre sua unidade m�nima (\emph{pixel}) e objeto imageado. O valor de GSD corresponde a dimens�o m�nima que cada objeto � representado dentro de um \emph{pixel}.}, respectivamente), abordo de plataformas de longo ou curto alcance \cite{haala2010}. O n�vel de acur�cia vertical tornou popular a utiliza��o de laser \emph{scanner} para fins topogr�ficos \cite{rottensteiner2005, blais2004}. A utiliza��o de laser \emph{scanner} ou, comumente referenciado, LiDAR (\emph{Ligth Detection And Ranging}), possui ainda custos elevados, sobretudo, no detalhamento de fachadas, em que o imageamento de curto alcance (fixo ou m�vel) quase sempre � requerido.  

%It consists of two groups of devices whose final product tends to differ mainly by geometric precision, although this difference has decreased in recent years \cite{lafarge2015, haala2015}. While laser sensors have been providing high levels of vertical accuracy, optical sensors complements with rich information, flexibility, low cost, and affordability. In addition, a use of optical and photogrammetry has been a trend in the literature on a 3D urban reconstruction \cite{haala2015}. In large-scale, oblique cameras have been widely used for 3D modeling, precisely because they present a certain advantage in complex scenarios, which the spectral and spatial variation are maximized \cite{zhu2015}.
% Trata-se de dois segmentos cujo produto final tende a diferenciar-se principalmente pela precis�o geom�trica, embora essa diferen�a tenha se atenuado durante os �ltimos anos \cite{lafarge2015, haala2015}. Enquanto sensores laser vem proporcionando n�veis elevados de acur�cia vertical, sensores �pticos complementam pela riqueza de informa��es espectrais, flexibilidade e acessibilidade financeira. Ademais, a utiliza��o de imagens �pticas e t�cnicas em fotogrametria tem mostrado ser uma tend�ncia na literatura quanto a reconstru��o 3D de cidades \cite{haala2015}. Em larga escala, sensores �pticos com visadas obliquas tem sido amplamente utilizados na modelagem 3D, justamente por apresentarem certa vantagem quanto a objetos cuja varia��o espectral e espacial s�o elevadas, resultando em modelos 3D mais completos e sofisticados \cite{zhu2015}. 

%Em larga escala, representa��es das edif�cios restringe-se no detalhamento de sua altura, caracter�sticas de cobertura (telhado), ocupa��o, �rea e volume d\cite{salehi2017}. A medida em que h� um crescimento tecnol�gico de sensores, naturalmente, h� tamb�m mudan�as positivas na qualidade das cenas imageadas. Por conseguinte, novos ramos na pesquisa s�o criados e explorados de forma a estabelecer novos dom�nios e compreens�o do espa�o urbano \cite{demir2012}. Recentemente, e, portanto, madura, uma nova demanda das �rea s de Fotogrametria e Sensoriamento Remoto est� reconduzindo a pesquisa � an�lises mais aprofundadas desses objetos, em que usufruem dos ent�o mencionados levantamentos �pticos, para analisar detalhes de suas fachadas e adquirir automaticamente mapas mais detalhados.

%Na �ltima d�cada, houve um amadurecimento quanto a pesquisa para a reconstru��o 3D de cidades em larga escala, tal que o ponto crucial restringe-se em adquirir representa��es 3D de edif�cios e suas caracter�sticas de cobertura (telhado), ocupa��o, �rea, altura ou volume \cite{salehi2017, REF}. A medida em que h� um crescimento tecnol�gico de sensores, naturalmente, h� tamb�m mudan�as positivas na qualidade das cenas imageadas. Por conseguinte, novos ramos na pesquisa s�o criados e explorados de forma a estabelecer novos dom�nios e compreens�o do espa�o urbano \cite{demir2012, REF}. Recentemente, e, portanto, madura, uma nova demanda das �reas de Fotogrametria e Sensoriamento Remoto est� reconduzindo a pesquisa � an�lises mais aprofundadas desses objetos urbanos, em que usufruem das ent�o mencionados levantamentos �pticos, para analisar detalhes de suas fachadas e, ent�o, adquirir automaticamente mapas mais detalhados.

In this sense, acquiring knowledge from remotely sensed data was always a permanent problem for Computer Vision and Pattern Recognition communities, which basically have the mission of interpreting huge amounts of data automatically. Until mid-2012, extracting any kind of information from images would require methodologies that would certainly not fully solve the problem, in many cases, only part of it. However, with the resurgence of Machine Learning (ML) technique in 2012 \cite{krizhevsky2012}, built on top of the original concept from 1989 \cite{lecun1989}, has changed the way to interpret images due its high accuracy and robustness on complex scenarios. The respectively ML concept called Convolutional Neural Network (CNN) has enormous potential for interpretation, specially when dealing with large amount of data. In Remote Sensing, has also been used to detect urban objects \cite{zheng2015, segnet, teichmann2016} with high quality inferences.

During many years, however, the use of artificial neural models to solve complex problems such as automatic image classification was common, getting popular specially to adapt in complex situations with no human interferences. Still, the methods or workflows available that time were too difficult to use considering the results it could provide, making its popularity lower in mid-2000. Recently, new neural networks architectures brought back the interest by autonomous classifiers, especially those for image classification. CNN has been a trend for pixelwise image segmentation, showing an extreme flexibility to detect and classify any kind of object even in scenarios where humans could not perceive.

Identifying simple facade features such as doors, windows, balconies, and roofs might be a tough task due to the infinite variations in shape, material compositions, and an unpredictable possibility of occlusions. That means not only a good method would be required, but the addition of another variable, such as geometry, should be used to improve class separability. A new demand in the areas of Photogrammetry and Remote Sensing is leading the research to further analysis of these urban objects, in which takes advantage of the aforementioned optical campaigns, such as in \citeonline{bodis2017, hayko2014, teboul2011, gadde2017}, to acquire the object geometries on a low-cost and simple-to-use manner, which is the use of common cameras and minimal knowledge for acquiring.

% Identificar caracter�sticas simples como quantidades e composi��es de portas, port�es, janelas, sacadas, beirais, e telhados, n�o s� melhoram os variados diagn�sticos de simula��es (como em \ref{REF-em-filippo}), como tamb�m permitem a redu��o de custos em projetos de modelagem, planejamento, monitoramento por c�meras e outros \cite{yang2016, truong2014}. Metodologias sofisticadas exploram a utiliza��o de m�todos como Campos Aleat�rios Condicionais (do ingl�s \emph{Conditional Random Fields} - CRF)\cite{liu2016, gadde2017} e Redes Neurais Convolutivas (do ingl�s \emph{Convolutional Neural Network} - CNN)\cite{zheng2015, segnet, } para classifica��o de fei��es de fachadas em imagens �pticas terrestres\footnote{Outros termos comuns na literatura s�o: \emph{street-side images} ou \emph{ground view images}.} como pr�-etapa � an�lise de sua geometria por malhas triangulares (\emph{mesh}) \cite{bodis2017, hayko2014, teboul2011, gadde2017}. Tamb�m promissora e com resultados que transmitem certa simplicidade, a t�cnica de Gram�tica de Forma (do ingl�s \emph{Shape Grammar} (SG)) vem sendo utilizada na modelagem de estruturas complexas ao unir primitivos geom�tricos � opera��es aritim�ticas, que a partir de premissas de simetria, conseguem inferir quanto aos demais elementos n�o imageados \cite{becker2009, teboul2011, vangool2013}. Trabalhos com foco na an�lise geom�trica, como os abordados em \cite{lafarge2015, verdie2015}, vem mostrando evolu��o quanto a m�todos de classifica��o em \emph{mesh}s, bem como procedimentos de refinamento e melhor representabilidade do modelo 3D. 

%Cen�rios urbanos possuem alta variabilidade espectral e espacial, e por se tratar de cen�rios "vivos", h� tamb�m a din�mica da mudan�a como presen�a de carros, ve�culos, constru��o civil, e materiais diversos que obstruem as edifica��es de qualquer m�todo que tente extrair informa��es uteis das edifica��es. Considerando essas dificuldades e o fato de que hoje poucos experimentos com cen�rios complexos t�m sido reportado, apresentamentos nossa metodologia 
Urban environments used to have high spectral and spatial variability, because they are dynamic scenarios, which means that not only the presence of cars, vegetation, vehicles and pedestrians are aggravating the extraction of information, but also the constant actions of man on urban elements. But not all cities are that complex. One city could present a better geometry when compared to another in terms of architectural styles, for example, the streets of Manhattan (wide), in United States, and the streets of Hong Kong (narrow), in China. In addition, suburbs use to have less traffic than city-centers, and that also affects the extraction. The term \textquotedblleft complex\textquotedblright$~$in this study refers to images where no preprocessing is performed, no cars are removed, no trees are cut off to benefit the imaging, no house or street was chosen beforehand, only images representing the perfect register of a real chaotic scenario were taken.

Considering these difficulties and the fact that today only a few experiments with complex scenarios have been carried out, a methodology using ground images\footnote{Also referenced here as street-side images.} is used, since they can provide all the facade details that aerial imagery might not be able to \cite{musialski2013}. The purpose here, is to delineate interest regions of facade images and assign each of them to a particular semantic label: roof, wall, window, balcony, door, and shop. Right after, these features are used to associate them onto their respective geometries. In order to detect these features, 6 datasets with distinct architectural styles are used as training samples to a CNN model. Once trained, the artificial knowledge generated for each dataset is tested to an unknown scene in Brazil. The facade geometry is then extracted through the use of a SfM/MVS pipeline, which is finally labeled by ray-tracing analysis according to each segmented image.

% Matheus:
%- A varia��o espectral intraespec�fica, ao n�vel de dossel, de esp�cies arb�reas da Floresta Estacional Semidecidual � significativamente menor do que a interespec�fica;
% - A exatid�o de classifica��o de esp�cies arb�reas da Floresta Estacional Semidecidual est� diretamente relacionada � diferen�a entre a variabilidade intra e interespec�fica;
% - A resposta espectral ao n�vel de dossel de esp�cies arb�reas da Floresta Estacional Semidecidual pode ser simulada de forma acurada por modelagem de transfer�ncia radiativa em tr�s dimens�es, a ponto de n�o alterar a variabilidade intra e interespec�fica original das esp�cies.

% Uma hip�tese � uma afirma��o que pode ser desafiada. Como tal, uma hip�tese de trabalho � uma frase que possibilita questionar o "Como?", "De que modo?" e o Por qu�? de algo. O objetivo de uma hip�tese � aclarado com a quest�o "Para qu�?"

% Em resumo, uma hip�tese de trabalho deve ser: uma afirma��o; simples; sujeita � nega��o.
%	Afirma��o: uma hip�tese n�o � uma pergunta, uma hip�tese � uma afirma��o sobre algo.
%	Simples: uma boa hip�tese � escrita em linguagem simples de maneira a expressar exatamente o que est� em jogo.
%	Sujeita � nega��o: uma hip�tese deve poder ser negada. Caso seja imposs�vel estabelecer a sua nega��o dificilmente ser� considerada uma hip�tese.[1]
\section{Hypotheses}
This work is based on the following hypotheses:    
\begin{itemize}      

    \item The volumetry of buildings, as well as their facade features (e.g. roof, windows, balconies, wall and doors), can be accurately extracted through optical images and SfM/MVS technique;           
    \item Facade features can be automatically detected with CNN even under complex scenarios with no preprocessing need;    
    \item The geometric quality of the 3D model, as well as the quality of the 3D labeling, is a direct function of the point cloud density;  
    \item The geometric quality point cloud by SfM/MVS technique depends on the camera parameter estimation, image spectral and spatial characteristics. Therefore, the targets geometry and texture are fundamental in the process of reconstruction and classification.      
    %\item The sequence of facade images, when properly acquired, allows 3D reconstruction and, consequently, the observation of their volumetry by the SfM/MVS technique, most of the time, with no cost;
    %\item Details of facades can be analyzed by (i) sensors on close-range platforms, such as mobile UAVs (fixed or dynamic wings) and terrestrial (fixed or mobile); and (ii) long/medium-range, as airplanes, since it can carry sensors with specific acquisition capability and geometry (e.g. oblique cameras);
    
\end{itemize}

\section{Objectives}
\subsection{Main}
From structural data campaigns (image-based point cloud, see Section \ref{aquisicao}), the main objective of this research is to explore the extraction of geometric information of buildings, simultaneously, to detect their facade features and, finally, relate both information in one single 3D labeled model.

\subsection{Specific}
\begin{enumerate}	
    \item Develop a routine to classify facade elements in 2-dimensional (2D) images using a CNN architecture;
    \item Using the same images, obtain the facade geometry using SfM/MVS;
    \item Evaluate the performance of the neural model for different urban scenarios and architectural styles;
    \item Evaluate a case study with an application in Brazil, whose architecture differs from the datasets used during the neural model training;
    \item Classify the 3D model of the extracted facade using the images previously segmented in the 2D domain by the technique of \emph{Ray-Tracing}.    
\end{enumerate}

\subsection{Thesis's structure}
Based on this workflow, in Section \ref{chapter2} it is highlighted the essential urban characteristics for the extraction of information through remote sensing, its challenges and the evolution of techniques. In Section \ref{chapter3}, the details of the methodology and data adopted for the study are presented. In Section \ref{chapter4}, it is analyzed the results in both categories: on 2-dimension (quality of detection) and 3-dimension (quality of 3D labeling), as well as the training effects between the architectural style and the inference quality under an unknown one. Finally, in Section \ref{chapter5}, our main conclusions and future prospects.

    \chapter{LITERATURE REVIEW}\label{chapter2}
Essentially, the results in the range of alternatives to reconstruct cities vary according to the definition of three main phases: (i) sensors and an appropriate measurement of the targets, (ii) processing and classification, according to a desired level of detail and (iii) standardization in well-established formats, such as CityGML \cite{kolbe2005}. The following sections introduce the main works and methodologies in 3D reconstruction of buildings and facades, as well as the spatial and spectral characteristics usually found in these environments, which represent the great challenges of the area.
%According to the guidelines in this study, we aim to semantic label facade features and assign to each of them its corresponding geometry. The facades images are optical, taken on terrestrial close-range platform (phase (i)), and the methodology to extract information (phase (ii)) is determined according to the resolution of the acquired data. %This review section is structured to highlight the challenges of acquiring and extracting information from dynamic scenarios, under an infinite varieties of objects, shapes, and compositions

%In�meras fontes de dados, m�todos e recursos computacionais s�o conhecidos na literatura para a reconstru��o 3D urbana. Essencialmente, os resultados dessa gama de alternativas variam de acordo com a defini��o de tr�s principais fases: (i) sensores e adequada medida dos alvos, (ii) processamento e classifica��o conforme o n�vel de detalhamento desej�vel e (iii) transi��o das caracter�sticas extra�das para formatos de representa��o digital padronizados. De acordo com as diretrizes deste trabalho, nosso objetivo � a extra��o de informa��es em alto n�vel de detalhamento de fachadas, na qual o grau de detalhes � adquirido por imagens �pticas tomadas em plataformas terrestre de curta-dist�ncia (fase (i)), de modo que a metodologia para extra��o de informa��es (fase (ii)) seja determinada de acordo com a resolu��o dos dados adquiridos. Esta se��o de revis�o � estruturada de forma a destacar os desafios em se adquirir e extrair informa��es de cen�rios din�micos, com infinitas variedades de objetos e composi��es e o que tem sido feito at� os dias atuais.

\section{Structural data}\label{aquisicao} 
Structural data is considered here as all data that carries with it geometric information of the scene, that after being assigned to a valid coordinate system, can represent geometric values very close to the reality. In the following section, it is presented a summary of current platforms and sensors for acquiring this information.

\subsection{Common alternatives for structural data acquirement}
Urban environments are easily recognized in images by its linear aspects, repetitive patterns, with structures frequently devoid of natural elements, with a wide variety of sizes, shapes, compositions and arrangements. Applications involving studies of these environments equally demands a high resolution acquisition, usually performed on a large or small scale. In Table \ref{tabela-dado-estrutural}, is shown a schema regarding conventional types of structural data acquisition and the maximum Level of Details (LoD) that each configuration could reach in terms of urban objects. The list makes a balance between their respective costs, knowledge required to manage the equipment, accuracy, spatial and spectral resolutions, among others. It is important to note the respective table is a non-exhaustive list, which highlights only common forms of acquirement. Other forms of acquisition, such as civil construction cranes, onboard balloons, helicopters, and any other resources are not listed. 
%Ambientes urbanos s�o caracterizados por padr�es lineares, repetitivos, com estruturas quase sempre destoantes de elementos naturais, com elevada variedade de tamanhos, formas, composi��es e arranjos. As aplica��es envolvendo estudos desses ambientes recorrem igualmente ao sensoriamento remoto de elevada resolu��o, realizados normalmente em larga ou pequena escala. Apesar deste trabalho abranger apenas a reconstru��o 3D pelo sensoriamento remoto de curto alcance, avan�os da tecnologia pelo uso de dados de longo alcance s�o tamb�m apresentados. Portanto, de forma a categorizar esta se��o, na Tabela \ref{tabela-dado-estrutural} � apresentado esquemas com os tipos convencionais de levantamento de dado estrutural\footnote{A lista destaca as formas comuns de levantamento de dados estruturais por sensoriamento remoto. Portanto, trata-se de uma listagem n�o exaustiva. Outras formas de aquisi��o, como a bordo de bal�es, helic�pteros, guindastes e demais recursos n�o s�o listados.}, de forma a permutar as diferentes configura��es de levantamento com objetivo de classificar seus respectivos custos, n�veis m�ximos de detalhamento dos modelos 3D urbanos (do ingl�s, \emph{Level of Detail}, \cite{kolbe2005}), conhecimento necess�rio para o levantamento e objetos urbanos observ�veis pela respectiva alternativa de aquisi��o. 

% Point cloud (PC) is the standard data format to 3D surface reconstruction methods, and can be generated by multiple devices, such as laser scanners (see Section \ref{laser}), Structured-Light cameras (Kinect) \cite{guhring2001, chen2008kinect}, and Photogrammetric Stereo-pair techniques (see Section \ref{Photogrammetry}) \cite{oesau2016}.

\begin{table}[!htp]
    \renewcommand{\arraystretch}{1.4}
    \caption{Maximum Level of Detail (LoD) and quality of 3D urban models according to the acquisition and sensor characteristics.}
    \tiny \centering
    \begin{tabular}{C{0.01cm}C{0.02cm}C{0.6cm}C{0.6cm}L{0.6cm}C{1.3cm}C{0.5cm}C{0.5cm}C{0.5cm}C{0.6cm}C{0.6cm}C{0.5cm}C{0.7cm}C{0.7cm}L{0.7cm}}
        \toprule      
        
        \multicolumn{3}{c}{\multirow{2}{*}{Plataform}}  & \multirow{2}{*}{\parbox{0.7cm}{\centering Spectral region}} & \multirow{2}{*}{\parbox{0.6cm}{\centering Spatial resol.}} & \multirow{2}{*}{\parbox{1.3cm}{\centering Sensor view}} & \multicolumn{3}{c}{Building parts} & \multirow{2}{*}{\parbox{0.6cm}{\centering Max. LoD}} & \multirow{2}{*}{\parbox{0.6cm}{\centering Costs}} & \multirow{2}{*}{Operated} & \multirow{2}{*}{PC} & \multirow{2}{*}{Software} & \multirow{2}{*}{\parbox{0.7cm}{\centering Accuracy}}  \\ \cmidrule{7-9}
            & & & & & & Roof & Facade & Indoor &&&&&& \\ 
            
            \toprule      
            
            \multicolumn{2}{c}{\multirow{3}{*}{\rotatebox[origin=c]{90}{\textbf{\tiny Orbital}}}} & Satellite & Optical &  \textbullet\textbullet\textbullet\color{lightgray}{\textbullet}  & Multi-view & \cellcolor{green!20}\checkmark & $\times$ & $\times$ & \cellcolor{blue!15}LoD2 & \faDollar\faDollar\faDollar & \faGraduationCap & $\star\star\star$ & - &  \textbullet\textbullet\textbullet\color{lightgray}{\textbullet} \\
            
            & & Satellite& Laser & \textbullet\textbullet\textbullet\color{lightgray}{\textbullet} & Multi-view & \cellcolor{green!20}\checkmark & $\times$ & $\times$ & \cellcolor{blue!15}LoD2 & \faDollar\faDollar\faDollar & \faGraduationCap & \checkmark & \faDownload & \textbullet\textbullet\textbullet\color{lightgray}{\textbullet}\\ 
            
            & & Satellite& Hybrid & \textbullet\textbullet\textbullet\color{lightgray}{\textbullet} & Multi-view  & \cellcolor{green!20}\checkmark & $\times$ & $\times$ & \cellcolor{blue!15}LoD2 & \faDollar\faDollar\faDollar & \faGraduationCap & \checkmark & \faDownload & \textbullet\textbullet\textbullet\color{lightgray}{\textbullet} \\ \hline
            
            \multirow{9}{*}{\rotatebox[origin=c]{90}{\textbf{\tiny Aerial (multiple ranges)}}} & \multirow{4}{*}{\rotatebox[origin=c]{90}{\tiny long}} & Airplane & Optical & \textbullet\textbullet\color{lightgray}{\textbullet}\color{lightgray}{\textbullet} & Nadir & \cellcolor{green!20}\checkmark & \cellcolor{yellow!20}\checkmark$~\star$ & $\times$ & \cellcolor{yellow!15}LoD2$\star$ & \faDollar\faDollar\faDollar & \faGraduationCap & $\star\star\star$  & - & \textbullet\textbullet\color{lightgray}{\textbullet}\color{lightgray}{\textbullet}\\
            
            & & Airplane & Laser & \textbullet\textbullet\color{lightgray}{\textbullet}\color{lightgray}{\textbullet} & Nadir & \cellcolor{green!20}\checkmark & \cellcolor{yellow!20}\checkmark$~\star$ & $\times$ & \cellcolor{yellow!15}LoD2$\star$ & \faDollar\faDollar\faDollar & \faGraduationCap & \checkmark & \faDownload & \textbullet\textbullet\color{lightgray}{\textbullet}\color{lightgray}{\textbullet} \\
            
            & & Airplane & Hybrid & \textbullet\textbullet\color{lightgray}{\textbullet}\color{lightgray}{\textbullet} & Multi-view & \cellcolor{green!20}\checkmark & \cellcolor{green!20}\checkmark & $\times$ & \cellcolor{blue!30}LoD3 & \faDollar\faDollar\faDollar & \faGraduationCap & \checkmark & - & \textbullet\textbullet\color{lightgray}{\textbullet}\color{lightgray}{\textbullet} \\
            
            & & Airplane & Optical & \textbullet\textbullet\color{lightgray}{\textbullet}\color{lightgray}{\textbullet} & Multi-view & \cellcolor{green!20}\checkmark & \cellcolor{green!20}\checkmark & $\times$ & \cellcolor{blue!30}LoD3 & \faDollar\faDollar & \faGraduationCap & $\star\star\star$  & - & \textbullet\textbullet\textbullet\color{lightgray}{\textbullet} \\ \hhline{~~-------------}
            
            & \multirow{2}{*}{\parbox{0.1cm}{\rotatebox[origin=c]{90}{\tiny medium}}} & UAV & Optical & \textbullet\textbullet\color{lightgray}{\textbullet}\color{lightgray}{\textbullet} & Multi-view & \cellcolor{green!20}\checkmark & \cellcolor{green!20}\checkmark & $\times$ & \cellcolor{blue!30}LoD3 & \faDollar & \faGraduationCap & $\star\star\star$  & - & \textbullet\textbullet\color{lightgray}{\textbullet}\color{lightgray}{\textbullet} \\
            
            & & UAV & Laser & \textbullet\textbullet\color{lightgray}{\textbullet}\color{lightgray}{\textbullet} & Nadir & \cellcolor{green!20}\checkmark & $\times$ & $\times$ & \cellcolor{blue!15}LoD2 & \faDollar & \faGraduationCap & \checkmark & \faDownload & \textbullet\textbullet\color{lightgray}{\textbullet}\color{lightgray}{\textbullet} \\ 
            
            & & UAV & Hybrid & \textbullet\textbullet\color{lightgray}{\textbullet}\color{lightgray}{\textbullet} & Multi-view & \cellcolor{green!20}\checkmark & \cellcolor{green!20}\checkmark & $\times$ & \cellcolor{blue!30}LoD3 & \faDollar & \faGraduationCap & \checkmark & \faDownload & \textbullet\textbullet\color{lightgray}{\textbullet}\color{lightgray}{\textbullet} \\ \hhline{~~-------------}
            
            & \multirow{2}{*}{\parbox{0.1cm}{\rotatebox[origin=c]{90}{\tiny close}}} & UAV & Optical & \textbullet\textbullet\textbullet\textbullet & Multi-view & \cellcolor{green!20}\checkmark & \cellcolor{green!20}\checkmark & $\times$ & \cellcolor{blue!30}LoD3 & - & \faUser & $\star\star\star$  & - & \textbullet\textbullet\textbullet\textbullet\\
            
            & & UAV & Laser & \textbullet\textbullet\textbullet\textbullet & Multi-view & \cellcolor{green!20}\checkmark & \cellcolor{green!20}\checkmark & $\times$ & \cellcolor{blue!30}LoD3 & \faDollar & \faGraduationCap & \checkmark & \faDownload & \textbullet\textbullet\textbullet\textbullet \\ \hhline{---------------}
            
            \multicolumn{2}{c}{\multirow{5}{*}{\rotatebox[origin=c]{90}{{\textbf{\tiny Terrestrial}}}}} & User & Optical & \textbullet\textbullet\textbullet\textbullet & Multi-view & $\times$ & \cellcolor{yellow!20}\checkmark$~\star\star$ & $\times$ & \cellcolor{blue!30}LoD3 & - & \faUser & $\star\star\star$  & - & \textbullet\textbullet\textbullet\textbullet \\
            
            & & User & Laser & \textbullet\textbullet\textbullet\textbullet  &  Multi-view & $\times$ & \cellcolor{yellow!20}\checkmark$~\star\star$ & $\times$ & \cellcolor{blue!30}LoD3 & \faDollar & \faGraduationCap & \checkmark & \faDownload & \textbullet\textbullet\textbullet\textbullet \\
            
            & & User & Optical & \textbullet\textbullet\textbullet\textbullet & Multi-view & $\times$ & $\times$ & \cellcolor{green!20}\checkmark & \cellcolor{blue!45}LoD4 & - & \faUser & $\star\star\star$  & - & \textbullet\textbullet\textbullet\textbullet \\
            
            & & User & Laser & \textbullet\textbullet\textbullet\textbullet & Multi-view & $\times$ & $\times$ & \cellcolor{green!20}\checkmark & \cellcolor{blue!45}LoD4 & \faDollar & \faGraduationCap & \checkmark & \faDownload & \textbullet\textbullet\textbullet\textbullet \\       
        
        \bottomrule     
    \end{tabular}
    \label{tabela-dado-estrutural} 
    \vspace{0.1cm}
    \begin{flushleft}
        \scriptsize{\textbf{Low}, \textbf{Medium}, \textbf{High}, and \textbf{Very high}, respectively \textbf{(\textbullet\textbullet\textbullet\textbullet)};}\\  \vspace{0.1cm}
        \scriptsize{\textbf{Estimated cost (\faDollar) - 0} to \textbf{3}; \textbf{Specialist (\faGraduationCap)}; \textbf{Common user (\faUser)}; \textbf{Embedded (\faDownload)}.}\\ \vspace{0.15cm}
        \scriptsize{$\star$ Parameter is a function of the sensor's Field of View (FOV). In cases of comprehensive FOVs, it is possible to observe not only the characteristics of roofs, but also of facades, allowing an acquisition of values above LoD2. In cases of FOVs with narrow angles, only the building footprint are observable, consequently, only LoD1 is reachable.}\\
        \scriptsize{$\star\star$ The quality of facades images by terrestrial platforms, depends directly on the height of the building. The imaging of buildings with a height greater than 50 meters, for instance, is affected by the acquisition geometry in its upper part, as well as inner structures.}\\
        \scriptsize{$\star\star\star$ Control points are required.}\\      
    \end{flushleft}
    \FONTE{Author's production.}
\end{table}

% The imaging of urban environments, especially the analysis of facades, are carried out conventionally by orbital, aerial (long, medium and close-range) and terrestrial platforms. Orbital platforms usually have a common goal, the imaging of areas whose spectral and spatial characteristics does not vary often, e.g. forests and oceans. Although in orbit, some sensors have centimeter scales, which places it as one of the alternatives in the urban monitoring. The data are often on private or military domain, when not, with high costs or requiring some kind of pre-processing due to atmospheric attenuation. Observing Table \ref{tabela-dado-estrutural}, it can be seen that the three different orbital configurations have characteristics whose facade observation is possible, but only fine resolutions would allow them to observe the details, which is not always possible by this type of alternative. %When available, both images and laser measurements are rare and inaccessible to the urban context.
%O imageamento do espa�o urbano, sobretudo � an�lise de fachadas, s�o realizadas convencionalmente por plataformas orbitais, a�reas (de longo, m�dio e curto alcance) e terrestres. Plataformas orbitais normalmente possuem um objetivo comum, o imageamento de �reas cuja varia��o espectral e espacial s�o baixas, p.e. florestas e oceanos. Contudo, ainda que em �rbita, alguns sensores possuem capacidade de resolu��o em escalas centim�tricas, o que o coloca como uma das alternativas no monitoramento de ambientes urbanos. Imagens orbitais possuem ainda custos elevados, muitas vezes, sobre dom�nio privado ou de restri��o militar, al�m de exigirem pr� tratamentos devido as in�meras atenua��es do meio atmosf�rico. Observando a tabela, nota-se que as tr�s diferentes configura��es orbitais apresentam caracter�sticas cuja observa��o de fachadas � poss�vel, por�m, somente com resolu��es finas permitiriam observar seus detalhes, o que nem sempre � poss�vel por este tipo de alternativa. Quando dispon�veis, tanto imagens como medidas laser, s�o raras e inacess�veis para o contexto urbano.

Aerial platforms, are categorized here into three segments: long, medium and close-range. Surveys whose sensor is onboard airplanes, flying over altitudes around 1,000 meters, are usually considered as long-range surveys. They are, in fact, the most used category over the years, allowing sophisticate imaging, sometimes, with hybrid sensors and configurations that adequately serve the innumerable urban activities. Although many of these products still present high costs, aerial images are still the most widely used in large-scale studies, since it does not require a detailed analysis.
%Plataformas a�reas, por sua vez, s�o categorizadas aqui em tr�s segmentos: de longo, m�dio e curto alcance. Levantamentos cujo sensor � acoplado em aeronaves, sobrevoando altitudes em torno de 6.000 metros, s�o tidos como levantamentos de longo-alcance. S�o, de fato, a categoria mais utilizada ao longo dos anos, permitindo levantamentos urbanos sofisticados, com sensores de alta capacidade, por vezes h�bridos, com configura��es que servem adequadamente �s in�meras atividades urban�sticas. Apesar de ainda possu�rem custos elevados, as imagens a�reas de alta resolu��o ainda s�o as mais adotadas em estudos em larga escala e que n�o exijam an�lise detalhada. 

Different imaging requires different costs, efforts and technical infrastructure. The estimated cost shown in Table \ref{tabela-dado-estrutural} is relative and may vary according to area coverage. For example, the cost of orbital laser imaging or high-resolution optical sensors requires a high cost, but may benefit from covering an area whose terrestrial imaging would cover only in parts. The estimated cost in table, therefore, is the absolute cost.

On medium-range, the surveys are normally carried out by fixed-wing Unmanned Aerial Vehicles (UAV) or even by airplanes, whose altitude does not exceed 1,000 meters. In this category, remote data naturally has a gain in resolution, in addition to enabling large-scale imaging, e.g. farms, forests, neighborhoods and others. In facade analysis, only sensors with wide FOV or multi-views gives guarantees of imaging these areas. In Brazil, however, this category still goes through standardization and other bureaucratic procedures for practice, although it is regularly offered by numerous companies (for details, see Section \ref{map-brazil} and \ref{3d-urban-brazil}). 
%De m�dio alcance, destaca-se os levantamentos realizados normalmente por VANTs de asa fixa ou at� mesmo por aeronaves, cuja altitude n�o ultrapassa os 1.000 metros. Nesta categoria, os dados remotos possuem naturalmente um ganho de resolu��o, al�m de possibilitarem o imageamento em larga escala, p.e. fazendas, florestas, bairros e outros. Na an�lise de fachadas, somente sensores com �ngulos de visada (em ingl�s, \emph{Field of View} - FOV) amplos ou m�ltiplos sensores (Multi-view) d�o garantias de imageamento dessas �reas. No Brasil, entretanto, essa categoria ainda passa por normatiza��o e demais processos burocr�ticos para a pr�tica, ainda que regularmente oferecida por in�meras empresas. Diferentes imageamento exigem diferentes custos, esfor�os e infraestrutura t�cnica. O custo estimado exibido na Tabela \ref{tabela-dado-estrutural} � relativo e pode variar de acordo abrang�ncia da �rea. Por exemplo, o custo do imageamento orbital a laser ou por sensores �pticos de alta resolu��o espacial demandam elevado custo, por�m, podem beneficiar abrangem uma �rea cujo imageamento terrestre n�o abrangeria. O custo estimado na tabela, portanto, trata-se do custo absoluto.

Finally, close-range surveys consists of small and medium-sized platforms, with target's distance around 200 meters. This specific configuration has gained interest in areas such as Agriculture and Cartography. Close-range devices, such as the UAV (commonly called drones), are usually used for its flexibility, low cost and stability in flying over narrow paths. These qualities, makes this a right alternative to image all the faces of a particular building (e.g. roof, facades, inner gardens). Contrary to the lasers sensors, the UAVs are not only accessible to experts, but also to common users. These small devices are easily purchased on the market for recreational use, but depending on their physical characteristics, they can also be adapted for scientific studies. The handling of laser scanners, however, requires a certain expertise, not just onboard UAVs, but on any other platform. 
%Por fim, os levantamentos de curto alcance, formados por plataformas de pequeno e m�dio porte, com dist�ncia do alvo em torno de 150 metros, t�m ganhado distaque nas �reas cartogr�ficas nos �ltimos anos \cite{REF}.  Plataformas a�reas de curto alcance, como VANTs na categoria Quadric�ptero (comumente chamados de drones), s�o normalmente utilizados justamente pela flexibilidade, baixo custo e estabilidade em voo sobre regi�es estreitas. Por conta dessas qualidades, � poss�vel imagear todas as faces de um determinado edif�cio (p. e. telhado, fachadas, jardins internos). Com excess�o dos sensores laser, esse tipo de categoria est� n�o s� acess�vel aos especialistas, como tamb�m a usu�rios comuns. Esses pequenos dispositivos s�o facilmente adquiridos no mercado para uso recreativo, mas que dependendo de suas caracter�sticas f�sicas, podem ser tamb�m adaptados para estudos cient�ficos. O manuseio de sensores lasers, por�m, requer certo \emph{expertise}, n�o s� a bordo de VANT, mas para qualquer outra plataforma. 

The terrestrial platforms complete the range of alternatives in close-range imaging, where three main platforms are adopted: total-station, vehicles or even carried by the own specialist. Although the ground survey does not require the use of an aerial platform, this type of imaging has almost the same properties as the close-range aerial survey. In this category, the images are taken with lateral geometry, at the lower sight level, interesting configuration for facade analysis, but not feasible when the purpose requires the complete coverage of the building. For example, indoor gardens, roof, backyards among other structures are hardly observed.
%As plataformas terrestres completam a gama de alternativas de imageamento de curto alcance, nela, pode-se citar tr�s principais plataformas: esta��o total, autom�veis ou o pr�prio especialista. Apesar do levantamento terrestre n�o demandar a utiliza��o de uma plataforma a�rea, esse tipo de imageamento possui quase as mesmas propriedades do levantamento a�reo de curto-alcance. Nesta categoria, as imagens s�o tomadas com geometria lateral, no n�vel de visada inferior, caracter�sticas interessantes para an�lise de fachadas, mas invi�veis quando o prop�sito requer a cobertura completa da edifica��o. Por exemplo, jardins internos, telhado, jardins de fundo de demais estruturas s�o dificilmente observadas.

The green color in table, shows the features benefited by the respective configuration, while the yellow determines the dependence of some technical factor. In blue, the maximum LoD reachable by the respective configuration, where the lighters denote non-detailed 3D models, and darkers, detailed ones. In indoor surveys, measures are taken as a complement to lower levels, for instance, LoD2 and LoD3. Of course, the LoD4 model shown would be obtained in addition to an existing model. Measures indoor by itself, do not support the generation of LoD1, LoD2 or LoD3.
%Na tabela, a cor verde real�a as fei��es beneficiadas pelo respectivo levantamento, enquanto que o amarelo determina a depend�ncia de algum fator t�cnico para que, de fato, a fei��o seja imageada. As tonalidades em azul, o grau de detalhamento em LoDs, em que as mais claras denotam n�veis de pouco detalhamento, e escuras, mais detalhemento. Nos levantamentos terrestres em interiores de edif�cios (\emph{indoor}), s�o adquiridos medidas para complementar n�veis inferiores, por exemplo, LoD2 e LoD3. Evidentemente, o modelo em LoD4 mostrado na Tabela \ref{tabela-dado-estrutural}, � obtido em complemento a um modelo j� existente. Medidas \emph{indoor} por si s�, n�o d�o suporte a gera��o de LoD1, LoD2 ou LoD3.

\subsection{Other alternatives for structural data}   
The generation of 3D urban models are mainly archived using remote systems, either by active or passive sensors, embedded on a long or close-range acquisition platforms, which is determined often by the application purposes, as presented in the previous section. In addition, complete cities may also be reconstructed from instruments other than those previously presented, for example, by radar (Radio Detection and Ranging), which by its imaging properties, normally allows studies such as terrain variations using SAR (Synthetic Aperture Radar) Interferometry, topography analysis under forest regions, and also cities reconstruction by the technique known as Persistent Scattered Interferometry (PSI) \cite{ferretti2001}, that despite presenting accurate results, it is still expensive technology for requiring systematic radar surveys \cite{schunert2016, schack2012}.

Beyond Photogrammetry, laser, and radar reconstruction techniques, cities can also be generated by procedural models or even manually\footnote{Reconstruction is the process of creating digital representations as close as possible to the measurement in terms of accuracy \cite{musialski2013}. While generation consists of artificially creating realistic representations given a set of rules or procedural mechanisms \cite{muller2006}.}. The concept of virtual and smart cities has become even more common over the years, and has stimulated the continuous development of standalone applications independent of the imaging devices or platform.

\section{Urban environments and their representations}\label{urban-environment}
The different categories of artificial coverage (man-made) constantly change in small fractions of space and time, often altered by humans as well. Understanding the aspects of texture, geometry, material, architectural styles, coverage, among other physical properties helps define the level of abstraction in the method to be developed \cite{vangool2013}. The following sections briefly discuss some of the geometric aspects of buildings and their features, in order to contextualize the main factors for 3D modeling and reconstruction of these environments. For a more comprehensive reading, it is recommended the work of \cite{musialski2013}.
%Em ambientes urbanos, as caracter�sticas tanto espectrais quanto espaciais possuem alto grau de heterogeneidade. As diferentes categorias de cobertura artificial (\emph{man-made}) se alteram constantemente em pequenas fra��es de espa�o, se misturando tamb�m � cobertura natural, por vezes, alterada pela a��o do homem. Compreender os aspectos de textura, geometria, material, estilos arquitet�nicos, cobertura, entre outras caracter�sticas f�sicas ajudam a definir o n�vel de abstra��o do m�todo a ser desenvolvido \cite{vangool2013}. De forma breve, nas se��es seguintes s�o discutidos alguns dos aspectos f�sicos de edifica��es e suas fei��es, de forma a contextualizar os principais fatores para modelagem e reconstru��o 3D desses ambientes. Para leitura mais abrangente, recomanda-se ao leitor a refer�ncia \cite{musialski2013}.

\subsection{Buildings} 
% Estas caracter�sticas geom�tricas e espectrais s�o importantes para a concep��o de modelos de extra��o simples, tal como em \cite{lotte2017a}, 
Regardless the type of imaging, artificial structures are easily distinguished from natural mostly by their linear characteristics. Areas of vegetation are generally identified by their homogeneous texture and typical spectral properties, for example, the numerous vegetation indices, which often allow to separate adequately between vegetation and artificial coverage \cite{gerke2014, demir2012}. However, in some cases, vegetation mixes with urban elements, such as terrace gardens or vertical gardens on balconies, aspects that make it difficult to classify them at the spectral level, but could be easily categorized as being part of a building with the volumetry complement.
%Independente do tipo de imageamento, estruturas artificiais s�o facilmente distinguidas de estruturas naturais por suas caracter�sticas lineares. �reas de vegeta��o s�o geralmente identificadas pela sua textura heterog�nea e de propriedades espectrais t�picas, por exemplo, os in�meros �ndices de vegeta��o, que permitem na maioria das vezes separar adequadamente entre cobertura vegetal e artificial \cite{gerke2014, demir2012}. Contudo, em alguns casos, a cobertura vegetal se mistura quase que integralmente � ambientes urbanizados, como � o caso de terra�o-jardins ou jardins verticais sobre sacadas, aspectos que dificultam a classifica��o a n�vel espectral, mas que poderiam ser facilmente categorizados como para de uma edifica��es pela complementa��o da informa��o volum�trica.

A 3D urban model is a representation of an urban environment with 3D geometries of common urban objects and structures, with buildings as the most prominent feature \cite{billen2014, lancelle2010}. The perception of a building through the human visual system provides innumerable premises on which characteristics should be considered at first. Although the methodology presented in this study does not use spectral or geometrical properties of buildings to extract their features, some of their characteristics are discussed in this section.

A building consists of a cubic element formed by a flat roof with one or more faces \cite{tutzauer2015, haala1999}. Some of them have dome or arched characteristics, and may overlap each other (American type) or present spectral variations due to their composition. In Figure \ref{building-roofs-topologies}, typical roofing examples are illustrated, where it is possible to find from the simplest form (Figure \ref{telhado-a}), to more complex ones (Figure \ref{telhado-e} and \ref{telhado-f}). These characteristics are essential when the extraction of information is carried out essentially by aerial images. 
%A percep��o de um edif�cio por meio do sistema visual humano fornece in�meras premissas sobre quais caracter�sticas devem ser consideradas em um primeiro momento, por exemplo, estruturas lineares s�o as que exp�em claramente uma estrutura artificial e possivelmente uma regi�o de interesse. Um edif�cio consiste de um elemento c�bico formado por um telhado de tipo plano ou de uma ou mais �guas \cite{tutzauer2015, haala1999}. Al�m disso, alguns possuem caracter�stica de c�pula ou em arcos, podendo ainda se sobreporem (tipo americano) ou apresentar varia��es espectrais devido a sua composi��o. Na Figura \ref{telhados}, s�o ilustrados exemplos t�picos de telhados, em que � poss�vel encontrar desde a forma mais simples (Figura \ref{telhado-a}), at� formas mais complexas, consistindo em telhados do tipo c�pula (Figura \ref{telhado-f}) ou de formas livres.
\begin{figure}[!htp]
    \centering  
    \caption{Typical representations of roofs and building topologies. (a) Plane roof. (b) One-face roof. (c) Two-faces roof. (d) Four-faces roof. (e) Arched and (f) domed. (g) and (h) Parametric shape. (i) Prismatic. (j) Polyhedral. (l) Curved and (k) free-form.}
    \vspace{6mm}
    \subfigure[]{\label{telhado-a}\includegraphics[width=0.15\textwidth]{\figspath/telhados/telhados-a.pdf}}    
    \subfigure[]{\label{telhado-b}\includegraphics[width=0.15\textwidth]{\figspath/telhados/telhados-b.pdf}} 
    \subfigure[]{\label{telhado-c}\includegraphics[width=0.15\textwidth]{\figspath/telhados/telhados-c.pdf}} 
    \subfigure[]{\label{telhado-d}\includegraphics[width=0.15\textwidth]{\figspath/telhados/telhados-d.pdf}}
    \subfigure[]{\label{telhado-e}\includegraphics[width=0.15\textwidth]{\figspath/telhados/telhados-e.pdf}}
    \subfigure[]{\label{telhado-f}\includegraphics[width=0.15\textwidth]{\figspath/telhados/telhados-f.pdf}} \\
    \subfigure[]{\label{classes-building-a}\includegraphics[width=0.15\textwidth]{\figspath/classes-building/classes-building-a.pdf}}\hspace{0.1cm}
    \subfigure[]{\label{classes-building-b}\includegraphics[width=0.15\textwidth]{\figspath/classes-building/classes-building-b.pdf}}\hspace{0.1cm}
    \subfigure[]{\label{classes-building-c}\includegraphics[width=0.15\textwidth]{\figspath/classes-building/classes-building-c.pdf}}\hspace{0.1cm}
    \subfigure[]{\label{classes-building-d}\includegraphics[width=0.15\textwidth]{\figspath/classes-building/classes-building-d.pdf}}\hspace{0.1cm}
    \subfigure[]{\label{classes-building-e}\includegraphics[width=0.15\textwidth]{\figspath/classes-building/classes-building-e.pdf}}\hspace{0.1cm}
    \subfigure[]{\label{classes-building-f}\includegraphics[width=0.15\textwidth]{\figspath/classes-building/classes-building-f.pdf}}
    \vspace{2mm}
    \legenda{}    
    \label{building-roofs-topologies}   
    \FONTE{Adapted from \citeonline{haala1999} and \citeonline{brenner2000}.}    
\end{figure}

\citeonline{brenner2000} considers six different models of building classes: parametric, combined, prismatic, polyhedral, curved and free-form (Figures \ref{classes-building-a} to \ref{classes-building-f}). Parametric buildings are the simplest forms, usually referring to houses or large sheds, sometimes combining one another (Figure \ref{classes-building-b}), forming large complexes, usually present in industrial regions, in which buildings tend to have a sparse disposition. The prismatic model, frequent in commercial areas, represents buildings whose roof is flat and not necessarily in rectangular form. Polyhedral models expose the geometric details, such as chimneys, eaves, different level of roofs, among others. The curved model, is commonly observed in churches or religious buildings. The other buildings, whose shape does not have a defined pattern, are characterized as free forms, such as the Gherkin building in London, England, or the famous Copan building, in S�o Paulo, Brazil.
%\citeonline{brenner2000} considera seis diferentes modelos de classes de edifica��es: param�tricas, combinadas, prism�ticas, poli�dricas, curvadas e de forma livre (Figura \ref{classes-building}). As edifica��es param�tricas constituem-se das formas mais simples de edif�cios, normalmente referentes a casas ou grande barrac�es, por vezes, combinando entre si (Figura \ref{classes-building-b}), formando grande complexos, presentes geralmente em regi�es industriais, em que os edif�cios tendem a apresentar uma disposi��o mais esparsa. O modelo prism�tico, frequentes em �reas comerciais, representam os edif�cios cujo telhado s�o planos e n�o necessariamente em formas retangulares. No modelo poli�drico s�o reunidas tamb�m as fei��es de chamin�s, beirais, poss�veis desn�veis das faces dos telhados, entre outros. O modelo curvo � comumente observado em igrejas ou edif�cios religiosos. As demais constru��es, cuja forma n�o apresenta um padr�o definido, caracterizam-se como formas livres, como � o caso do edif�cio Gherkin, em Londres (Inglaterra), ou o famoso edif�cio Copan, na cidade de S�o Paulo. 

In real world, buildings are complex structures with different orientations, slopes, shapes, textures and compositions. In addition, tasks involving the extraction of facades features can be impaired by the presence of other objects surrounding, such as cars, lighting poles, pedestrians, and trees \cite{verdie2015, cheng2011}. They are located too close to the walls, which could be treat either as an occluder object or could be simply modeled as being part of the wall.
%No mundo real, edifica��es, s�o em sua grande totalidade, estruturas complexas, com diferentes orienta��es, desn�veis, inclina��es ou telhados em diferentes texturas e composi��es. Em aplica��es envolvendo a extra��o de fachadas ou bordas de edif�cios, a an�lise dessas informa��es torna-se complexa conforme a presen�a de outros objetos em seu entorno. Por se localizarem pr�ximas �s paredes, tais objetos obstruem e dificultam a aplica��o dos procedimentos de identifica��o \cite{verdie2015, cheng2011}.

Besides to the physical properties of buildings, choosing a specific platform and sensor would define, for example, which areas of the building are better imaged than the others. The Figure \ref{building-field-of-view-1} illustrated a close-range acquisition through multiple views. The faces in gray represent the regions not observed by the respective type of acquisition, while red consists the ones that totally imaged. Then, the best configuration for buildings, according to the picture, is the multi-view and terrestrial close-range acquisition (Figure \ref{building-field-of-view-1d}).
\begin{figure}[ht]
    \centering     
    \caption{Building's faces benefited according to the platform. (a) Sensors with geometry to nadir, allow to observe only roofs (some sensors, however, have wide FOVs that allow the analysis of adjacent facades). (b) Sensors with multi-view. A trend in urban mapping by aerial platforms. Oblique images provide wide imaging coverage. (c) Ground sensors on board mobile platforms. They allow the observation of details in high resolution. (d) Urban hybrid imaging by terrestrial sensors and close-range UAVs. They allow the observation of all the faces of buildings.}	 
    \vspace{6mm}
    \subfigure[]{\includegraphics[width=0.24\textwidth]{\figspath/building-field-of-view/building-field-of-view-01.png}}
    \subfigure[]{\includegraphics[width=0.24\textwidth]{\figspath/building-field-of-view/building-field-of-view-02.png}}
    \subfigure[]{\includegraphics[width=0.24\textwidth]{\figspath/building-field-of-view/building-field-of-view-03.png}}
    \subfigure[]{\label{building-field-of-view-1d}\includegraphics[width=0.24\textwidth]{\figspath/building-field-of-view/building-field-of-view-04.png}}
    \includegraphics[width=1\textwidth]{\figspath/building-field-of-view/building-field-of-view.pdf}
    \vspace{2mm}
    \legenda{}    
    \label{building-field-of-view-1}    
    \FONTE{Author's production.}
\end{figure}

% Despite that, a good representation is not restricted only to geometric precision \citeonline{lafarge2015}, but also to:  
% \begin{itemize}
%   \item The complexity of the 3D model, which measures the degree of the output compression;
%   \item The structural guarantees impose global regularities on the geometry and semantics of the output, for example, the standardization of LoDs;
%   \item The visual aspect.
% \end{itemize}
% Segundo \citeonline{lafarge2015}, uma representa��o com boa qualidade n�o se restringe � precis�o geom�trica, mas tamb�m por:
%   \begin{itemize}
%    \item A complexidade do modelo, que mede o grau de compacta��o da sa�da;
%    \item As garantias estruturais imp�e regularidades globais na geometria e sem�ntica da sa�da, por exemplo, a padroniza��o quanto aos n�veis de detalhamento;
%    \item O aspecto visual da representa��o.
%   \end{itemize}

%Como destacado na Se��o \ref{aquisicao}, o m�todo de extra��o depende exclusivamente do tipo de dado utilizado. A reconstru��o de edifica��es poderia, portanto, ser realizada via imagens a�reas, devido a possibilidade de imageamento amplo e sistem�tico da cidade. Levantamentos terrestre demandaria uma grande quantidade de armazenamento, ainda que toda a informa��o em altas resolu��es n�o fossem de fato necess�rias. 

% A informa��o tridimensional, desde o processo de cria��o ao processo de an�lise � muito interdisciplinar. S�o essenciais conhecimentos das mais diversas �reas como desenho t�cnico, programa��o e SIG para conseguir criar um modelo urbano tridimensional com informa��o relevante e �til para an�lise \cite{stoter2011}.

\subsection{Facade features}\label{facade-features}
The reconstruction of facades fulfills an important segment in inspecting and enforcing urban planning laws. For instance, the mapping of facade features\footnote{Sometimes called openings and referenced here as facade features, such as doors, windows, balconies, gates, etc.} could assist whether a new building can be erected in front or at a given distance from a reference point. \citeonline{burochin2014} analyzed the facade characteristics in order to validate building constructions in accordance to French planning laws. Not only the geometry of buildings is important, but their semantics as well. For the validation of urban plans, it is essential that windows and doors are not only geometrically represented but also explicitly labeled as such \cite{tutzauer2015}.
%A reconstru��o de fachadas cumpre um importante segmento na verifica��o da observ�ncias �s leis de planejamento urbano. A exist�ncia de aberturas (janelas e portas) em fachadas determina, por exemplo, se um novo edif�cio pode ser erguido em frente ou a uma dada dist�ncia dessas fachadas \cite{ibam}. \cite{burochin2014} investigam as caracter�sticas de fachadas a fim de identificar e validar as permiss�es de edifica��es em rela��o �s leis de planejamento francesa. N�o s� a geometria das edifica��es s�o importantes, como tamb�m sua sem�ntica. Para a valida��o de planos �, por exemplo, essencial que janelas e portas n�o sejam somente geometricamente representadas, mas tamb�m explicitamente rotuladas como tal \cite{tutzauer2015}. 

In this study, we focus only on the details of the building facades and, so far, there is no better way than close-range imaging to observe those details (as exemplified in Figure \ref{building-field-of-view-1}). The campaigns of urban data via terrestrial platforms benefits from the rich information collection, but it is a disadvantage as it does not allow a wide observation of the structure, such as their internal architecture, roof and, depending to their height, the upper part. To illustrate this issue, the diagram in the Figure \ref{building-field-of-view-2} shows each of these non-imaged features. The points $A$ and $D$, the high structures that, due to the limited FOV, might only be partially observed. The point $B$, the roof characteristics, which are hardly observed by terrestrial configurations. Finally, the point $C$, another example of structures that could not be observed by such imaging.
\begin{figure}[!htp]
    \centering     
    \caption{Diagram representing a typical terrestrial campaign. In particular, points A and D denote areas that were negatively affected. High buildings are only partially observed in this type of acquisition. B and C, show details of characteristics that can not be observed either by the type of imaging (only street-side view), or because they are internal structures, such as winter gardens - highlighted in C.}
    \vspace{6mm}
    \includegraphics[width=1\textwidth]{\figspath/building-field-of-view/building-field-of-view-5.pdf}
    \includegraphics[width=1\textwidth]{\figspath/building-field-of-view/building-field-of-view.pdf}
    \vspace{2mm}
    \legenda{} 
    \label{building-field-of-view-2}   
    \FONTE{Author's production.}
\end{figure}

%But as soon as the these features are correctly mapped, other geospatial databases could equally be benefit with merges of information.  
%Por�m, estamos interessados em mais detalhes, detalhes presentes nas faces do edif�cio, denominados aberturas, tais como portas, janelas, sacadas, port�es, etc. O levantamento de dados urbanos via plataformas terrestres beneficia com a riqueza de detalhes de fachadas, ao passo que perde ao n�o permitir a observa��o \textquotedblleft ampla\textquotedblright das edifica��es, como sua topologia, arquitetura interna, cobertura e, conforme sua altura, somente parte da fachada (Figura \ref{building-field-of-view-2}). 

The characteristics of doors and windows are pretty much distinguishable, a rectangular geometric pattern, sometimes, occluded by cars, poles, or other objects. The structure's texture uniformity and repeatability of such openings could be verified by the use of radiometric statistics, Histogram of Oriented Gradient (HOG), and Gradient Accumulation Profile (GAP), as in \citeonline{burochin2014}. However, the symmetry between the openings may not favor the respective method, unless the imaging is done by two different platforms or the architectural style be a well defined facade layout. 

In terms of architectural style classification, recent works such as \cite{mathias2011}, \cite{henn2012}, and \cite{weissenberg2014}, has addressed the problem to the first stage in a 3D reconstruction methodology: first, to identify what to deal with - Is it a residential area? Is it industrial? The success of any method depends solely on the distribution, testing and execution of micro tasks, which means in dismembering the global task in smaller tasks, in the end, all micro tasks performed and properly tested, will lead to a more consistent method. In this case, the classification of the facade geometry is a micro task of 3D reconstruction, which even though is not approached in this study, it is important to emphasize the need of this stage as a pre-step in 3D reconstruction issue. 

%As caracter�sticas de aberturas em fachadas, tais como portas e janelas, apresentam padr�o geom�trico retangular, ora sobre efeito de oclus�o (dependendo da orienta��o e visada do sensor) e baixa resolu��o espacial, ora n�o. A uniformidade da textura, estrutura e repetividade que tais aberturas apresentam poderiam ser verificadas pelo uso de estat�stica radiom�trica, histograma de orienta��o, gradiente e perfil de ac�mulo de gradiente, como em \cite{burochin2014}. Entretando, dependendo do estilo arquitet�nico, a simetria entre as aberturas, podem n�o favorecer o respectivo m�todo.

The methodology presented here does not include facade layouts categorization. Instead, it is intended to segment facade features by using Machine Learning (ML) techniques, which has proven they are robust under complex scenarios, such as undefined architectural styles or areas occluded by obstacles. The method considers not only multi-scale analysis, but it is also sensitive towards context, when objects obstructing doors or walls, for example, are easily ignored by the neural model when the obstruction does not cover the facade entirely. %In the following section, it is listed and discussed some of the main works focused on the extraction and reconstruction of urban environments.

\subsection{Architectural styles}\label{architectural-style}
In fact, the number of possibilities to get rid from these scenarios is wide. The range of alternatives, however, can grow even more according to the regions in which it is mapped in the city. Industrial sectors have more differentiated structures than residential or commercial sectors. For example, bigger buildings, few windows, the space in between is also bigger, straight lines geometries for walls, windows, doors, paths, so on. These, however, are more critical because they are sectors of higher traffic, in general, with dynamic structures, greater density that, in the end, make difficult for both to access (at acquisition) and to apply any autonomously method to extract the features.
%De fato, � infinito o n�mero de possibilidades para se contornar nesses cen�rios. A dimens�o de alternativas, no entanto, pode crescer ainda mais conforme as regi�es em que se mapeia na cidade. � evidente, portanto, que setores industriais possuam estruturas mais diferenciais que setores residenciais ou comerciais. Estes, no entanto, s�o mais cr�ticos por se tratar de setores de maior tr�fego, estruturas din�micas em geral, maior condensamento dos edif�cios que, por fim, dificultam tanto o acesso (na aquisi��o) quanto a pr�pria extra��o das fei��es. 

Looking at the architectural styles shown in Figure \ref{facade-layouts}\footnote{Each of them can be access in: Figure \ref{facade-layoutsa}: RueMonge2014 dataset; Figure \ref{facade-layoutsb}: \href{https://imgc.allpostersimages.com/img/print/posters/michael-runkel-neoclassical-architecture-st-petersburg-russia-europe_a-G-10810723-4990827.jpg}{Link}; Figure \ref{facade-layoutsc}: \href{https://www.wondermondo.com/wp-content/uploads/2017/10/Orvieto_cathedral.jpg}{Link}; Figure \ref{facade-layoutsd}: \href{https://www.impressiveinteriordesign.com/wp-content/uploads/2017/12/Warden-St-Residence-by-Mata-Design-Studio-1.jpg}{Link}; Figure \ref{facade-layoutse}: \href{https://static.dezeen.com/uploads/2016/07/mac-niteroi-oscar-niemeyer-rio-de-janeiro_dezeen.jpg}{Link}; Figure \ref{facade-layoutsf}: \href{https://www.verticalgardenpatrickblanc.com/sites/default/files/styles/slideshow/public/medias/image/4-9-20108-42-56500.jpg}{Link}; Figure \ref{facade-layoutsg}: \href{https://media.guiame.com.br/archives/2017/04/11/3673916577-favela-do-cantagalo-no-rio-de-janeiro.jpeg.}{Link}.}, the layout of facade features for these styles differ drastically and, in some cases, would be hardly detected by an automatic method due to the variety of shapes, textures, non-symmetry, and other.
%Observando os estilos arquitet�nicos mostrados na Figura 3, pode-se perceber que a disposi��o das fei��es de fachadas para estes estilos, diferem drasticamente e, em alguns casos, � dificilmente detect�vel por um m�todo autom�tico. 
\begin{figure}[!htp]
    \centering     
    \caption{Examples of architectural styles according to different countries and regions in the city. (a) Hausmaniann. (b) Neoclassic. (c) Religious. (d) Modern. (e) Free-form. (f) Vertical-gardens. (g) \emph{Aglomerados subnormais - favelas}.}    
    \subfigure[]{\label{facade-layoutsa}\includegraphics[width=0.169\textwidth]{\dropbox/phd/pics/archs/hausmaniann.jpg}}
    \subfigure[]{\label{facade-layoutsb}\includegraphics[width=0.262\textwidth]{\dropbox/phd/pics/archs/neoclassic.jpg}}    
    \subfigure[]{\label{facade-layoutsc}\includegraphics[width=0.175\textwidth]{\dropbox/phd/pics/archs/religious.png}}
    \subfigure[]{\label{facade-layoutsd}\includegraphics[width=0.3\textwidth]{\dropbox/phd/pics/archs/modern.jpg}}
    \subfigure[]{\label{facade-layoutse}\includegraphics[width=0.35\textwidth]{\dropbox/phd/pics/archs/free-form.png}}
    \subfigure[]{\label{facade-layoutsf}\includegraphics[width=0.33\textwidth]{\dropbox/phd/pics/archs/vertical-garden.png}}
    \subfigure[]{\label{facade-layoutsg}\includegraphics[width=0.3\textwidth]{\dropbox/phd/pics/archs/favela.png}}
    \legenda{} 
    \label{facade-layouts}  
    \FONTE{Author's production. The images was taken from multiple sources in internet, which authors were unknown.}
\end{figure}
 
The Haussmanian style, in Figure \ref{facade-layoutsa}, was the result of a French renovation during the 1870s and represents the predominant style in France, with straight facades, with around 4 to 5 floors, windows often together with balconies, ground floor with the presence of shops, with clear walls and well defined textures. This style, as well as Classicism and Historicism (Figure \ref{facade-layoutsb} and \ref{facade-layoutsc}) also present in many European countries (e.g. Austria, Germany and Spain), is also characterized by the symmetry among the facade features, such that by the layout of the windows and roof, would be possible to estimate the number of floors or amount of internal luminosity of each sector. These styles make up part of the datasets used in this study and in the literature regarding facade feature detection and 3D reconstruction.
%O estilo Haussmaniann, na Figura 3a, foi decorr�ncia de uma renova��o fran�esa durante a d�cada de 1870 e representa o estilo predominante na Fran�a, com fachadas retas, de 4 a 5 andares, janelas acompanhadas quase sempre por sacadas, t�rreo com a presen�a de lojas e texturas claras e bem definidas. Esse estilo, assim como o Classicismo, Historicismo (Figura3b e Figura3c) presentes tamb�m em grande parte de pa�ses europeus (Austria, Alemanha e Espanha), � caracterizado tamb�m pela simetria entre as fei��es, tal que pelo identifica��o das janelas e disposi��o do telhado, poderia-se estimar o n�mero de andares ou quantidade de luminosidade interna de cada setor. Esses estilos, marcam grande parte dos conjunto de dados utilizados neste trabalho e na literatura quanto a detec��o de fachadas e reconstru��o 3D.

Figures \ref{facade-layoutsd} to \ref{facade-layoutsg}, on the other hand, present examples of complex architectural geometries that, in terms of autonomous methods of detection and reconstruction, require approaches different from those treated in this study. That is, it is evident that methodologies dealing with the extraction and reconstruction of facade features in Hausmaniann architectures would certainly fail on modern architectures. However, these discussions are difficult to find in the literature that, in certain moments discard this question. Modern architectures have glass in abundance, not only in windows, but also on every surface of their walls. Reflexive surfaces, for example, have features that render the SfM/MVS method useless.
%As Figura3d a 3g, por outro lado, s�o apresentados exemplos de geometrias arquiteturais complexas que, em termos de m�todos aut�nomos de detec��o e reconstru��o, requerem abordagens diferentes daquelas tratadas neste estudo. Ou seja, � evidente que metodologias que tratam a extra��o e reconstru��o de fei��es de fachadas em arquiteturas hausmaniann certamente falhariam sobre arquiteturas modernas. Por�m, esses di�logos s�o dificilmentes encontrados na literatura que, em certos momentos descartam essa quest�o. Arquiteturas modernas possuem vidros em abund�ncia, n�o s� em janelas, como tamb�m em toda superf�cie de suas paredes. Superf�cies reflexivas, por exemplo, possuem caracter�sticas que tornam o m�todo SfM/MVS in�til.

Modern architectures, as in Figure \ref{facade-layoutse}, demarcate another stage of reconstruction. These structures, because they are atypical, do not require automatic methods of reconstruction. They are usually mapped or generated artistically. 

Not only vertical gardens, in Figure \ref{facade-layoutsf}, but also terraced gardens, are becoming common throughout the years. A sustainable practice that tries to recover the space occupied by constructions, but which represent a difficulty in 3D reconstruction. Its texture is confused with the undergrowth or arboreal vegetation which, for methods that use this metric, can be easily confused. Separate one facade from another, only using spectral information, is still a remaining problem. Although works such as the proposed by \citeonline{martinovic2015} still get good separability, the case does not look on real situations, with the presence of any kind of object in front of it, or even under the facade surface. 
%Arquiteturas modernas, como na Figura 3e, demarcam um outro est�gio da reconstru��o. Essas estruturas, por serem at�picas, n�o demandam m�todos autom�ticos de reconstru��o. Normalmente, s�o mapeadas ou geradas artisticamente. N�o s� jardins verticais, na Figura3f, como tamb�m jardins sobre terra�os, est�o tornando-se comuns ao longo dos anos. Uma pr�tica sustent�vel que tenta recuperar o espa�o ocupado por constru��es, mas que representam uma dificuldade na reconstru��o 3D. Sua textura se confunde a vegeta��o rasteira ou arb�rea que, para m�todos que utilizam dessa m�trica, podem se confundir facilmente. 

Finally, although in Figure \ref{facade-layoutsg} the building does not have a defined architecture, it represents one of the most important structures that, once reconstructed, would greatly benefit 3D mapping in Brazil: the subnormal clusters (in Portuguese, \emph{aglomerados subnormais}), popularly called \textquotedblleft favelas\textquotedblright. These buildings are difficult to access, usually accessible only by walking. By aerial campaigns, they are difficult to map by their spectral characteristics, they do not present any symmetry or pattern. However, they have cultural centers, a large part of the population lives in these areas, it is impossible to measure issues such as sanitation or disasters, attacks on public safety, among others.
%Por fim, arquiteturas indefinidas mas que representam uma das estruturas de maior import�ncia para o mapeamento 3D no Brasil: os aglomerados subnormais, popularmente chamados de favelas. Esses edif�cios s�o de dif�cil acesso, normalmente, acess�veis apenas caminhando. Por levantamentos a�reos, s�o dif�ceis de se mapear por suas caracter�sticas espectrais, n�o apresentam qualquer simetria ou padr�o. No entanto, possuem centros culturais, grande parte da popula��o vive nessas �reas, � imposs�vel mensurar quest�es como saneamento ou de desastres, os embates quanto a seguran�a p�blica, entre outros. 
%Buildings erected out of any civil constructing standards, sometimes in areas of risk, predominant in countries such as Brazil, the \emph{favelas} are consequence of bad income distribution and the housing deficit in the country.

It is necessary to attack smaller causes, and science, as in any area, will merge each of these branches according to their context. Although this study has approached the analysis of different scenarios, including complex ones, it is stated that the 3D reconstruction issue must be treated according to the city or region in which intends to study. Urban environments are extremely dense and complex environments. The conception of a generic model that draws attention to all these disparities is impractical. However, constant observation of these structures narrows down the range of options and encourages future studies. Bringing a natural evolution of such routines.
%� necess�rio que se ataque causas menores e a ci�ncia, como em qualquer �rea, unir� cada uma dessas ramifica��es conforme seu contexto. Embora este estudo ter compreendido a an�lise de diferentes cen�rios, incluindo os complexos, fixa-se que o problema da reconstru��o 3D deve ser tratado segundo a cidade ou regi�o na qual se deseja estudar. Ambientes urbanos s�o ambientes extremamente densos e complexos. A concep��o de um modelo gen�rico que extraia e que se atente a todas essas disparidades � impratic�vel. Por�m, a observa��o constante dessas estruturas estreita a gama de op��es e fomenta estudos futuros.

%Although the standardization of the 3D models have not been adopted in this study, the next section brings it up to clarify and to mention the importance of this practice. The format briefly presented, is already adopted by many institutions and companies \cite{biljecki2015a}, and constitutes one of the most important elements in the construction of semantic 3D maps.

\subsection{Standardization - Level of Details (LoDs)}\label{padroniza��o}  
The Open Geospatial Consortium (OGC)\footnote{Association of companies, universities and government agencies for the development of publicly available geospatial interface standards to support interoperable solutions, making geospatial services accessible and useful \cite{ogcstandard}.} with the purpose of standardizing the file formats and quantifying different levels of details of 3D models, established the CityGML language, which provides support for the declaration of physical characteristics and relations between urban elements. 

The language consists of a common semantic model for 3D representations of urban environments. Developed as open-source in XML (eXtensible Markup Language) format, the standard adopts the Geography Markup Language 3 (GML), which represents the international standard for spatial data exchange and coding, idealized by OGC and ISO TC211 \cite{kresse2004}. The standard establishes LoDs consisting of degrees that each 3D object is represented. Objects in the same scene may have higher or lower degrees than others.

LoDs are classified into five levels: first, LoD0, corresponds to the basic level of detail, comprising the planimetric information. The second level, LoD1, corresponds to the 3D representation, simple extrusions, where the buildings are represented not only by its spatial location, but also by its height. At the LoD2 level, simple structural features such as external columns or garages are aggregated. The LoD3, presents more refined external structural details. At this level, texture components and openings are added to their facades, such as doors, windows and balconies. Finally, the LoD4 represents the highest level, with information on the internal structure, usually, acquired by indoor instruments \cite{kolbe2005} (Figure \ref{lods}).
\begin{figure}[!htp]
    \centering     
    \caption{Levels of detail (LoDs) of the standard \emph{CityGML} for 3D buildings models.}
    \vspace{6mm}
    \includegraphics[width=1\textwidth]{\figspath/lods/lods.pdf}    
    \legenda{}
    \label{lods}   
    \FONTE{Adapted from \citeonline{kolbe2005}.}
\end{figure}

Even though there are other standards such as COLLADA (COLLAborative Design Activity) and KML (Keyhole Markup Language), the OCG CityGML standard is particularly the most adopted. CityGML mainly describes the geometry, attributes and semantics of different kinds of 3D urban elements. These can be supplemented with textures or colours in order to give a better impression of their appearance. In addition, semantic information can also be provided, where a description of the geometry, attributes and relationship between different objects can be specified, for example, a building can be decomposed into three parts, or has a safe or a balcony. 

\subsection{Advances in 3D urban reconstruction}\label{monitoring-cities}
The use of high resolution images has been the most effective method for systematic monitoring in all contexts, be it forest, ocean or city. Understanding patterns of changes over time is, however, a task that requires enormous effort when executed manually. In addition, a human interpreter is susceptible to failure and could require prior experience in target perception. As matter of fact, so far few works have been carried out in the field of architectural style identification or facade feature extraction and reconstruction \cite{martinovic2015}. Research has reached a certain level of maturity today, with a variety of technologies for acquisition, high graphics processing, storage and dissemination of information that is available for the automatic operators to make use of. 

The development of intelligent operators for image labeling can be categorized into model-free, model-based, and procedural-models. The first, classical segmentation methods such as Normalized Cuts (NC) \cite{shi2000}, Markov Random Fields (MRF) \cite{kolmogorov2004}, Mean Shift (MS) \cite{comaniciu2002}, Superpixel \cite{achanta2012}, and Active Contours (ACon) \cite{kass}, do not consider the shape of objects or their spectral characteristics, which, in practice, means they always fail in regions where the elements of a same facade do not share the same spectral attributes. Model-based or parametric-model operators, use \emph{a priori} knowledge base, which together with segmentation procedures, provide more consistent results on a given region. However, this knowledge is finite and opens up new possibilities for failures when applied in regions with different characteristics. The third and last, procedural-models like Grammar Shape \cite{stiny1971}, comprise the group of rule-based methods, in which algorithms are applied to the production of geometric forms \cite{teboul2011}.
%Model-free: no assumption on the geometry and the appearance of facade components -> normalized cuts [18], MRFs [10], mean shift [3] and snakes/active contours/level sets [9, 16]
%      Unfortunately, these methods are prone to failure simply because pixels belonging to the same facade element do not necessarily share common visual characteristics
%Model-based: provide a segmentation which is a compromise between the available image support and a prior knowledge 
%      The main assumption of these methods is that the space of solutions can be parameterized in a convenient way. This is far from being sufficient when dealing with building facades, simply because no static parameterization is generic enough to encompass different facade layouts.
%Procedural-models: offer a flexible and powerful tool to account for such variability while being compact and providing a semantic representation of the obtained results

Image segmenters that carry some intelligence usually carry issues with them as well. First, it is necessary to configure and train the model, then, make inferences in what each pixel or region corresponds to. \citeonline{teboul2011} proposed a Grammar-based procedure to segment building facades, where a finite number of architectural styles are considered. The proposed method is able to classify a wide variety of facade layouts and its features using a Tree-based classifier, which improves the detection with only a small percentage of false negatives. 

The Grammar-based approaches, however, are normally formulated by rules which follow common characteristics, such as the sequentiality, which is normally present on \textquotedblleft Manhattan-world\textquotedblright$~$or European styles \cite{wenzel2008}, where the shapes found seem to have lower geometric accuracy since the 3D model is generated, not reconstructed. Still, the outcomes of Grammar-based approaches provide simplicity and perfectly resume the real scene. Other similar works, such as \cite{becker2009}, \cite{nan2010}, \cite{wan2012}, and \cite{boulch2013}, have proposed equivalents solutions to the problem, but still lies on the same deficiencies mentioned above.

In addition to procedural modeling, urban 3D reconstruction incorporates a new class of research, one based on physical (structural) measurements, which comprise measurements by laser scanners and MVS workflows (mainly). As far as it is known, in this category lie the works that present the most consistent methodologies and results, which take into account the geometric accuracy, where the classification of objects can also be explored by their shapes or volume, in addition to their spectral information. 

\citeonline{jampani2015} and \citeonline{gadde2017}, respectively proposed a 2D and 3D segmentation based on Auto-Context (AC) classifier \cite{tu2010}. The facade features are explored by their spectral attributes and then, iteratively refined until results are acceptable. The AC classifier applied in a urban environment is a good choice since it considers the vicinity contribution, which is essential in this particular scenario. The downside, however, is that in this case the AC only succeeds when the feature detection (based on spectral attributes) is good enough, otherwise, it could demand many AC stages to get an acceptable output.

Unlike the approaches mentioned above, there are also lines of research that address the problem of 3D reconstruction over the mesh itself, for this reason, other structural data can be explored, such as LiDAR. \citeonline{lafarge2011, verdie2015, oesau2016} present different contributions, however, with strictly related focuses. It should be noted, therefore, that the approaches in this line of research demand complex geometric operations, for instance, regularities using parallelism, coplanarity or orthogonality. These operations usually have refinement purposes, and also give the 3D modeling an alternative to acquire more consistent and simplified models.
%Diferentemente de abordagens mencionadas acima, h� tamb�m linhas de pesquisa que tratam o problema da reconstru��o sobre o pr�prio mesh, por essa raz�o, outros dados estruturais podem ser explorados, como LiDAR. \cite{lafarge2011, verdie2015, oesau2016} apresentam diferentes contribui��es, por�m, com focos estritamente relacionados. Nota-se, portanto, que as abordagens nesta linha de pesquisa demanda complexas opera��es de ajustes geom�tricos, como regularidades de paralelismo, coplanaridade e paralelismo entre faces do mesh, s�o voltados especificamente para a reconstru��o e refinamento da superf�cie do objeto em quest�o, dando � modelagem uma alternativa para a aquisi��o de modelos 3D mais consistentes e simplificados.

Automatic 3D reconstruction from images using SfM/MVS workflows is challenging due to the non-uniformity of the point cloud, and it might contains higher levels of noise when compared to laser scanners. In addition to that, missing data is an unavoidable problem during data acquisition due to occlusions, lighting conditions, and the trajectory planning \cite{li2016}. The papers mentioned in the last paragraph, explore what it is understood as the best methodologies in 3D reconstruction, according to our established goals: exploring the texture first, and then acquiring the semantic 3D model \cite{brostow2008}. 

\citeonline{martinovic2015} proposed an end-to-end facade modeling technique by combining image classification and semi-dense point clouds. As seen in \citeonline{hayko2014} and \citeonline{bodis2017}, the facade features are extracted and semantized by the analysis of its texture, not only, the informations extracted are also associated with the geometry, extracted by the use of SfM. Both approached have motivated this thesis in the sense that facade 2D information can be explored more thoroughly in order to improve its volumetry reconstruction. \citeonline{martinovic2015}, for instance, uses the extracted facade features to analyze the alignment among them, where a simple discontinuity shows the boundaries between different facades. Thereby, it could be used to pre-classify subareas, such as residential, commercial, industrial, and others. \citeonline{sengupta2013}, \citeonline{hayko2014} and \citeonline{bodis2017}, similarly, explored different spectral attributes in order to discriminate the facade features as well as possible. Moreover, the 3D modeling is later supported by these outcomes by performing a complex regularization and refinements over the mesh faces.


% No dom�nio 3D, tanto Martinovic quanto Gadde, aplicam seus m�todos sobre edifica��es baixas (m�ximo, 5 andares). A nuvem de pontos baseadas em imagens s�o satisfat�rias para edifica��es mais elevadas quando adquiridas por aerolevantamentos de curto alcance, tarefa ainda invi�vel, devido as in�meras leis para a aquisi��o de dados nessas �reas e tamb�m pela dificuldade operacional

%Em \cite{verdie2015}, a classifica��o � feita no mesh e n�o na nuvem de pontos, via MRF com atributos geom�tricos de planaridade, horizontalidade e eleva��o. Depois da classifica��o, s�o aplicadas regras para o refinamento da classifica��o. Duas regras s�o consideradas. 1o: Superfacet rotulados como vegeta��o adjacente a rotulos de somente telhados � remarcada como telhado. 2o: Superfacet rotuladas como fachada e adjacente a superfacets de vegeta��o e solo, � remarcada como vegeta��o, entre outros


\section{3D mapping around the World}\label{map-mundo}
%It is not purpose of this section to present an exhaustive survey of countries and cities that already adopt 3D mapping. Instead, we just give an insight in how the research is heading on. 
Investigating the exact number of cities that actually use 3D urban models as strategic tool in their daily lives can be a difficult task. However, \citeonline{biljecki2015a} presented a consistent review of entities (industry, government agencies, schools and others) that make or made use of 3D maps beyond the visual purpose. Hence, only applications supported by the respective technology are, in fact, listed. Examples of such applications are the visibility analysis for security cameras installation \cite{ming2002, yaagoubi2015}, urban planning \cite{sabri2015, leszek2015}, air quality analysis \cite{amorim2012, janssen2013}, evacuation plans in emergency situations \cite{kwan2005}, urban inventories with cadastral updating \cite{qin2014}, among others.
%N�o � objetivo apresentar nesta sess�o um levantamento exaustivo de pa�ses e cidades que j� adotam o mapeamento 3D. Investigar o n�mero exato de cidades que utilizam de fato modelos 3D urbanos como ferramenta estrat�gica em seu cotidiano pode ser uma tarefa ma�ante. Para levantamento mais detalhado, recomenda-se a leitura de \cite{biljecki2015}, em que � apresentado uma revis�o detalhada de entidades (ind�stria, �rg�os governamentais, escolas e outros) que fazem ou fizeram uso de modelos 3D urbanos al�m do prop�sito visual, de modo que s�o listadas apenas aplica��es amparadas, de fato, pelo uso de modelos 3D. Exemplos dessas aplica��es s�o a an�lise de visibilidade para instala��o de c�meras de seguran�a \cite{ming2002, yaagoubi2015}, planejamento urbano \cite{sabri2015, leszek2015}, an�lise da qualidade do ar \cite{amorim2012, janssen2013}, planos de evacua��o em situa��es de emerg�ncia \cite{kwan2005}, invent�rios urbanos com pontos de atualiza��o cadastral \cite{qin2014}, entre outros. 

Although the number of cities adopting this tool is uncertain, some of these are known for their technological advances and social development, an important indicator in the implementation of innovative projects. Countries such as the United States, Canada, France, Germany, Switzerland, England, China and Japan are among the leading suppliers of Earth Observation equipment, for example, Leica Geosystems\texttrademark~(Switzerland) laser systems, FARO\texttrademark~(USA), Zoller-Fr�hlich\texttrademark~(Germany), RIEGL Laser Measurement Systems\texttrademark~(Austria), Trimble Inc.\texttrademark~(USA), TOPCON\texttrademark~(Japan), and countless optical sensors used in ground and airborne surveys. It is natural, therefore, that these great providers also become reference in conducting research in the sector.
%Apesar do n�mero de cidades que adotam o recurso seja incerto, algumas destas s�o conhecidas pelos seus avan�os tecnol�gicos e desenvolvimento social, indicativos importantes na implanta��o de projetos inovativos. Pa�ses como Estados Unidos (EUA), Canad�, Fran�a, Alemanha, Su��a, Inglaterra, China e Jap�o est�o entre os principais fornecedores de equipamentos de Observa��o da Terra, por exemplo, sistemas laser Leica Geosystems\texttrademark (Su��a), FARO\texttrademark (EUA), Zoller-Fr�hlich\texttrademark (Alemanha), REIGL Laser Measurement Systems\texttrademark (�ustria), Trimble Inc.\texttrademark (EUA), TOPCON\texttrademark (Jap�o), e incont�veis sensores �pticos utilizados nos levantamentos terrestres e a�reos. � natural, portanto, que estes grandes polos tornam-se tamb�m refer�ncia na realiza��o de pesquisa no setor.

In Germany, the so-called \textquotedblleft city-models\textquotedblright~were built with the basic purpose of assisting and visualizing simple scenarios or critical situations. At that time, these models did not have sufficient quality for certain analyzes or permanent updating, making use of the old 2D registers for queries. In the end, the 3D models never became part of the register. The concept of urban 3D reconstruction has become, due to demand, a scientific trend in Cartographic, Photogrammetry and Remote Sensing (RS) almost everywhere in the world, especially in the aforementioned countries. 

Naturally, new questions arose. How to merge information already available in 2D databases with the ones in 3D? In certain circumstances, what is the limit on the use of 3D information? When the 2D is already enough and when not? \citeonline{biljecki2015a} argue that all applications that require 2D information can be solved with 3D, but that does not make it a unique feature, but an optional one. For example, \citeonline{kluijver2003} carried out a study of the propagation of noise in urban environments from 2D data. Years later, in \citeonline{stoter2008}, the study was complemented with 3D information, showing considerable improvement in the estimation.
%Na Alemanha, os ent�o chamados \textquotedblleft modelos de cidades\textquotedblright, foram constru�dos com o objetivo b�sico de auxiliar e visualizar cen�rios como este ou, at� mesmo, de cen�rios alternativos em situa��es cr�ticas. Naquele per�odo, tais modelos n�o possu�am qualidade suficientes para determinadas an�lises ou de atualiza��o permanente, fazendo com que se recorresse � antigos cadastros 2D para consultas como localiza��o exata e dimens�es das edifica��es. Por fim, esses modelos 3D nunca se tornavam parte do cadastro. Em meio a necessidade, o conceito de reconstru��o 3D urbana tornou-se ent�o uma tend�ncia cientifica nas �reas cartogr�ficas, fotogram�tricas e sensoriamento remoto em quase todo o mundo, sobretudo, nos respectivos pa�ses. Naturalmente, novas d�vidas surgiram. Como integrar informa��es j� dispon�veis em cadastros 2D com informa��es 3D? Em certas circunst�ncias, questiona-se o limite do uso da informa��o 3D em complemento � 2D \cite{herbert2015}. Em que situa��es o uso da informa��o 2D � suficiente e quais somente a 3D � requerida? \cite{biljecki2015} argumentam que todas aplica��es que exigem informa��es 2D podem ser solucionadas com 3D, mas que n�o a torna um recurso exclusivo, e sim, opcional. Por exemplo, \cite{kluijver2003} realizaram um estudo da propaga��o de ru�dos sonoros em ambientes urbanos a partir de dados 2D. Anos depois, em \cite{stoter2008}, o estudo fora complementado com informa��es 3D, mostrando consider�vel melhoria na estimativa. 

% ADICIONAR TRABALHOS - ABORDANDO OUTROS M�TODOS 
% Also promising and good results by its simplicity, the Shape Grammar (SG) technique has also been used in the modeling of complex structures when joining geometric primitives to the arithmetic operations, that from premises of symmetry, can infer as to the other elements not imaged \cite{becker2009, teboul2011, vangool2013}. In geometry analysis, such as those discussed in \cite{lafarge2015, verdie2015}, have been showing evolution as to the methods of classification in meshes, as well as procedures of refinement and better representability of the 3D model.

\section{3D mapping in Brazil}\label{map-brazil}
In Brazil, the Geographic Service Directorate (in Portuguese, \emph{Diretoria de Servi�o Geogr�fico} (DSG))\footnote{Available at http://www.dsg.eb.mil.br. Accessed \today.} is the unit of the Brazilian Army responsible for establishing Brazilian cartographic standards for the 1: 250,000 and larger scales, which implies standardizing the representation of urban space for basic reference mapping. Recently, the National Commission of Cartography (CONCAR) has put forward the new version of the Technical Specification for Structuring of Vector Geospatial Data (ET-EDGV) \cite{concar2016}, which standardizes reference geoinformation structures from the 1:1000 scale. The data on this scale serve as a basis for the planning and management of the urban geographic Brazilian space. In Brazil, the demand for 3D urban mapping is still low and faces challenges that go beyond its standardization.
%A Diretoria de Servi�o Geogr�fico (DSG)\footnote{Portal da Diretoria de Servi�o Geogr�fico dispon�vel em: http://www.dsg.eb.mil.br. Acesso em: \today.} � a unidade do Ex�rcito Brasileiro respons�vel por estabelecer as normas cartogr�ficas brasileiras para as escalas 1:250.000 e maiores \cite{brasil1967}, o que implica em normatizar a representa��o do espa�o urbano para o mapeamento b�sico de refer�ncia. No contexto do arcabou�o normativo da representa��o do espa�o urbano, � poss�vel identificar o Comit� de Normatiza��o do Mapeamento Cadastral (CNMC) da Comiss�o Nacional de Cartografia (CONCAR), com o objetivo de propor normas para mapeamento cadastral, mas com atua��o limitada nos �ltimos anos. Recentemente, a CONCAR prop�s a nova vers�o da Especifica��o T�cnica para Estrutura��o dos Dados Geoespaciais Vetoriais (ET-EDGV) \cite{concar2016}, que padroniza as estruturas para geoinforma��o de refer�ncia a partir da escala 1:1000. Os dados nessa escala servem de base para o planejamento e gest�o do espa�o geogr�fico urbano. No Brasil, a demanda pelo o mapeamento 3D urbano ainda � baixa e enfrenta desafios que v�o al�m de sua normatiza��o.

As documented in this section, the state-of-art in automatic 3D urban reconstruction covers areas with moderate modeling, whose architectural styles are very specific, streets with large spacing, or symmetry between facade elements, which makes the creation of automatic methods a bit more feasible \cite{vangool2013}. In countries where there is a high density of buildings, such as Brazil, India, or China, this factor is aggravated by the urban geometry. Many of the Brazilian cities do not have a specific style. In suburbs, for example, this factor can prove even more aggravating, where settlement areas or subnormal settlements (in Portuguese, \emph{favelas} -- see Figure \ref{facade-layoutsg}) are all built under these circumstances, with high density and sometimes, erected irregularly or over risky areas, which makes them even more prominent in mapping. 

Initiatives such as the T�NoMapa, by Grupo Cultural AfroReggae\footnote{Available at https://www.afroreggae.org/ta-no-mapa/. Accessed \today.}, together with the North American company Google\texttrademark, consist of mapping hard-to-reach areas such as streets with narrow paths, cliff, among others, by the own local residents. Such areas, in addition to being geometrically complex, require not only cooperation from the government but also from the community who lives there, which by social or security reasons, may require some consent.
%Como documentado neste cap�tulo, o estado da arte na reconstru��o 3D urbana aut�noma abrange �reas com modelagem moderada, cujo estilos arquitet�nicos s�o bem espec�ficos, ruas com grande espa�amentos, simetria entre os elementos de fachadas, por conseguinte, a cria��o de m�todos autom�ticos torna-se fact�veis. Em pa�ses em que h� elevada densidade de edifica��es, como o Brasil, este fator se agrava devido a geografia e geometria urbana locais. Muitos dos centros urbanos brasileiros n�o possuem um estilo espec�fico, o que dificulta a modelagem das edifica��es. Em sub�rbios, por exemplo, esse fator � ainda mais agravante. �reas de assentamentos ou de aglomerados subnormais (favelas), numerosos nessas regi�es, possuem densidade elevada, por vezes, erguidos irregularmente ou sobre �reas de risco, o que as colocam ainda mais em evid�ncia no mapeamento 3D. Iniciativas como o projeto T�NoMapa\footnote{Available in: https://www.afroreggae.org/ta-no-mapa/. Access in: \today.}, do Grupo Cultural AfroReggae, em conjunto com a empresa Norte Americana Google\texttrademark, consiste em mapear �reas de dif�cil acesso, como ruas, vielas, com�rcio e pontos de interesse das comunidades locais, levantamentos �ptico terrestre realizados pelos pr�prios moradores. �reas como essas, al�m de geometricamente complexas, exigem n�o s� colobora��o do governo como tamb�m das comunidades que, por quest�es sociais ou de seguran�a, podem requerer certo consentimento. 

Even though it faces many obstacles, Brazilian urban mapping is moving towards more sophisticated levels. In 2016, the National Civil Aviation Agency (in Portuguese, \emph{Ag�ncia Nacional de Avia��o Civil} (ANAC)) regulated the use of UAVs for recreational, corporate, commercial or experimental use (Brazilian Civil Aviation Regulation, in Portuguese, \emph{Regulamentos Brasileiros da Avia��o Civil} (RBAC), \textquotedblleft portaria\textquotedblright~E~n\textordmasculine~94 \cite{anac-portaria}). The regulation, widely discussed with society, associations, companies and public agencies, establishes limits that still follow the definitions established by other civil aviation entities such as the Federal Aviation Administration (FAA), the Civil Aviation Safety Authority (CASA) and the European Aviation Safety Agency (EASA), regulators from the United States, Australia and the European Union, respectively \cite{anac}. Thus, close-range acquisitions through the use of UAVs became feasible and have fostered research in these fields.
%Ainda que com muitos obst�culos, o mapeamento urbano brasileiro encaminha-se para n�veis mais sofisticados. Em 2016, a Ag�ncia Nacional de Avia��o Civil (ANAC), regulamentou o uso de VANTs para uso recreativo, corporativo, comercial ou experimental (Regulamento Brasileiro de Avia��o Civil Especial - RBAC E n\textordmasculine 94 \cite{anac-portaria}). O regulamento, amplamente discutido com a sociedade, associa��es, empresas e �rg�os p�blicos, determina limites que ainda seguem as defini��es estabelecidas por outras entidades de avia��o civil como \emph{Federal Aviation Administration} (FAA), \emph{Civil Aviation Safety Authority} (CASA) e \emph{European Aviation Safety Agency (EASA)}, reguladores dos Estados Unidos, Austr�lia e da Uni�o Europeia, respectivamente \cite{anac}. Com isso, aquisi��es de curto alcance pelo uso de VANTs tornam-se vi�veis e fomenta a pesquisa nessas �reas.

\section{Geometry extraction}    
\subsection{Structure-from-Motion (SfM)}\label{sfm}
The drawbacks of 3D reconstruction by Photogrammetry are not so many when taking in account its advantages \cite{remondino2005}. In the past decades, close and long-range Photogrammetry have become a widely used tool in 3D city modeling \cite{remondino2006, fraser2009}. Its low costs and availability in remote areas increased its demand, especially in the analysis of such environments. The SfM technique is based on the same physical principles of Stereoscopy, in which a structure can be reconstructed from a series of images taken at small positional variations and overlapping (Figure \ref{sfm-camera}). In conventional Photogrammetry, however, both the position and orientation of the sensor must be known. By the SfM technique, the positional parameters can be estimated. In contrast, point clouds generated by SfM have coordinate system in 3D local reference, or image space, which must later be aligned to the coordinate system of the object \cite{westoby2012}. 
\begin{figure}[!htp]
    \centering 
    \caption{Structure-from-Motion and camera projection. Instead of a single stereo pair, the SfM technique requires multiple, overlapping images as input to feature extraction and 3D reconstruction algorithms.}
    \vspace{6mm}
    \includegraphics[width=1\textwidth]{\figspath/sfm/schema/camera-sfm.png}         
    \vspace{-6mm}
	\legenda{}	
    \label{sfm-camera}       
    \FONTE{Adapted from \citeonline{westoby2012}.}   
\end{figure}

The workflow to obtain the point cloud using SfM begins with the detection of corners in all input images. For every point, a descriptor based on Histogram Oriented Gradient (HOG) is attributed. Then, a set of descriptors from one image are correlate to another one. The ones remaining are called keypoints. The keypoints are used as descriptors in a correspondence analysis among other images, also known as Image-Matching (IM), where pairs are formed according to the level of correlation between each descriptor (i.e., which feature points in several different images depict the same 3D point on the original object). Once the matching has been concluded, the camera orientation and position estimation is started. With the estimated parameters, the matching points now have depth, which together provides detailed structural information of the entire imaged environment. 

The camera position and orientation are estimated iteratively in a process called Bundle Adjustment (BA) \cite{triggs1999}, which in turn, requires a set of overlapping images sharing a set of measurements of point images at \textquotedblleft von Gruber\textquotedblright~positions \cite{agouris2004}. The procedure consists of refining the 3D coordinates based on the number of overlapping images and the intersection of the light rays from each point image and camera projection center. The level of overlap and alignment, therefore, determine the quality of estimation (geometric accuracy) of the point cloud in any SfM methodology \cite{snavely2008, hofer2016}.

Different approaches for SfM pipelines have change the way how to 3D reconstruct cities. As summarized in the Table \ref{tabela-dado-estrutural}, the main contribution for 3D reconstruction by SfM reside in the fact that geometry can be acquired only using optical images, specialists on image acquisitions are no more required since the pictures available on internet could provide sufficient dataset for 3D modeling. For instance, the Bundler \cite{snavely2008} takes an unorganized image collection as input, then it is able to reorganize the collection in a way that each pair correspond to an overlapped image. 

This approach solves, for example, the question of the physical presence of the human operator in the places of interest. Although it only solves the problem mostly at touristic sites, the idea of using images from ordinary users has stimulated areas such as Cartography, Photogrammetry and Remote Sensing since then. Another example, the VarCity project\footnote{Details about the project are available at https://varcity.ethz.ch/. Accessed \today.}\cite{hayko2014}, which uses information from a variety of internet sources and integrates them to reconstruct every scenario in 3D.

Keypoints are simply distinctive points denoting the image identity, areas with optimal characteristics for correlating two images, free of spectral and spatial variations, for example, sharp edges and corners. Fundamental in the areas of Computer Vision, Photogrammetry and Remote Sensing, the concept of corner detection is old and is based on the analysis of images in differential space. The issue, in this case, is how to detect corners in one image and at the same time correlate it with another one, which in turn can present variations in scale, texture or orientation? First, \citeonline{harris1988} proposed to find keypoints using measures based on eigenvalues of smoothed gradients, which allowed the detection of corners independently of rotation, translation and brightness, but not in depth (scale). Then, in 2004, \citeonline{lowe2004} proposed the Scale Invariant Feature Transform (SIFT) method, solving the scale problem and also adding the image matching resolution.

\citeonline{yuan2017} categorizes the matching algorithms in two segments, the ones based only in radiometric information and those based either on radiometric and geometric information. Respectively examples of approaches are the Normalized Cross Correlation (NCC) \cite{gonzalez1992}, Scale Invariant Feature Transform (SIFT) \cite{lowe2004}, and the Distinctive Order Based Self-Similarity (DOBSS) \cite{sedaghat2015}, which stand in the high quality of the images. The second group of algorithms include the Semi-global Matching (SGM) \cite{heiko2008}, Path-based Multi-View Stereos (PMVS) \cite{furukawa2007pvms}, and Multi-photo Geometrically Constrained Matching (MGCM) \cite{zhang2006}. These last algorithms take advantage not only from fine radiometric information, but also from the predetermined geometric information, functions as a supplementary matching constraint.

Among these, SIFT has become the most popular. SIFT is independent of scales, variations of brightness, contrast and orientation. After identified, the matching points are determined in pairs of images according to the correlation of local descriptors in each point. A common descriptor is the use of the HOG, which considers the magnitude and orientation of its neighbors as an identifier in the matching process. In the case of low correlation, these are discarded by the algorithm. The size and quantity of images are also part of the problem in the identification of keypoints, depending, demanding days of processing. More sophisticated procedures have improved SIFT by the use of GPUs (Graphic Processing Unit)\footnote{Electronic circuit dedicated to high-performance graphics processing.}, for example, SiftGPU \cite {wu2007}.

Some of the SfM stages have been improved over the years. Examples of these transformations are the recent use of BA for the iterative adjustment of camera positioning, the use of SiftGPU \cite{wu2007} for computational optimization for image-matching and Multicore Bundle Adjustment (MBA) \cite{wu2011}, the technique Clustering Views for Multi-View Stereo (CMVS) \cite{furukawa2010cmvs}, which the entries are decomposed into smaller processing groups in order to enable parallel processes, PMVS-2 \cite{furukawa2010pmvs2}, an improved version of PMVS, responsible for the automatic reconstruction of only rigid objects, pedestrians, moving cars and others. In Figure \ref{sfm-workflow}, SfM is summarized. The post-processing phase is composed of a dense point cloud refinement, where points can be either simplified (points reducing), filtered (noise removal), or used as input for surface reconstruction (e.g. Poisson \cite{hoppe2008}) and texturing.
\begin{figure}[!htb]
    \centering     
    \caption{Most recent Structure-from-Motion pipelines. From photographs to sparse point-clouds.}    
    \vspace{6mm}
    \includegraphics[width=0.78\textwidth]{\figspath/sfm/flow/sfm-flow-en.pdf}
    \vspace{2mm}
    \legenda{}
    \label{sfm-workflow}       
    \FONTE{Adapted from \citeonline{yuan2017, westoby2012}.}    
\end{figure}


% The term SfM became popular on the reconstruction of 
% MVS assumes known camera calibration, both internal and external (i.e. camera poses), whereas SfM computes camera poses together with reconstructing the environment (the sparse point cloud).

%\cite{hayko2014} exploram a geometria de representa��es 3D urbanas (\emph{mesh}) para a classifica��o e constru��o de sistemas sem�nticos autom�ticos a partir de imagens terrestres (\emph{street-side images}). No estudo, � proposto a otimiza��o da detec��o de fei��es de fachadas, com o objetivo principal de reduzir o n�mero de images para aquisi��o da geometria, consequentemente, a redu��o do custo computacional na extra��o de informa��es. A estrat�gia de detec��o atinge n�veis significativos de acur�cia considerando a complexidade da cena.

%Algumas das etapas que comp�em a t�cnica SfM foram aperfei�oadas ao longo dos anos. Exemplos dessas transforma��es, � o uso recente de BA para o ajuste iterativo do posicionamento das c�meras, o uso do SiftGPU \cite{wu2007} para otimiza��o computacional para correspond�ncia entre imagens (\emph{image-matching}) e MBA (do ingl�s, \emph{Multicore Bundle Adjustment} - (MBA)) \cite{wu2011}, para suporte computacional � estimativa de par�metros de c�mera, da t�cnica Agrupamento de Visadas para Imagens Estereo (do ingl�s, \emph{Clustering Views for Multi-view Stereo} -  (CMVS)) \cite{furukawa2010cmvs}, em que as entradas s�o decompostas em grupos de processamento menores a fim de viabilizar processos paralelos, do PMVS2 \cite{furukawa2010pmvs2}, respons�vel pela reconstru��o autom�tica somente de objetos r�gidos, excluindo, por exemplo, pedestres, carros em movimento e outros. Na Figura \ref{sfm-workflow} � ilustrado o resumo do SfM descrito acima. A fase de p�s-processamento � composta pelo refinamento da nuvem de pontos densa, em que os pontos podem ser simplificados (redu��o do n�mero de pontos), filtrados (remo��o de ru�dos), ou utilizados como entrada para reconstru��o de superf�cie (e.g. Poisson \cite{hoppe2008}) e texturiza��o.
%\begin{figure}[!htp]
% \centering     
% \includegraphics[width=1\textwidth]{\figspath/sfm/flow/sfm-flow-en.png}     	  
% \caption{Most recent Structure-from-Motion pipelines. From photographs to sparse and dense point-clouds.\\ \textbf{Source.} Adapted from \citeonline{yuan2017, westoby2012}.}
% \label{sfm-workflow}       
%\end{figure}

%A import�ncia no levantamento volum�trico de ambientes urbanos se concentra, principalmente, na possibilidade de construir modelos mais real�sticos, agregando informa��es de volume e ocupa��o do espa�o e com isso analisar as poss�veis influ�ncias desse efeito. Uma vez extra�das informa��es precisas quanto � localiza��o e volume de objetos urbanos, estudos como a identifica��o de ilhas de calor, navega��o, turismo, entretenimento, no aux�lio ao planejamento urbano, controle de tr�fego, insola��o de grandes edifica��es, fiscaliza��o de constru��es irregulares ou an�lise de propriedades, valida��o de planos diretores, entre outras, tornam-se poss�veis ao passo que a investiga��o dos potenciais impactos ao ambiente necessitam n�o s� das caracter�sticas qu�micas ou biol�gicas dos objetos, como tamb�m de suas informa��es estruturais.  

%Outra aplica��o da representa��o digital 3D urbana � a possibilidade de simula��o de eventos naturais din�micos como terremotos, movimentos de massa, expans�o urbana, inunda��es, condi��es de microclima, circula��o do ar entre altos edif�cios, estudos de aptid�o de s�tio para aeroportos, tra�ado de metr�, identifica��o de zonas de sombras de telefonia m�vel, defini��o de rota para helic�pteros, entre outros que viabilizariam estimar informa��es confi�veis acerca das consequ�ncias e cen�rios alternativos de ocupa��o do solo urbano em casos de ocorr�ncias reais desses eventos. Al�m disso, as informa��es de volumetria amparam setores como o turismo, ao gerar formas mais real�sticas do espa�o urbano e, ao mesmo tempo, agregando m�dias de acesso para usu�rios; a arquitetura e urbanismo, ao tornar o uso dessas m�dias uma importante ferramenta no setor de planejamento urbano, ao dispor de f�runs \emph{online} de participa��o popular (denominadas plataformas de planejamento urbano participativo) com o objetivo de fortalecer o entendimento das a��es e pol�ticas � popula��o \cite{almeida2007}.

%Coberturas completas de cidades pelo sistema de mapeamento m�vel j� s�o oferecidos. Entretanto, s�o ainda relativamente caras \cite{haala2015}. 

%Realizados em pequena escala, assim como o mapeamento interno de edifica��es (\emph{indoor}), tamb�m realizado por sensores de Luz-Estruturada (\emph{Structured-Light}, conhecido como Kinect) \cite{yue2014}.

% \cite{remondino2006}
%The 3D modelling  of an object can be seen as the complete process that starts from data acquisition and ends with a 3D model. Often 3D modelling is meant only as the process of converting a measured point cloud into a triangulated network (��mesh��) or textured surface, while it should describe a more complete and general process of object reconstruction. 3-dimensional modelling of objects and scenes is an intensive and long-lasting research problem in the graphic, vision and photogrammetric communities \cite{remondino2006}.

%Nowadays, the generation of 3D models are mainly archieved using remote systems, either by active or passive sensors, embedded on a long or close-range acquision platforms, which is determined often by the application purposes.  

%Laser scanning (LiDAR) and digital imagering are key technologies which sometimes compete with each other, but are quite complementary \cite{fritsch2013}

%As conquistas mais recentes em termos de automatiza��o na gera��o de modelos tridimensionais urbanos s�o baseados na integra��o de duas ou mais fontes de dados \cite{rottensteiner2003a}, como a integra��o de dados LiDAR e �pticos \cite{steger1998, couloigner2000, tatem2001, dell2001, csatho2003, rottensteiner2003a, rottensteiner2004}, LiDAR e multiespectrais \cite{brunn1997, hug1997, haala1999, vosselman2002} ou LiDAR e InSAR (Interferometria SAR) \cite{stilla2001, stilla2003, dowman2004}. Notoriamente, estudos envolvendo a volumetria de objetos urbanos frequentemente remetem ao uso de dados estruturais.

%N�o h� na literatura hoje uma metodologia consagrada que determine um roteiro �timo de gera��o desses modelos. Nota-se, por�m, um padr�o de processamento compreendendo dois est�gios fundamentais: detec��o e reconstru��o \cite{rottensteiner2003a, sohn2007, shan2007, beraldin2010}. 

%In a spatial searching for tall objects, for example, possibly results would be trees, antennas, buildings, and other many. However, with a more detailed searching, such as linear edge objects, we could eliminate all those possibilities respect to heterogenous surface, which would lead in a more specific range of alternatives. Eliminate possibilities based on its constraints eitheir its physical characteristics is a common rule and has been used along the years in different ways and fields.



%Nos �ltimos anos, a tecnologia de imageamento por cameras obliquas tem sido amplamente utilizada na modelagem tridimensional. Com a vantagem de ter mais informa��es de fachadas, imagens obliquas produzem modelos 3D de edifica��es ainda mais completos e real�sticos \cite{zhu2015}

% Comparisons between range-based and image-based modelling are reported in B�hler and Marbs (2004), Kadobayashi et al. (2004), B�hler (2005) and Remondino et al. (2005)

%O uso de fonte de dados LiDAR e �pticos possuem suas respectivas vantagens e desvantagens para a detec��o de objetos. Dados LiDAR fornecem coordenadas tridimensionais e, geralmente, nenhuma infroma��o espectral*. Dados �pticos, por       outros lado, possuem uma rica informa��o espectral, o que em regi�es urbanas torna-se facilmente uma disvantagem devido a alta varia��o de propriedades espectrais \cite{demir2012}         

%\cite{leberl2010} compara os dois tipos de imageamento por laser e �ptico


%     Esses dispositivos possuem um papel importante em diversas �reas de aplica��o, sobretudo, na cartografia, tendo suas principais contribui��es no in�cio do s�culo 21 \ref{REF}, quando os primeiros aerolevantamentos sobre cidades foram realizados \ref{REF}. 
% 
%       A amostragem dos dados � fun��o da dist�ncia do alvo at� o sensor, da orienta��o do sensor, assim como a geometria da fei��o \cite{berger2014}
% 
%       A distribui��o dos dados � comumente fun��o dos artefatos da aquisi��o tais como o ru�do do sensor, a dist�ncia do alvo ou orienta��o do sensor em rela��o a superf�cie \cite{berger2014}
% 
%       A presen�a de outliers (pontos que encontram longes e isolados da superf�cie) � devido a artefatos estruturais no processo de aquisi��o \cite{berger2014}
% 
%       As imperfei��es quanto ao registro dos diferentes ranges de aquisi��o (diferentes faixas de sobrevoo) � devido principalmente ao algoritmo de alinhamento utilizado \cite{berger2014}
% 
%       A aus�ncia de dados � devido ao alcance do sensor, superf�cies ou regi�es com alta absor��o da radia��o ou oclus�o por objetos no processo de aquisi��o \cite{berger2014}
% 
%       Como resultado da aplica��o de sensores LiDAR terrestre, a nuvem de pontos adquirida tipicamente exibe perda significativa de dados devido a oclus�o, al�m disso, nuvem de pontos irregular \cite{nan2010}
% 
%       Iterative Closest Points (ICP) algoritmos s�o utilizados para registrar nuvem de pontos adquiridas por diferentes sensores. \cite{pu2009}
% 
%       A reconstru��o 3D de edifica��es a partir de sensores laser normalmente apresenta erros, devido a superamostragem do sensor e limita��es quanto a algoritmos de modelagem \cite{pu2009}
% 
%       Imagens �pticas s�o prec�rias no quesito de acur�cia geom�trica, especialmente quando comparada com dados LiDAR, que por sua vez permite adquirir diretamente o espa�o de coordenadas dos objetos \cite{yang2016}

% \textbf{a)} Quais e quantas fontes de dados s�o necess�rias para que a reconstru��o volum�trica em alto n�vel de detalhamento seja poss�vel?

% Com a utiliza��o de dados ALS, normalmente a densidade de pontos pode atingir at� 12 $p/m^2$, dependendo do sensor utilizado e geometria da aquisi��o (�ngulo de visada  e altitude do sensor). A aquisi��o por sensor aerotransportado � comumente feita \emph{on-nadir}. No entanto, alguns estudos t�m utilizados dados ALS, adquirido \emph{off-nadir} com angula��o elevada (c�mera obl�quoa) para estudos de fachadas.

% Outra alternativa � a utilizada��o de dados terrestres, podendo ser adquiridos por plataformas fixas ou m�veis. Dados TLS fornecem alt�ssimo n�vel de densidade de pontos. Dependendo do sensor e dist�ncia do alvo, a nuvem de pontos pode apresentar desensidade superior a 16 $p/m^2$

% 3. Quais destas fontes est�o acess�veis, quais n�o est�o?
% Grande parte devem-se a aquisi��o privada de informa��es, principalmente, as que se referem a dados terrestres

%Sensores terrestres podem ser divididos em fixos ou din�micos, como mostrado na Figura \ref{sensoreslidar}. No primeiro, as medidas s�o realizadas em tr�s diferentes modos: (i) Panor�mico (Figura \ref{lidar-terrestrial-view-a}), limitado somente pelo campo de visada de sua pr�pria base; (ii) H�brido (Figura \ref{lidar-terrestrial-view-b}), neste caso, o �ngulo vertical � limitado � uma determinada abertura; e (iii) C�mera (Figura \ref{lidar-terrestrial-view-c}), possuindo limita��es em ambos os �ngulos, tanto vertical quanto azimutal \cite{staiger2003}. Sensores terrestres din�micos s�o normalmente acoplados em suportes na parte superior dos ve�culos, possuem o mesmo princ�pio do escaneamento fixo, acrescido do sistema de posicionamento global GNSS (\emph{Global Navigation Satellite System}) e por uma unidade de medida inercial (\emph{Inertial Measurement Unit} - IMU), que permitem posicionar e orientar o ve�culo em um determinado referencial geod�sico (Figura \ref{lidar-terrestrial-view-d}).

%  Em aplica��es envolvendo a extra��o volum�trica urbana, a demanda por nuvem de pontos LiDAR com densidade alta (acima de 5 $p/m^2$) � pertinente, justamente pela necessidade em discriminar as in�meras classes poss�veis neste ambiente. Tais objetos se inserem frequentemente em contextos complexos, em que s�o sobrepostos ou estejam muito pr�ximos entre si, por exemplo, copas de �rvores com folhagem densas recobrindo telhados ou fachadas, arboriza��es muito pr�ximas �s estruturas artificiais de interesse, como pontes, edifica��es, entre outros. Portanto, � medida que se diminui a densidade dos pontos, a separabilidade dessas e outras classes torna-se ainda mais trabalhosa.

%Sensores a bordo de plataformas orbitais possuem velocidade e dist�ncia do alvo elevadas, aproximadamente $30.000~km/$ e $250.000~m$, respectivamente, que, neste caso, exigem frequ�ncias mais altas de opera��o (\emph{Pulse Repetition Frequency} - PRF), custo elevado, entre outras limita��es t�cnicas que tornam sua utiliza��o em estudos de ambientes urbanos invi�vel \cite{tack2012}.        

%     In the first part of this section, the principles surrounding the the image-based point cloud generation are introduced. The second and third part are  respectively present a review of approaches about long and close-range image-based building and facades reconstruction.

%In \citeonline{remondino2006}, a image-based 3D modelling systems review is presented. Ten years later, the a comparison regarding this review could 

%photographic interpretation is concerned with the examination of photographs in order to identify objects, and is an essential part of air survey

%Aerial Photogrammetry
%using photographs taken from the air or from space with the camera usually pointing vertically downwards
%the term air survey is used to describe survey techniques using photographs taken from the air or from the space

%Close-range Photogrammetry
%using photographs taken on the ground with the camera usually pointing in a horizontal direction

%Digital imaging has arisen in the 1980s when the first CCD video cameras were used in machine vision applications []. Since 2000, digital airborne camera systems have replaced film-based Photogrammetry. Although first image matching algorighms have been invented in the 1980s their results could not compete with LiDAR point cloud at all. Since 2011 Photogrammetry is having a renaissance by using Semi-Global Matching algorithms \cite{fritsch2013}

% Take form internet -- adapt
% - Photogrammetry is a relatively old technique for measuring and processing lengths and angles in photographs for mapping purposes \ref{slama1980}


%     LoD2 normalmente � suficiente para simula��es ou visualiza��es em pequena e m�dia escala \cite{haala2015}
%     
%     \cite{micusik2014} apresentaram o m�todo SLAM, SfM tamb�m baseado em fei��es lineares, em que � poss�vel reconstruir cenas em larga escala. Bons resultados para ambientes fechados
%     
%     A t�cnica de image-matching em nuvem de pontos possui problemas sobre objetos reflectivos \cite{tutzauer2015}
%     
%     \cite{zhang2014} propuseram um m�todo sofisticado de SfM baseado em fei��es lineares
%  
%         \cite{hofer2015} descrevem o princ�pio do m�todo Line3D. Line3D++ � sua extens�o. Line3D++ utiliza LSD \cite{von2010} para obter os segmentos de linha. Somente segmentos longos s�o considerados, baseando-se em um determinado limiar \cite{hofer2016}
%     
%         Line3D++ utiliza como m�todo SfM o ICG3D \cite{irschara2007}
% 
%         Em \cite{hofer2016}, � apresentado o sistema Line3D++ para gera��o de modelos 3D urbanos sem�nticos 
% 
%         SfM: Sparse 3D model (344 imagens, gerou-se 97.699 pontos em 0.75 horas) \cite{lowe2004}
%         PMVS: Semi-dense point cloud (344 imagens, gerou-se 12.156.644 pontos em 11 horas) \cite{furukawa2010}
%         Line3D++: 3D Lines (344 imagens, gerou-se 7.877 pontos em 210 segundos)
% 	
%      Devido a complexidade e diversidade em ambientes urbanos, erros de calibra��o em cameras podem facilmente serem gerados no processo de image-matching, o que poduziria ru�do e nuvem de pontos imprecisa comparada a uma nuvem de pontos LiDAR \cite{zhu2015} 

%Informa��es 3D n�o s�o facilmente obtidas por imagens MVS (Multi-View Stereo) devido a necessidade de calibra��o e registro de imagens \cite{lafarge2015}

%Atualmente, gra�as a disponibilidade de softwares livres ou �s sofisticadas ferramentas de processamento, at� mesmo usu�rios n�o especialistas conseguem manusear e gerar modelos 3D \cite{hofer2016}

%Proposes an efficient approach for building compact, edge-preserving, view-centric triangle meshes from either dense or sparse depth data, with a focus on modeling architecture in large-scale urban scenes

% Definition
%Multi-View Stereo (MVS) algorithms reconstruct a surface (or volume) from multiple calibrated images, where the calibration is typically provided automatically via SfM.
%     \subsection{Diferen�as LiDAR e Stereo-images}
%     http://geoawesomeness.com/drone-lidar-or-Photogrammetry-everything-your-need-to-know/?platform=hootsuite

%Some recent MVS algorithms developed in Simultaneous Localization and Mapping (SLAM) frameworks can run in real-time on the GPU [1], CPU [2], or on top-grade mobile phones [3]
%  \textbf{a)} Newcombe, R.A. , Davison, A.J. , 2010. Live dense reconstruction with a single moving camera. CVPR. 
%	Pizzoli, M. , Forster, C. , Scaramuzza, D. , 2014. REMODE: Probabilistic, monocular dense reconstruction in real time. ICRA
%Newcombe, R. A. , Lovegrove, S. J. , Davison, A. J. , 201a) DTAM: dense tracking and mapping in real-time. ICCV .
%      2. Geiger, A. , Ziegler, J. , Stiller, C. , 201a) Stereoscan: Dense 3D reconstruction in real�time. Intelligent Vehicles Symposium (IV) .
%      3. Tanskanen, P. , Kolev, K. , Meier, L. , Camposeco, F. , Saurer, O. , Pollefeys, M. , 2013. Live metric 3D reconstruction on mobile phones. In: ICCV, pp. 65�72 .


\section{Deep Learning (DL)}\label{deep-learning}
In terms of image labeling, years of advances have brought what is now considered a gold-standard in segmentation and classification: the use of DL. The technology is the new way to solve old problems in Remote Sensing \cite{audebert2017}. It is one of the branches of ML that allows computational models with multiple processing layers to learn representations in multiple levels of abstraction. The term \textquotedblleft deep\textquotedblright, refers to the amount of processing layers that is usually used during training. 
%Em termos de detec��o autom�tica de objetos, anos de avan�os trouxeram o que hoje � considerado o gold-stardard na segmenta��o e classifica��o de imagens: a utiliza��o de Deep-Learning . \emph{Deep Learning} � uma nova forma de resolver velhos problemas em Sensoriamento Remoto \cite{audebert2017}. � um dos ramos do Aprendizado de M�quina (\emph{Machine Learning} - ML) que permite modelos computacionais que comp�em m�ltiplas camadas de processamento a aprenderem representa��es de dados com m�ltiplos n�veis de abstra��o. O termo \emph{deep}, refere-se portanto � quantidade de camadas de processamento. 

\citeonline{mitchell1997} succinctly defines the learning from a ML as: \textquotedblleft A computer program learns from a given experience $E$ relative to application classes $T$ with a performance $P$, if the performance on task $T$ is better than in $E$ \textquotedblright. Widely categorized as unsupervised or supervised, ML is defined by the type of experience that can be applied during the learning process. Shortly, unsupervised learning does not require samples during the training process (learning occurs with the dataset itself), while the supervised assume its behavior according to a reference set. \citeonline{lecun2015} believe that unsupervised learning will become far more important in the longer term, since machine learning tends to look more like human learning, which in turn has its learning largely unsupervised: \textquotedblleft \textit{they discover the structure of the world by observing it, not by being told the name of every object}\textquotedblright.
%\citeonline{mitchell1997} define sucintamente aprendizado de m�quina como: \textquotedblleft Um programa de computador aprende a partir de uma determinada experi�ncia $E$ relativa � classes de aplica��es $T$ com um desempenho $P$, se seu desempenho sobre as tarefas $T$, tal como medido em $P$, for melhor em $E$\textquotedblright.  Amplamente categorizado como n�o-supervisionado ou supervisionado, o aprendizado de m�quina � definido pelo tipo de experi�ncia que pode ser aplicada durante o processo de aprendizagem. Em suma, o aprendizado n�o-supervisionado n�o requer exemplos durante o processo de treinamento (aprendizado ocorre com o pr�prio conjunto de dados), enquanto o segundo, assume seu comportamento conforme o conjunto refer�ncia a ser apresentado \cite{goodfellow2016}.

These models have made remarkable advances in the state-of-art of pattern recognition, speech recognition, detection of objects, faces, and others. Shortly, DL architectures are trained to recognize structures in a massive amount of data using, for example, supervised learning with the concept of backpropagation\footnote{Backpropagation is a method commonly used in ML to calculate the error contribution of each neuron after each training iteration.}, where the model is indicated the proportion of what must be changed in each of its layers until \textquotedblleft learning\textquotedblright~occurs (error decay) \cite{lecun2015}.
%\emph{Deep Learning} � uma nova forma de resolver velhos problemas em Sensoriamento Remoto \cite{audebert2017}. � um dos ramos do Aprendizado de M�quina (\emph{Machine Learning} - ML) que permite modelos computacionais que comp�em m�ltiplas camadas de processamento a aprenderem representa��es de dados com m�ltiplos n�veis de abstra��o. O termo \emph{deep}, refere-se portanto � quantidade de camadas de processamento. Esses modelos permitiram avan�os not�veis no estado-da-arte do reconhecimento de padr�es, de fala, na detec��o de objetos, faces e muitos outros dom�nios. Em suma, DLs s�o treinadas a reconhecer estruturas em uma quantidade massiva de dados utilizando, por exemplo, aprendizagem supervisionada com o conceito de retropropaga��o de erros (\emph{backpropagation})\footnote{\emph{Backpropagation} � m�todo comumente utilizado em RNA para calcular a contribui��o de erro de cada neur�nio ap�s cada itera��o de treinamento.}, em que � indicado ao modelo a propor��o do que deve ser alterado em cada uma de suas camadas at� que o \textquotedblleft aprendizado\textquotedblright ocorra (diminui��o do erro) \cite{lecun2015}. 

%A aprendizagem n�o-supervisionada, tamb�m teve efeito catalisador no conceito DL, mas desde ent�o, tem sido ofuscada pelos �xitos da aprendizagem supervisionada. A longo prazo, \citeonline{lecun2015} acreditam que a aprendizagem n�o-supervisionada tornar�-se muito mais importante devido a sua capacidade em moldar-se a natureza. A aprendizagem humana e animal � em grande parte n�o-supervisionada, em que a aprendizagem ocorre ao descobrir-se a estrutura do mundo, observando-a, n�o necessariamente informando o nome de cada objeto. 
Unsupervised learning had a catalystic effect in reviving interest in DL, since then, it has been overshadowed by the successes of supervised learning. \citeonline{lecun2015}, on the other hand, believes that unsupervised learning will become far more important in the longer term, as the technological tendency is to get closer and closer to human behavior. %Human and animal learning is largely unsupervised: they discover the structure of the world by observing it, not by being told the name of every object.

\subsection{Convolutional Neural Network (CNN)}\label{cnn-sect}
In 1943, the first Artificial Neural Network (ANN) appeared \cite{pitts1943}. With only a few connections, the authors were able to demonstrate how a computer could simulate the human learning process. In 1968, \cite{hubel1968} proposed an explanation for the way in which mammals visually perceive the world using a layered architecture of neurons in their brain. Then, in 1989, a neural model started to get attention not only because of its results, but also for its similarities with the biological visual system, with processing and sensation modules. 

\citeonline{lecun1989} presented a sophisticated neural model to recognize handwritten characters, named first LeNet, then later, Convolutional Neural Network (CNN). Precisely, the reason for this name is due the successive mathematical operations of image convolutions. Since then, many engineers have been inspired by the development of similar algorithms for pattern recognition in Computer Vision. Different models have emerged and contributed to the evolved state of neural networks in the present days. In Figure \ref{cnn-arch}, the LeNet neural model, developed by Yann Lecun in 1989 \cite{lecun1989}.
%Em 1943 surgia a primeira Rede Neural Artificial (RNA) \cite{pitts1943}. Apenas com poucas conex�es, os autores demonstraram como simular computacionalmente o aprendizado humano. Em 1968, \cite{hubel1968} propuseram hip�teses de como os mam�feros percebiam visualmente o ambiente ao seu redor baseando-se em arquitetura de camadas de neur�nios. Ent�o, em 1989, um modelo neural chamou a aten��o n�o s� pelos seus resultados em ambientes complexos, mas tamb�m por suas semelhan�as com o sistema visual biol�gico, com m�dulos de processamento e sensa��o. \cite{lecun1989} apresentaram um sofisticado modelo neural para o reconhecimento de caracteres escritos manualmente, nomeada Rede Neural Convolucional (CNN) precisamente pelas sucessivas opera��es matem�ticas de convolu��o aplicada � imagem. Desde ent�o, uma legi�o de engenheiros inspiraram-se no desenvolvimento de algoritmos semelhantes para o reconhecimento de padr�es em Vis�o de Computacional. Diferentes modelos surgiram e contribu�ram para o estado evolu�do das redes neurais nos dias atuais.
\begin{figure}[!htp]
    \centering     
    \caption{LeNet: The first version of a CNN, projected by Yann Lecun in 1989.}
    \vspace{6mm}
    \includegraphics[width=1\textwidth]{\figspath/cnn/lenet/leNet_1.pdf}
    \includegraphics[width=1\textwidth]{\figspath/cnn/lenet/leNet_2.pdf}
    \legenda{}    
    \label{cnn-arch}   
    \FONTE{Adapted from \citeonline{lecun1989}.}
\end{figure}

The CNN is structured in stages, the first ones are composed of two types of layers: convolution and pooling (in the figure, in blue and red). Units in a convolution layer are organized into feature maps or filters, where each unit is connected to a window (also called patch $C_x$ - details in the bottom figure) in the feature map of the previous layer. The connection between the window and the feature map is given by weights ($w$) and biases ($b$). The weighted sum of the convolution operations is followed by a nonlinear activation function, called Rectified Linear Unit (ReLU). For many years, activations in neural networks were composed of smoother functions, such as the $tanh(x)$ or $1/(1+e^{-x})$ sigmoids, but recent study has shown ReLU to be faster on learning in multilayered architectures \cite{lecun2015}. The fully-connected layer corresponds to a traditional Multi-Layer Perceptron (MLP), with hidden layer and logistic regression. Then, the input to the MLP layer is the set of all features maps at the previous one.

CNN have made advances in image, video, speech and audio analysis, while Recurrent Networks (RR) have supported the path to sequential data such as text and speech \cite{lecun2015}. It was designed to recognize visual patterns directly from pixel images with minimal preprocessing. Due to convolution over images and its reduced versions (downsampling), they can recognize patterns with extreme variability, scales, and with robustness to distortions and simple geometric transformations such as handwritten characters. 

%Due its robustness on classify and segment images around complex environments, CNN has been a trend in the variety of Computer Vision applications. Not only, the approach has also been a gold-standard on face recognition \cite{lawrence1997}, speech recognition \cite{abdel2013}, natural language \cite{kalchbrenner2014, ciresan2011}, and others. In image analysis, not the first but the boom of CNNs for images could be nominated to AlexNet approach \cite{krizhevsky2012}, when this technique start to be exhaustive tested, and became a charming and fast way to classify objects on the scene.

Because of its robustness in image classification and segmentation over complex environments, CNN has been a trend in many Computer Vision applications. For example, the approach is considered gold-standard in applications such as face recognition \cite{lawrence1997}, speech recognition \cite{abdel2013}, natural language \cite{kalchbrenner2014, ciresan2011}, and others. In image analysis, not the first, but the reference on the use of CNNs for images can be named to the model AlexNet \cite{krizhevsky2012}, when the technique began to be exhaustively tested and became a practical and fast way in object classification. 

With the success on ordinary image classification, neural models came to be used also in numerous applications involving Remote Sensing \cite{zhu2017}. Examples of that are the analysis of orbital images \cite{castelluccio2015, marmanis2016}, radar \cite{chen2014sar, chen2016, sun2012}, hyperspectral \cite{chen2014, romero2014} and in urban 3D reconstruction \cite{hane2013, blaha2016, blaha2016towards}. Although not focused specifically on the analysis of facades, excellent results have been reported in the classification of urban elements through the use of DL. 

An example of this evolution can be observed in the annual PASCAL VOC \cite{pascalvoc} challenge that, brings together experts to solve classical tasks in Computer Vision and related areas. The applications range from recognition \cite{krizhevsky2012} to environment understanding, where the analysis is focused on the relationship between the object itself. Therefore, certain constraints could be imposed to the relation, for example, between a pedestrian and street or vehicles in applications involving self-driving \cite{segnet, teichmann2016}, such that a distance and speed constraints could be imposed between these detected objects. \citeonline{lettry2017} used CNN to detect repeating features in rectified facade images, wherein the repeated patterns are verified on a projected grid, then, it is used as a device to detect those regular characteristics and reconstruct the scene. 

%Convolutional neural networks (CNNs) are the current state-of-the-art model architecture for image classification tasks. CNNs apply a series of filters to the raw pixel data of an image to extract and learn higher-level features, which the model can then use for classification. CNNs contains three components:
%Convolutional layers, which apply a specified number of convolution filters to the image. For each subregion, the layer performs a set of mathematical operations to produce a single value in the output feature map. Convolutional layers then typically apply a ReLU activation function to the output to introduce nonlinearities into the model.
%Pooling layers, which downsample the image data extracted by the convolutional layers to reduce the dimensionality of the feature map in order to decrease processing time. A commonly used pooling algorithm is max pooling, which extracts subregions of the feature map (e.g., 2x2-pixel tiles), keeps their maximum value, and discards all other values.
%Dense (fully connected) layers, which perform classification on the features extracted by the convolutional layers and downsampled by the pooling layers. In a dense layer, every node in the layer is connected to every node in the preceding layer.
\subsubsection{Autoencoder}
How humans perceive the environment around them is still a topic of recurrent research and, so far, not fully understood. The human visual system consists basically of two processes: radiation capturing and visual perception. The first, numerous chemical and optical processes take place in the eyes of the observer. All visible radiation captured is transformed into synaptic signals by photo-receptor cells, called cones and rods. From this point, these signals representing this lapse moment are then directed to specific regions of the brain, which through chemical reactions and many others to be understood, allow to connect these signals to the perceptual neural senses, which then allows to see and perceive the environment, completing the basic mechanism of the human visual system.
%Como os humanos percebem o ambiente ao seu redor � ainda tema de pesquisa recorrente e, at� o momento, n�o totalmente compreendido. O sistema visual humano consiste basicamente de dois processos: a captura e a percep��o visual. Na primeira, in�meros processos qu�micos e �pticos acontecem nos olhos do observador. Toda radia��o vis�vel captada � transformada em sinais sin�pticos por c�lulas fotoreceptoras, denominadas cones e bastonetes. A partir deste ponto, esses sinais que representam essa imagem s�o ent�o direcionados a regi�es espec�ficas do c�rebro, que por rea��es qu�micas e outras muitas a serem compreendidas, permitem conectar tais sinais � sentidos neurais perceptivos, que ent�o possibilita ver e perceber o ambiente e seu entorno. Completando o mecanismo b�sico do sistema visual humano e similares.
%How humans perceive the environment is still a subject of research and not fully understood so far. However, some new advances has shown remarkable results. The human vision system consists basically of two steps, the capture and visual perception. In the first, all radiation on its field view is captured by the visual system, which by numerous chemical and optical procedures process this radiation in signals, which is pretty much the description of that environment; from this point on, these signals representing this environment are then directed to specific regions of the brain, which by chemical reactions and many others, allows to connect these signals to the perceptive senses, completing the basic mechanism of the human visual system. 

CNNs can assume different processing mechanisms, different numbers of layers and how they are connected. The architecture called Autoencoders or encoder-decoder \cite{segnet, teichmann2016}, presents similarity not only in architecture but also in the behavior of human visual system. As described in the previous paragraph, two basic mechanisms can be noticed in the biological system: one for processing the radiation in synaptic signals by photo-receptor neurons; and another, for environmental perception, mainly performed by the visual cortex. 
%CNNs podem assumir mecanismos diferentes de processamento, n�meros diferentes de camadas e como s�o conectadas. A arquitetura denominada \emph{autoencoders} ou \emph{encoder-decoder} \cite{segnet, teichmann2016}, apresenta  semelhan�a n�o s� na arquitetura como tamb�m no comportamento do sistema visual humano. Em suma, dois mecanismos b�sicos podem ser notado no sistema biol�gico: um para processamento da radia��o em sinais sin�pticos por neur�nios fotoreceptores; e outro, para percep��o do ambiente, cortex visual. Analogamente, o mecanismo \emph{encoder} e \emph{decoder} simula respectivamente este processo. 

Similarly, the encoder and decoder mechanism respectively simulate this two process. First, the input image is subjected to numerous convolution and pooling operations. For each pooling, the downsampling occurs, where the image resolution is reduced and passed on as input to the subsequent convolution layer. Each convolutional layer has a set of convolutional filters, also called filter bank, each of these looks for specific features on image, such as horizontal lines, vertical, curved ones, faces, cars, and others. Consequently, the filter bank learns different patterns in different scales, orientation and location (functions similar to those of the biological lens and retina) as it goes down on deep layers. 

Signals in small dimensions are then transmitted to the decoder, which from three convolution layers performs the upsampling, which is simply deconvolutional operations, the transformation of signals from smaller resolutions to larger resolutions. The result of this transformation is therefore a map of \textquotedblleft perceptions\textquotedblright, the segmented image (Figure \ref{encoder-decoder-arch}). 
%A primeira, s�o realizadas sucessivos processos convolutivos na imagem de entrada. Para cada subregi�o, a camada realiza um conjunto de opera��es matem�ticas para produzir um simples valor no mapa de fei��es, geralmente, fun��es de ativa��o do tipo ReLU (do ingl�s, \emph{Rectified Linear Unit}) s�o aplicadas � sa�da para introduzir n�o-linearidades ao modelo. As camadas de \emph{pooling} s�o dispostas sempre ao fim das camadas convolucionais. Seu papel � reduzir a dimensionalidade (\emph{downsample}) do mapa de fei��es, i.e. reduzir a imagem de forma a diminuir o tempo de processamento. A regra b�sica de redu��o � a utiliza��o do \emph{max-pooling}, que extrai sub-regi�es do mapa de fei��es (e.g em janelas de 2$\times$2 \emph{pixels}) e mant�m somente seu valor m�ximo, descartando todos os outros. Por fim, a camada densa realiza a classifica��o das fei��es extra�das pelas camadas convolucionais e de \emph{pooling}, tal que cada n� na camada densa � conectado aos n�s na camada anterior (Figura \ref{encoder-decoder-arch}).

%No primeiro, a imagem de entrada � submetida a in�meros processos de convolu��o e \emph{pooling}. A cada \emph{pooling}, o processo de \emph{downsampling} ocorre, em que a resolu��o da imagem � reduzida de forma a ser a entrada para as camadas convolucionais posteriores. Consequentemente, o banco de filtros aprendem diferentes padr�es em diferentes escala, orienta��o e localiza��o (fun��es semelhantes aos do cristalino e retina biol�gicos). Os sinais em dimens�es reduzidas s�o ent�o transmitidos ao \emph{decoder} (cortex visual), que a partir de tr�s camadas convolutivas realiza o \emph{upsampling}, e.g. a transforma��o dos sinais de resolu��es menores para resolu��es maiores. O resultado dessa transforma��o �, portanto, um mapa de \textquotedblleft percep��es\textquotedblright das classes (imagem segmentada).
\begin{figure}[!htp]
    \centering     
    \caption{Autoencoder architecture to segment images.}
    \vspace{6mm}
    \includegraphics[width=1\textwidth]{\figspath/cnn/flow/cnn-flow.pdf}
    \legenda{}
    \vspace{2mm}
    \FONTE{Adapted from \citeonline{segnet}.}      
    \label{encoder-decoder-arch}
\end{figure} 

%Analogously, CNNs have two basic processing mechanisms, called encoders and decoders. The first, responsible for processing the signals (eyes), so that the image is passed through multiple convolution and pooling layers, and different convolution filters learn to detect different patterns in images (lens, focus and retina). In this process, the features of "interest" at the moment of seeing something are highlighted by the eyes, so that at the moment when viewing something only part of the scene is actually captured. Then these signals are transmitted to the decoder (visual cortex), which from three convoluted layers transposed to perform the up-sampling procedure, which in short, turn the signals into smaller resolutions for larger resolutions. The result of this transformation is a map of "perceptions" (segmented image) of the respective input.

Another feature of these systems and also idealized observing human behavior is the way humans perceive space and objects in their environment. In a lapse of time, innumerable processes are triggered instantly in the brain. In one of these, it is possible to note the human capacity to identify distinct features of all the rest. It is like the human visual system \textquotedblleft would not worry\textquotedblright~at capturing all the information on that environment, but rather, only those that might be interesting. Permanently, the human eyes scan the environment in very fast movements, process called Saccade \cite{ko2016, pirkl2016}. These very fast movements, allow the eyes to capture fractions of images step-by-step. In the Figure \ref{feature-imagem-trajeto-retina}, for instance, the features highlighted by the CNN and the path of the iris when exposing a random image is accurately close.
\begin{figure}[!htp]
 \centering     
 \subfigure[]{\label{feature-a}\includegraphics[width=0.48\textwidth]{\figspath/cnn/feature/feature-a.png}
 \includegraphics[width=0.48\textwidth]{\figspath/cnn/feature/feature-b.png}}\\
 \subfigure[]{\label{feature-b}\includegraphics[width=0.475\textwidth]{\figspath/eyetracking/eyetracking.png}
 \includegraphics[width=0.50\textwidth]{\figspath/eyetracking/eye-tracking-2.png}}
 \caption{Analogy to the computational neural system and biological. (a) Output of a Recurrent Neural Network (RNR), a special type of \emph{deep} network. The network is trained to translate high levels of representations into texts. In the figure, the network's ability to focus its attention on specific sections of the image; (b) Eye-tracking device and the capture (gray points) of eyeball movements - Saccade effect.}
 \label{feature-imagem-trajeto-retina}
 \FONTE{Adapted from \citeonline{lecun2015, pirkl2016, dalmaijer2014}.}             
\end{figure} 

The semantic segmentation of images by the use of neural networks has gained adherents in a wide range of applications, ranging from the recognition of objects of different natures \cite{krizhevsky2012}, as well as the understanding of the environment, in which the analysis is also applied on the relation between the objects themselves. Thus, certain constraints could be easily applied, for example, a pedestrian with the object street or automobiles in an application involving autonomous driving \cite{segnet, teichmann2016, lettry2017}. As emphasized by \citeonline{zhu2017}, Remote Sensing data represent a new challenge for DL. Their imaging nature can often be differentiated. For example, optical, laser and radar data provide different representations, which would require analysis that has not yet been so explored with the use of DL. %Remote Sensing data is georeferenced, images from internet are usually free of location metadata. In this case, DL could correlate specific facades from internet to a location in a RS data, supporting Security agencies. %Data from SR carries with it an enormous amount of information, physical and biochemical measurements that, in integration with DL, could improve prediction algorithms or emulators, as well as numerous other applications.
%Inspirando-se na segmenta��o sem�ntica e classifica��o, os ent�o modelos neurais \textquotedblleft inteligentes\textquotedblright passaram, ent�o, a ser utilizados em in�meras aplica��es em SR \cite{zhu2017}, como na an�lise imagens �pticas orbitais \cite{castelluccio2015, marmanis2016}, de radar \cite{chen2014sar, chen2016, sun2012}, hiperespectrais \cite{chen2014, romero2014} e na reconstru��o 3D urbana \cite{hane2013, blaha2016, blaha2016towards}. Conforme destacado por \citeonline{zhu2017}, dados de SR representam um novo desafio para DL. Sua natureza de imageamento pode frequentemente diferenciar-se. Por exemplo, dados �pticos, laser e radar fornecem representa��es diferentes da cena, o que exigiria an�lises na qual ainda � pouco explorada com o uso de DL. Dados de SR s�o georeferenciados, por�m, imagens da \emph{internet} sem quaisquer metadados de localiza��o, por exemplo, poderiam ser utilizadas para identifica��o autom�tica de endere�os em aux�lio � Seguran�a P�blica. Dados de SR carregam consigo uma enorme quantidade de informa��es, medidas f�sicas e bioqu�micas que, em integra��o com DL, poderiam melhorar algoritmos de predi��o ou de emuladores, al�m de in�meras outras aplica��es. 

The DL as an automatic extractor of urban features is a scientific question of great interest to the community and also covers a limited number of works. The efforts so far, show that there is progress in identifying facade features in specific architectural layouts, with well-defined, symmetrical, and accessible modeling facades. The use of benchmark datasets, such as the ones used in this study, is common and provide a wide overview about the extraction algorithms available today. 

In \citeonline{liu2017deepfacade}, for example, DL is used for the identification of facade features on two online datasets: eTRIMS and ECP datasets, presenting similar results to those shown in this study. The VarCity project \cite{hayko2014}\footnote{Available at https://varcity.ethz.ch/index.html. Accessed \today.}, provides accurate perspective on 3D cities and image-based reconstruction. The research involves not only studies in \textquotedblleft how\textquotedblright~to reconstruct, but how these semantic models could automatically assist to the daily life events (e.g. traffic, pedestrians, vehicles, green areas, among others).
%Tema ainda emergente, o uso de DL na extra��o autom�tica de fei��es urbanas voltada �s �reas Cartogr�ficas � quest�o cient�fica de grande interesse da comunidade e, ainda, abrange um n�mero reduzido de trabalhos. Os esfor�os realizados at� o momento, mostram que h� progresso sobre a identifica��o de fei��es de fachadas em tipos arquitet�nicos espec�ficos\cite{vangool2013} (ver Se��o \ref{architectural-style}), com fachadas bem definidas, sim�tricas e de modelagem acess�vel (i.e. devido aos padr�es de repeti��o entre as edifica��es). O uso de conjunto para \emph{benchmarks} � comum e fornece um panorama completo quanto aos algoritmos de extra��o dispon�veis hoje. Em \citeonline{liu2017deepfacade}, por exemplo, DL � utilizada para a identifica��o de fei��es em fachadas sobre dois conjuntos de dados eTRIMS e ECP (ver Tabela \ref{datasets}). 

    \chapter{MATERIAL AND METHOD}\label{chapter3}
\section{The current brazilian 3D urban maps}\label{3d-urban-brazil}
The content presented in this section, represents a complementary study to the Section \ref{map-brazil}, driven by the need to understand the current state of Brazilian 3D urban mapping. The goal in this respective study was to understand more the local mapping infrastructure among the Brazilian capitals, the availability of resources for their urban planning, and in how an accurate 3D urban model would help in the management of the city. 

For this task, a poll was elaborated on the use of 3D maps for the strategic planning of municipalities and sent to the secretaries of each Brazilian capital. The capitals were used as reference for the research, therefore, when reporting as being a \textquotedblleft non-user\textquotedblright~of any 3D information, it was considered that all cities referring to the respective state were also considered as non-users. The questions of the poll were:
\begin{enumerate}[topsep=0pt,itemsep=-1ex,partopsep=1ex,parsep=1ex]
    \item Does the infrastructure/planning department have urban 3D city maps?
    \item If so, how did this benefit the management and urban planning service?
    \item If yes, what applications are used today?
    \item If not, how is urban planning currently done?
    \item How could 3D urban maps contribute?
\end{enumerate}

Among the 26 Brazilian capitals and 1 federal district, only 8 of them answered the poll (Table \ref{poll-answers}). The details of each answer can be seen in Table \ref{poll-answers2}, in Appendix \ref{apendiceC}.
\begin{table*}[!htb]
    \renewcommand{\arraystretch}{1.4}
    \caption{Poll answers from each Brazilian Capital regarding the use of 3D maps on urban planning.}
    \scriptsize \centering
    \rowcolors{2}{white}{blue!15}
    \begin{tabular}{L{2.1cm}C{1.7cm}C{1.9cm}C{1.8cm}C{1.0cm}C{1cm}C{1cm}C{1.2cm}}
        \toprule
        \textbf{City} & \textbf{Question 1} & \textbf{Released or Planned?} & \textbf{Products} & \textbf{Sensor}  & \textbf{Platform}  & \textbf{Scale} & \textbf{Coverage}\\ 
        \toprule                
        Fortaleza & $\times$ & Planned & -- & Laser & \faPlane & Large & --\\
        Vit�ria & $\times$ & Planned & -- & Laser & \faPlane & Large & --\\
        Porto Alegre & \checkmark & Planned & DSM, DTM & Laser & \faPlane & Large & Full\\
        S�o Paulo & \checkmark & Released & DSM, DTM & Laser & \faPlane & Large & Full\\
        Belo Horizonte & \checkmark & Released & -- & Laser & \faPlane & Large & Medium\\
        Rio de Janeiro & \checkmark & Released & -- & Laser & \faPlane & Large & Medium\\
        Curitiba & \checkmark & Released & -- & Laser & \faPlane & Large & Medium\\
        Recife & \checkmark & Released & -- & Laser & \faPlane & Large & Full\\
        \bottomrule
    \end{tabular}      
    \label{poll-answers}        
    \vspace{0.1cm}
    \begin{flushleft}
        \scriptsize{(\faPlane) onboard of airplane}\\      
    \end{flushleft}
    \FONTE{Author's production.}
\end{table}
        
Even though only a few secretaries have replied, the answers from the largest capitals have been obtained. Despite the different responses, it is believed that the vast majority still adopt the same technologies for 3D surveys: long-range aerial surveys through LiDAR, to roughly observe the urban structuring. Still, the datasets acquired by some of these cities are out of operation for urban planning purposes. The names and emails of those responsible for the answers can be seen in Appendix \ref{apendiceC}.

\section{Study areas and datasets}\label{study-area}  
To train supervised DL methods with a good generalization, a large dataset is always required. This would contribute to both fine-tuning models and training small networks from scratch \cite{zhu2017}. In order to vary the building architectural styles, the experiments in this work cover different areas. In recent years, several datasets have been shared in order to contribute in new studies, such as to train deep neural networks, for benchmarks, or simply both.

In Table \ref{datasets}, seven different datasets are listed. The first 6 rows are online shared datasets, mainly used for evaluation and to perform benchmarks over different extraction models. They are then used in this study as diversified inputs, since each of them presents different facade characteristics. The last row is a dataset obtained exclusively for this work and used as test images. Eight semantic classes were defined: roof, wall, window, balcony, door, shop, and finally two more, but unrelated to the facade: sky and background. Some of the datasets listed did not provide all eight classes and, in some cases, their annotations had to be adapted.
\begin{table}[!htp]
    \renewcommand{\arraystretch}{1.4}
    \caption{Datasets for facade analysis and benchmarks. RueMonge2014 and Graz, ETH Z�rich; CMP, Center for Machine Perception; eTRIMS, University of Bonn; ECP, Ecole Centrale Paris; SJC, S�o Jos� dos Campos.}
    \tiny \centering		
    \begin{tabular}{L{1.4cm}L{1.2cm}L{1.4cm}C{0.8cm}C{1cm}C{0.9cm}C{1.2cm}L{3.8cm}}
        \toprule
        \textbf{Name} & \textbf{Location}  & \textbf{Architecture} & \textbf{Images} & \textbf{Labels} & \textbf{Rectified} & \textbf{PC Generation} & \textbf{Reference} \\ 
        \toprule
        RueMonge2014 & France & \emph{Hausmaniann} & 428 & 219 & $\times$ & \checkmark & \cite{hayko2014} \\      
        CMP & Multiple & Multiple & 378 & 378 & \checkmark & $\times$ & \cite{tylevcek2013}\\
        eTRIMS & Multiple & Multiple & 60 & 60 & $\times$ & $\times$ & \cite{korc2009}\\
        ENPC & France & \emph{Hausmaniann} & 79 & 79 & \checkmark & $\times$ & \cite{gadde2017}\\
        ECP & France & \emph{Hausmaniann} & 104 & 104 & \checkmark & $\times$ & \cite{teboul2010}\\      
        Graz & Austria & Classicism, \emph{Biedermeier}, Historicism, Art Nouveau & 50 & 50 & \checkmark & $\times$ & \cite{hayko2012}\\ \hline     
        SJC & Brazil & Multiple & 175 & - & $\times$ & \checkmark & - \\
        \bottomrule
    \end{tabular}
    \label{datasets}
    \FONTE{Author's production.}
\end{table}

\textbf{RueMonge2014}: The RueMonge2014 dataset was acquired to provide a benchmark for 2D and 3D facade segmentation, and inverse procedural modeling. It consists of 428 high resolution images, with street-side view (overlapped) of the facade, with Haussmanian architecture, a street in Paris, Rue Monge. Together with the 428 images, a set of 219 annotated images with seven semantic classes are also provided. Due to the geometry of acquisition, the dataset offers the possibility to generate a 3D reconstruction of the entire street scene. Into its evaluation framework, three tasks are proposed: (i) image segmentation; (ii) mesh labeling; and (iii) point cloud labeling. 

\textbf{Center for Machine Perception (CMP)}: The CMP consists of 378 rectified facade images of multiple architectural styles. Here, the annotated images have 12 semantic classes, among them, some facade features such as pillars, decoration, and window-doors were considered as being part of the wall (for pillars and decoration) and window (window-doors). Then, we have adapted the CMP dataset by unifying its classes and their respective colors.

\textbf{eTRIMS}: The facades in this set do not have a specific architecture style and sequence, as well as in the previous dataset. The eTRIMS provides 60 images, with two sets of annotated images, one with 4 semantic classes (wall, sky, pavement, and vegetation), and another with 8 (window, wall, door, sky, pavement, vegetation, car, and road). For our project, we chose the last, but adapted it to window, wall, and door features only. The other classes were considered as background.

\textbf{ENPC Art Deco}: The ENPC dataset provides 79 rectified and cropped facades in Hausmaniann style. The annotations, however, are shared not in image format, but in text, which also had to be adapted to the 7 classes and colors defined in this work. 

\textbf{Ecole Centrale Paris (ECP)}: Just like RueMonge2014, the 104 facade images provided by ECP are in Hausmaniann style, but the images are rectified, with cropped facades. In some cases, the classes windows, roof and walls were not perfectly delineated, which may be considered noise by supervised neural models. The same issue can also be found in ENPC dataset. Even though we noticed the problem, no adaptation was performed.

\textbf{Graz}: The Graz dataset consists of multiple architectural styles, selected from the streets in Graz (Austria), rectified, with the same 7 semantic classes defined in RueMonge2014.

\textbf{S�o Jos� dos Campos (SJC)}: The SJC dataset consists of buildings at a residential area in S�o Jos� dos Campos, S�o Paulo, Brazil. Like most of the country, the architectural style throughout this city is not unique, often diverging between free-form and modern styles. This set consists of 175 sequential images, overlapped, and taken at the same moment.

\section{Method}
The complete methodology of this case study, as shown in Figure \ref{methodology}, consists of three stages: A supervised CNN model for semantic segmentation (blue); Scene geometry acquirement (3D reconstruction) through SfM and MVS pipeline (red); Post-processing procedures (yellow); and 3D labeling through ray-tracing analysis (white). The boxes in gray represent the products, delivered in different steps of the workflow. The following sections, therefore, are presented according to this sequence.
\begin{figure}[H]
    \centering	   
    \caption{3D facade model: Facade feature extraction and reconstruction workflow.}
    \vspace{6mm}
    \includegraphics[width=1.0\textwidth]{\figspath/methodology/new-methodology-inkscape-en.pdf}    
    \vspace{2mm}
	\legenda{}
    \label{methodology}
    \FONTE{Author's production.}
\end{figure}

\subsection{Facade feature detection} 

\subsubsection{Training dataset}\label{training-set}
Each of the 6 datasets has been divided in three different subsets: training, validation, and tests. 80\% of the annotated images were used for training, and 20\% for validation. Only RueMonge2014 had a non-annotated set of images (209), which was used as test. Due to the small number of training samples, no set of test images was used for the other group of data. Instead, a new acquisition with similar geometry as RueMonge2014 was performed in the city of S�o Jos� dos Campos (SJC), S�o Paulo, Brazil. The images will be used only for test, where each of the mentioned datasets are used for training.
\begin{figure}[!htp]
    \centering	   
    \caption{Example of input images for training. (a) Pair of non-rectified images (original and annotated) from RueMonge2014 dataset. (b) Pair of rectified images (original and annotated) from ECP dataset.}
    \vspace{6mm}
    \includegraphics[width=1\textwidth]{\dropbox/phd/pics/legend/legend.png}
    \subfigure[]{\includegraphics[width=0.25\textwidth]{\data/data/facades-benchmark/ruemonge2014/dataset/IMG_5703.jpg}\hspace{0.1cm}
        \includegraphics[width=0.25\textwidth]{\data/data/facades-benchmark/ruemonge2014/annotation/IMG_5703.png}} \hspace{0.3cm}
    \subfigure[]{\includegraphics[width=0.212\textwidth]{\data/data/facades-benchmark/ecp/images/monge_13.jpg}\hspace{0.1cm}%sjc/images/sjc_55.jpg}\hspace{0.1cm}
        \includegraphics[width=0.212\textwidth]{\data/data/facades-benchmark/ecp/annotation/monge_13.png}} %sjc/annotation/sjc_55.png}}   
    \vspace{2mm}
	\legenda{}
    \label{inputs}
    \FONTE{Author's production.}
\end{figure}

\subsubsection{Neural model}\label{neural-model-description} 
The classic DL architectures used in visual data processing can be categorized in Autoencoders (AE) and CNN architectures \cite{zhu2017} (as discussed in Section \ref{deep-learning}). An AE is a neural network that is trained to reconstruct its own input as an output. It consists of three layers: input, hidden, and output. The hidden layer takes care of all operations behind this model. The weights are iteratively adjusted to become more and more sensitive to the input \cite{goodfellow2016}. 

The CNNs, on the other hand, take advantage of performing numerous convolution operations in the image domain, where a finite number of filters are repetitively applied in a downsampling image strategy, which allows the analysis of the scene at different orientations and scales. The convolution is one of the most important operations in signal and image processing, and can be applied in different domains: 1D, in speech processing \cite{swietojanski2014}, 2D, in image processing \cite{krizhevsky2012, audebert2017, long2015} or 3D, in video processing \cite{karpathy2014}.

Although it has been categorized by \citeonline{zhu2017} as being different neural architectures, AE and CNN can be at the same architecture, so called Deep Convolutional Encoder-Decoder, but it has only variations on the layers arrangement, not in the concept, then, considered here simply as CNN. 

The deep neural architecture adopted in this study consists of two main components: encoder and decoder. Most recent deep architectures for segmentation have identical encoder networks, i.e. VGG16 \cite{simonyan2014}, but differ on their decoder, training and inference \cite{segnet}. As the goal in this study is not to test variations of neural architectures, the encoder here corresponds to the same topological structure of the convolution layers of VGG16 \cite{simonyan2014}, initialized using pretrained VGG weights on ImageNet \cite{krizhevsky2012}, randomly initialized using a uniform distribution in the range ($-0.1,0.1$). 

The encoder is originally composed by 13 convolution layers, followed by their respective Pooling and Fully Connected (FC) layers, which in total gives VGG the name of 16. The FC layer, however, is replaced by a $1\times1$ convolution layer, which takes the output from the last pooling, and generates a low resolution segmentation \citeonline{teichmann2016}, with dimension of $1x1x4096$ (encoded input, see Figure \ref{cnn-lotte}). This makes the network smaller and easier to train \cite{segnet}.  

The decoder is seen as a component that interprets chaotic signals into something intelligible, similar to the human senses. For example, it would be like the equivalence between a noise (signal) and a person talking in a known language (interpreted signals by our brain), radiation (signal) and the perception of being under a garden with flowers and animals (environment perception, which is the brain interpretation of this same radiation), etc. The neural architecture presented by \cite{teichmann2016}, for instance, used 3 different decoders with 3 different tasks in a way that real-time application could be performed. Multiple decoders mimics the human multiple sense being performed simultaneously by the brain. Once the purpose here is only the visual sense, the use of multiple decoders is not interesting due to useless computational demand, and unnecessary processing.

Using the previous analogy, the input signals to the decoder is the final matrix from the encoder, which performs the upsampling operation resulting in a pixelwise prediction. The operations performed by this component is composed by 3 deconvolutional layers, so the predictions result in upsampled in the same size as the input image. To better understand the fully path, in the Figure \ref{cnn-lotte} is illustrated the details of the final CNN used in this study, as just described.
% decoder with a Fully Convolutional Network (FCN) architecture \cite{long2015}. 
\begin{figure}[!htp]
    \centering	   
    \caption{Convolutional Neural Network details used in this study. Encoder-Decoder convolutional architecture.}
    \vspace{6mm}
    \includegraphics[width=1\textwidth]{\dropbox/phd/pics/cnn/cnn-lotte.pdf}
    \vspace{2mm}
	\legenda{}
    \label{cnn-lotte}
    \FONTE{Author's production.}
\end{figure}

The convolution layers sequentially perform the linear operations regarding a certain kernel size and stride (the number of pixel jumping over the kernel slide). Each convolution layer has also a pre-determined number of filter, their functionality is to learn local features, such as horizontal lines, vertical, curves, so on. As soon as the convolution layers are going further, these filter become more and more, since the dimensionality get decreased. In green, the pooling layers take the convolution result and downsampled it to the next convolution layers. In this transition, the dimension get half of its original size. 

When the encoded input finally encounter the softmax layer (resulting in the probabilities performed by the cross-entropy function), 3 transposed layers decode the signals (deconvolution). In other words, the decoder performs the upsampling based on pooling indexes (from the encoder polling), finally getting the predicted map. The dropout layer is a technique for reducing overfitting, which avoids some redundant or complex adaptations on training data. 

%the weights connected to each filter
% VGG16
% Convolution using 64 filters
% Convolution using 64 filters + Max pooling
% Convolution using 128 filters
% Convolution using 128 filters + Max pooling
% Convolution using 256 filters
% Convolution using 256 filters
% Convolution using 256 filters + Max pooling
% Convolution using 512 filters
% Convolution using 512 filters
% Convolution using 512 filters + Max pooling
% Convolution using 512 filters
% Convolution using 512 filters
% Convolution using 512 filters + Max pooling
% Fully connected with 4096 nodes
% Fully connected with 4096 nodes
% Output layer with Softmax activation with 1000 nodes

	  
% Diferen�a entre Fully Connected e Fully Convolutional networks: https://www.quora.com/How-is-Fully-Convolutional-Network-FCN-different-from-the-original-Convolutional-Neural-Network-CNN 

% Procurar refer�ncia
%Fully convolutional indicates that the neural network is composed of convolutional layers without any fully-connected layers or MLP usually found at the end of the network. A CNN with fully connected layers is just as end-to-end learnable as a fully convolutional one. The main difference is that the fully convolutional net is learning filters every where. Even the decision-making layers at the end of the network are filters. A fully convolutional net tries to learn representations and make decisions based on local spatial input. Appending a fully connected layer enables the network to learn something using global information where the spatial arrangement of the input falls away and need not apply.

% TensorFlow > https://www.tensorflow.org/tutorials/image_recognition

% https://www.youtube.com/watch?v=LxfUGhug-iQ

% boas explica��es
% https://wildcse.blogspot.com/2018/03/convolution-neural-network-intro.html

\subsection{Multi-View surface reconstruction}  
The input is a set of images which are initially fed to standard SfM/MVS algorithms to produce a 3D model. 
To produce the 3D model using the standard SfM/MVS algorithms, the set of optical images to be presented must respect some specifications. For example, not all datasets listed in Table \ref{datasets} have properties that could allow the application of SfM/MVS. Random, rectified, and cropped images are not overlapped or, at least, were not taken in the same moment. Only with RueMonge2014 it was possible to run this experiment. A case where random images are taken at different time was proposed by \citeonline{snavely2009}, but not used in this study. 

Every measure made by any photogrammetric technique would require the camera calibration in order to obtain high accurate metric information, such as depth and dimensional measurements from 2D domain \cite{he2006}. Camera calibration involves the estimation of either internal (intrinsic) and external (extrinsic) parameters to a certain camera. These determine the relation between the scene and the instrument itself. The first, consists in the parameters particular to the camera, where is determined how the image coordinate of a point is derived, given the spatial position of the point with respect to the instrument (camera positioning). The external, on the other hand, determines the geometrical relation between the camera and the scene, or even different cameras \cite{photoscan}.

The SfM, however, is a fully automatic and iteratively way to solve this need. During the process, the intrinsic and extrinsic parameters are estimated and the outcome point cloud is then densified by the MVS technique. In this study, the estimated internal and external parameters are also used to ray-trace the segmented image toward the mesh (see Section \ref{ray-tracing-section}), where they are preserved from the reconstruct operation to ray-tracing (Figure \ref{ray-tracing-1}). The internal parameters are then given by the distance between the lens and image sensor, called focal length ($f$), the principal point offset ($c_x, c_y$), the skew coefficient ($b$), and the pixel size ($s_x, s_y$). 

The lens distortions parameters not considered
\begin{figure}[!htp]
    \centering	   
    \caption{Facade geometry obtained from photos. (a) Street-side images of facades and its reconstructed surface after the SfM/MVS technique. (b) The camera parameters are kept and the photos are replaced by the CNN facade features predictions.}
    \vspace{6mm}
    \subfigure[]{\label{ray-tracing-1a}\includegraphics[width=0.48\textwidth]{\dropbox/phd/pics/ray-tracing/ray-tracing-thesis-01.png}}
    \subfigure[]{\label{ray-tracing-1b}\includegraphics[width=0.48\textwidth]{\dropbox/phd/pics/ray-tracing/ray-tracing-thesis-02.png}}
    \vspace{2mm}
	\legenda{}
    \label{ray-tracing-1}
    \FONTE{Author's production.}
\end{figure}

The external are given by two parameters, $R$ and $T$. They are the coordinate system transformations from 3D world coordinates to 3D local camera coordinates system, where $R = (R_X(\omega), R_Y(\phi), R_Z(\kappa))$ is the rotation matrix, and $T = (T_X, T_Y, T_Z)$, the translation matrix, consisting in the local camera coordinate system with origin at the camera projection center. The $Z$ axis points towards the viewing direction, $X$ axis to the right, and $Y$ axis points down. The image coordinate system has origin at the top left image pixel, with the center of the top left pixel having coordinates ($0.5, 0.5$), $u_0$ and $v_0$, respectively (bottom-right in Figure \ref{ray-tracing-1b}).

Summarizing, the camera can be described by a matrix notation, which is the cumulative effect ($\varPi$) of all parameters \cite{hartley2004}:
\begin{equation}
    \varPi = K~P~[R~T] 
    \begin{bmatrix}
    X \\
    Y \\
    Z \\
    1
   \end{bmatrix}~,
\end{equation}
where $K$ corresponds to the intrinsic parameters:
\begin{equation}
K=
  \begin{bmatrix}
    f_x & b & u_0\\
    0 & f_y & v_0\\
    0 & 0 & 1
   \end{bmatrix}~,
\end{equation} 
which can be decomposed on the three principal 2D units: principal point offset, focal length, and skew:
\begin{equation}
K=
  \begin{bmatrix}
    1 & 0 & u_0\\
    0 & 1 & v_0\\
    0 & 0 & 1
   \end{bmatrix}
   \times
   \begin{bmatrix}
    f_x & 0 & 0\\
    0 & f_y & 0\\
    0 & 0 & 1
   \end{bmatrix}
   \times
   \begin{bmatrix}
    1 & \frac{b}{f} & u_0\\
    0 & f_y & v_0\\
    0 & 0 & 1
   \end{bmatrix}~,
\end{equation}
where $f_x$ and $f_y$ correspond to $f$, with same value. $P$ denote a $4\times3$ matrix, which corresponds to the projection:
\begin{equation}
P=
  \begin{bmatrix}
    1 & 0 & 0 & 0\\
    0 & 1 & 0 & 0\\
    0 & 0 & 1 & 0
   \end{bmatrix}~.
\end{equation}
The rotation and translation matrices are then given by:
\begin{equation}
\small
R=
    \begin{bmatrix}
    cos\kappa~cos\phi & -sin\kappa~cos\phi & sin\phi & 0\\
    cos\kappa~sin\omega+sin\kappa~cos\omega & cos\kappa~cos\omega-sin\kappa~sin\omega~sin\phi & -sin\omega~cos\phi & 0\\
    sin\kappa~sin\omega-cos\kappa~cos\omega~sin\phi & sin\kappa~cos\omega~sin\phi + cos\kappa~sin\omega & cos\omega~cos\phi & 0\\
    0 & 0 & 0 & 1
    \end{bmatrix}~,
\end{equation}
\normalsize
and:
\begin{equation}
T= 
\begin{bmatrix}
    1 & 0 & 0 & T_X\\
    0 & 1 & 0 & T_Y\\
    0 & 0 & 1 & T_Z\\
    0 & 0 & 0 & 1    
   \end{bmatrix}~,
\end{equation}
and the camera position $C$ given by:
\begin{align}
 C = -R^{-1}T~.
\end{align}

All the procedures in this stage were performed through Agisoft\texttrademark~PhotoScan\textregistered. Very similar results could be reached by the use of VisualSfM \cite{wu2011vsfm} or COLMAP \cite{schoenberger2016sfm}, for instance. The respective softwares have more flexible licenses and preserve most of the guarantees proposed by Photoscan. However, it still needs some expertise to install and to use. PhotoScan incorporates an improved SIFT algorithm for Feature Matching across the photos; for camera intrinsic and extrinsic orientation parameters, the software uses a greedy algorithm to find approximate camera locations and refines them later using BA.

In addition to the camera parameters estimation, operations involving the use of SfM require good photo-taking practices. One of these is the texture conditions of the targets. Targets whose surface is too homogeneous or with specular properties, will harm the method. Any instrument configuration (e.g. zoom, lens or distortions), once configured, must be preserved until the end of the acquisition. In case of facades, the motion between one photo and another must have orientation as constant as possible and always perpendicular to the target. Therefore, these condictions were considered during the acquisition of SJC images.

The geometric accuracy in this study corresponds to the proximity between the reconstructed model and the point cloud, not necessarily to the positional part. In this case, it was assumed that the point cloud had previously proven positional accuracy. It was beyond the scope of this study to analyze adjustments or positioning issues. Then, no geometrical accuracy has been reported due to the fact that no 3D references were modeled to compare against the 3D models reconstructed. 

\subsection{3D labeling by ray-tracing analysis}\label{ray-tracing-section}
At this methodological point, two products were archived: the classified facade features (2D image segmentation) and their respective geometry (mesh). The idea here is to merge each feature to its respective geometry, and that can be done by analysing the ray-tracing of each image respect to their camera projection (estimated during the SfM pipeline) onto the mesh.   

Often used in computer graphics for rendering real-world scenarios, such as lighting and reflections, ray-tracing analysis mimics real physical processes that happen in nature. A energy source emits radiation at different frequencies of the electromagnetic spectrum. The small portion visible to human eyes, called the visible region, travels straight in wave forms and it is only intercepted when it encounters a surface in its trajectory. Such surface has specific physical, chemical and biological properties, responsible to define the radiation behavior under this specific structure. A surface whose absorption characteristics is high, the tendency is that this object has a dark appearance, as the radiation incident on its surface is absorbed, and a minimal portion is reflected to the eyes.

Each facade image, in essence, is the record of the reflection of electromagnetic waves in a tiny interval of time, captured by a sensor at a certain distance and orientation. Once the camera's projection parameters are known, the original images used for its estimation during SfM are replaced by the CNN predictions. Thus, the \textquotedblleft reverse\textquotedblright~ray-tracing process can then be performed.

The ray-tracing consists in the transformation from point coordinates in the local camera coordinate system to the pixel coordinates in the image. It is essential to consider in which context the photos were taken in order to use the right transformation from the center of projection toward the mesh. For example, if a fish-eye model transformation is used in camera frame model, the result would certainly not be the right one. 

Semi-professional or even professional cameras are usually used to make these campaigns. Some cameras provide several categories of parametric lens distortions, some of them, by default, also include non-distorted configurations (RAW images), which does not apply any transformation to the image. Therefore, both RueMonge2014 and SJC are images obtained in the camera frame model, which pixel coordinate ($u, v$) from the 3D point projection are given by the following transformation. Let:
\begin{equation} 
\begin{pmatrix}
    x\\
    y\\    
   \end{pmatrix}
   =
   \begin{pmatrix}
    \frac{X}{Z}\\
    \frac{Y}{Z}\\    
   \end{pmatrix}
\end{equation}
be the homogeneous point $r$ the squared 2D radius from the optical center:
\begin{align}
 r = \sqrt{x^2 + y^2}~,
\end{align}
then, all the coefficients of distortions have to be considered. Then, the pixel coordinate ($u, v$) of the 3D point projection with distortion model is given by:
\begin{equation}
 \begin{pmatrix}
    u\\
    v\\    
   \end{pmatrix}
   =
   \begin{pmatrix}
    (w * 0.5) + c_x + \Theta_xf_x + \Theta_xb_1 + \Theta_yb_2\\
    (h * 0.5) + c_y + \Theta_yf_y 
   \end{pmatrix}~,
\end{equation}
where $b_1$ and $b_2$ are the skew coefficients, $w$ and $h$ the image width and height. The parameter $\Theta$ denotes the radial and tangential distortions: 
\begin{equation}
\scriptsize
 \begin{pmatrix}
    \Theta_x\\
    \Theta_y\\    
   \end{pmatrix}
   =
   \begin{pmatrix}
    x * (1 + \xi_1r^2 + \xi_2r^4 + \xi_3r^6 + \xi_4r^8) + (1 + \zeta_3r^2 + \zeta_4r^4) * (\zeta_1 (r^2 + 2x^2) + 2\zeta_2xy)\\
    y * (1 + \xi_1r^2 + \xi_2r^4 + \xi_3r^6 + \xi_4r^8) + (1 + \zeta_3r^2 + \zeta_4r^4) * (\zeta_2 (r^2 + 2y^2) + 2\zeta_1xy)
   \end{pmatrix}~,
\end{equation}
\normalsize
where $\xi_1$, $\xi_2$, $\xi_3$, and $\xi_4$ are the radial and, $\zeta_1$, $\zeta_2$, $\zeta_3$, and $\zeta_4$, the tangential distortion coefficients. As the images used have no radial and tangential distortions, they are null and turn $\Theta$ as being the own axis $x$ and $y$.

% As described in previous section, each image has its position $C$ and orientation $R$, consisting respectively of the camera projection center located at the origin of the 3D coordinate system $C = (C_X, C_Y, C_Z)$, and rotation matrix $R = (R_X, R_Y, R_Z)$. The origin of a certain ray, then, it is all camera projection center $C$, with its direction given by $R$. Considering no optical blur, distortion, or defocus, a point $C_i$ is mapped to a point in the image plane $(u, v)$ by:  
% \begin{align}
%     p_i(u,v,f) = \frac{f}{Z}\begin{pmatrix}X\\Y\\Z\end{pmatrix}~,
% \end{align}
% where $f$ is the known focal distance (left in Figure \ref{ray-tracing}), $u=f\frac{X}{Z}$, and $v=f\frac{Y}{Z}$. 
% 
Now, knowing the pixel class from the incident ray (Figure \ref{ray-tracing-2a}), the mesh triangle is finally labeled as such. It is evident, therefore, that the rays trajectory from the images can intersect one another. Because of that, different rays can reach an identical point on the mesh, what in fact creates questions such as \textquotedblleft which class should be assigned to each individual mesh facet?\textquotedblright. \cite{hayko2014} proposed the Reducing View Redundancy (RVR) technique, where the number of overlapped images is reduced, which does not fit to our purpose, since the greater the number of overlaps, better the labeling (more classes to choose).

If more than one ray reach the same triangle (detail in Figure \ref{ray-tracing-2a}), then, it is labeled by the most frequent class from the $C_n$ rays (Figure \ref{ray-tracing-2b}).
\begin{figure}[!htp]
    \centering	   
    \caption{Ray-tracing analysis: This diagram shows the intersections of rays between the overlapped images, where the class assignment is made by choosing the most frequent class (mode) at the intersections. The colors on the right side of the picture, correspond to the pixels from different images, overlapping the same region on the mesh. To decide which class to assign, a simple mode (most frequent class) operation is used. The labeling legend can be seen in Figure \ref{inputs}.}
    \vspace{6mm}
    \subfigure[]{\label{ray-tracing-2a}\includegraphics[width=0.4\textwidth]{\dropbox/phd/pics/ray-tracing/ray-tracing-thesis-03.png}}
    \subfigure[]{\label{ray-tracing-2b}\includegraphics[width=0.54\textwidth]{\dropbox/phd/pics/ray-tracing/ray-tracing-thesis-04.png}}
    \vspace{2mm}
	\legenda{}
    \label{ray-tracing-2}
    \FONTE{Author's production.}
\end{figure}

%That could be solved though the application of a simple rule such as the mode (most frequent class) or even a smarter decision rule (e.g. choose the class where the segmented image's ray had the highest accuracy during the CNN inference), but we have noticed that a simple mode operation can provide sufficient labeling.   
The final result of 3D labeling by ray-tracing is then acquired (Figure \ref{ray-tracing-3}). Regions that have not been segmented (such as facades not perpendicular to acquisition - adjacent streets) are also treated as such during the process (dark regions in the figure - class background). Other regions, although they have been labeled as background, they were a consequence of the acquisition and angle of incidence of the rays.  
\begin{figure}[!htp]
    \centering	   
    \caption{Ray-tracing analysis: Details of the final ray-tracing result. (Center) The most common labels from 2D images are given to the geometric feature, which are not always correctly label due to the acquisition view point.}
    \vspace{6mm}
    \includegraphics[width=1\textwidth]{\dropbox/phd/pics/ray-tracing/ray-tracing-thesis-05.png}    
    \vspace{2mm}
	\legenda{}
    \label{ray-tracing-3}
    \FONTE{Author's production.}
\end{figure}

\section{Experiments}
Aligning with the hypotheses of this study, in this section we present details about the analysis strategy and the validation metrics adopted throughout the tests. The experiments were elaborated as much as possible on the conditions that were also imposed on the objectives.

\subsection{Strategy of analysis}
The experiments in this study were divided in 2D and 3D domains. To mitigate the influences of each dataset, or the influences of the model under each specific architectural style, it was split the 2D experiments in three different CNN trainings. First, all the six online datasets listed in Table \ref{datasets} were trained and inferred independently. Second, each knowledge reached from the respective datasets is used to test under SJC, which has a complete undefined architectural style. Third, all datasets were then put together and a new training was made under SJC (see Table \ref{training-explained}). The term \textquotedblleft independent\textquotedblright~means the inference was made using only one dataset for training or prediction. In 3D analysis, the experiments consist of permuting the point cloud density, allowing us to know how the number of points that affects the 3D labeling, and how many is actually necessary to acquire reliable geometry.
\begin{table}[!htp]
    \renewcommand{\arraystretch}{1.6}  
    \caption{Experiments performed in this study.}
    \scriptsize \centering		
    \rowcolors{2}{white}{gray!25}
    \begin{tabular}{L{0.5cm}L{1.1cm}L{2cm}L{1.6cm}L{7.5cm}}    
        \toprule      
        \textbf{\#} & \textbf{Domain}  & \textbf{Dataset} & \textbf{Inferences} & \textbf{Goal} \\ 
        \toprule
        1 & 2D & Independent & Independent & Evaluate the performance of the neural model according to each dataset\\      
        2 & 2D & Independent & SJC & Evaluate the performance of the neural model according to SJC dataset, where the inferences are made six times, using each dataset knowledge separately\\
        3 & 2D & All-together & SJC & Evaluate the performance of the neural model according to the SJC dataset, where only one inference is made, using all-together knowledge\\
        4 & 3D & RueMonge2014 & - & Evaluate how accurate the 3D labeling is according to the point cloud density (sparse and dense), under a known dataset\\      
        5 & 3D & SJC & - & Evaluate how accurate the 3D labeling is according to the point cloud density (sparse and dense), under an unknown dataset\\ 
        \bottomrule
    \end{tabular}
    \label{training-explained}
    \FONTE{Author's production.}
\end{table}

\subsection{Evaluation}\label{evaluation-section}
Only two validation metrics were adopted to measure the quality of the predictions, accuracy and F1-score. In addition, this section also explains the objective function defined to the neural model, Cross-Entropy.

\subsubsection{Accuracy}
The accuracy is calculated according to a pixelwise analysis, in which the success and error rates are measured through values of True Positive (TP), True Negative (TN), False Positive (FP) and False Negative (FN), then, placed in a confusion matrix\footnote{Considering a matrix which ground truth is in $x$ and predicted in $y$, the TP is given by elements in diagonal, TN, sum of all elements except the diagonal, FP, the sum of elements in $x$ minus diagonal and, FN, the sum of elements in $y$ minus diagonal.} $M$ (more details in Annex \ref{annexB}). Finally, the accuracy is given by:
\begin{align}
 Accuracy &= \frac{\sum_{i=1, j=i}^n M_{ij}}{N}~,
\end{align}
where $n$, the number of classes, and $N$, the number of samples used. The numerator consists in the sum of all elements in diagonal.

\subsubsection{Objective function}
An optimization problem seeks to minimize a loss-function. The weight-loss consists of the error levels in which the neural network has according to an ideal of prediction that is given by the reference images. This ideal is an optimization problem which aims to be minimal, called then loss-function. In this case, the loss-function is given by the cross-entropy \cite{long2015}, a common and effective way to calculate the distance between multiple predicted and ground-truth classes, also known as multinominal logistic classification, given by:
\begin{align}
 Loss_{y'}(y) &= -\sum_{i}y_i'~\log~(y_i)~,
\end{align}
where $y_i$ is the predicted probability value for class $i$, and $y_i'$ is the ground-truth probability for that class (more about cross-entropy calculation and meaning can be found in Annex \ref{annexA}).

\subsubsection{F1-Score}
F1-Score expresses the harmonic mean of precision and recall. These are calculated values to understand how aligned the prediction is in relation to the reference object. F1-score is given by:
\begin{align}
 F1 &= 2 * \frac{precision * recall}{precision + recall}~, 
\end{align}
where $precision=TP/(TP+FP)$, and $recall=TP/(TP+FN)$. Then, both values reveal how good the segmentation was according to the correct object delineation (more details in Annex \ref{annexB}).

    \chapter{RESULTS AND DISCUSSION}\label{chapter4}

\section{Performance}
As a supervised methodology, the DL requires reference images\footnote{Referenced also as annotation, labels or ground-truth images.}. Which means the methodology is extensible for images of any kind, but it will always require their respective reference. On the other hand, the same neural model could fit to any other detection issue, for instance, in the segmentation of specific tree species in a vast forest image, as soon as a sufficient amount of training samples are presented.

All the source-code regarding DL procedures was prepared to support GPU processing. Unfortunately, the server used during all the experiments was not equipped with such technology, increasing training time significantly (Table \ref{dataset-params}).

\begin{table}[H]
    \renewcommand{\arraystretch}{1.4}
    \caption{Training attributes and performance. In bold, the ones which have reached the lowest performance. RueMonge2014 and SJC have larger dimensions and demanded a bit more time processing.}
    \scriptsize \centering
    \rowcolors{2}{white}{gray!25}
    \begin{tabular}{L{2.5cm}C{2.1cm}C{2.5cm}C{2.8cm}C{2.8cm}}
        \toprule
        \textbf{Dataset} & \textbf{Num. iterations} & \textbf{Resolution (pixels)} & \textbf{Training (hours)} & \textbf{Inference (secs per image)} \\ 
        \toprule
        RueMonge2014 & 50k & 800x1067 & \textbf{172.46} & 5.4\\      
        CMP & 50k & 550x1024 & 135.25 & 4.45 \\
        eTRIMS & 50k & 500x780 & 83.57 & 3.52 \\
        ENPC & 50k & 570x720 & 53.47 & 2.01 \\
        ECP & 50k & 400x640 & 38.32 & 3.13 \\      
        Graz & 50k & 450x370 & 29.27 & 2.05 \\ 
        SJC & - & 1037x691 & - & \textbf{6.2} \\ 
        \bottomrule
    \end{tabular}    
    \label{dataset-params}
    \FONTE{Author's production.}
\end{table}

The DL source-code was mainly developed under the Tensorflow\texttrademark~library\footnote{Available at https://www.tensorflow.org/. Accessed \today.} and adjusted to the problem together with other Python libraries. Except for the 3D tasks in Photoscan, the source-code are free available in a public platform, and can be easily extended. For training and inferences, it was used an Intel\textregistered~Xeon\textregistered~CPU E5-2630 v3 @ 2.40GHz. For SfM/MVS and 3D labeling, respect to RueMonge2014 and SJC datasets, it was used an Intel\textregistered~Core\texttrademark~i7-2600 CPU @ 3.40GHz. Both attended our expectations, but it is strongly recommended machines with GPU support or alternatives such as IaaS (Infrastructure as a Service).

\section{Experiments}
The experiments in this study were divided in 2D and 3D domains. To mitigate the influences of each dataset, or the influences of the model under each specific architectural style, it was split the 2D experiments in three different CNN trainings. First, all the six online datasets listed in Table \ref{datasets} were trained and inferred independently. Second, each knowledge reached from the respective datasets is used to test under SJC, which has a complete undefined architectural style. Third, all datasets were then put together and a new training was made under SJC (Table \ref{training-explained}). In 3D analysis, the experiments consists of permuting the density in the point cloud, allowing us to know how the number of points affects the 3D labeling, and how many is actually necessary to acquire reliable geometry.
\begin{table}[H]
    \renewcommand{\arraystretch}{1.6}  
    \caption{Experiments performed in this study. The term \textquotedblleft independent\textquotedblright~means the dataset or inference was made using only one dataset for training or prediction.}
    \scriptsize \centering		
    \rowcolors{2}{white}{gray!25}
    \begin{tabular}{L{0.5cm}L{1.1cm}L{2cm}L{1.6cm}L{7.5cm}}    
        \toprule      
        \textbf{\#} & \textbf{Domain}  & \textbf{Dataset} & \textbf{Inferences} & \textbf{Goal} \\ 
        \toprule
        1 & 2D & Independent & Independent & Evaluate the performance of the neural model according to each dataset\\      
        2 & 2D & Independent & SJC & Evaluate the performance of the neural model according to SJC dataset, where the inferences are made six times, using each dataset knowledge separately\\
        3 & 2D & All-together & SJC & Evaluate the performance of the neural model according to the SJC dataset, where only one inference is made, using all-together knowledge\\
        4 & 3D & RueMonge2014 & - & Evaluate how accurate the 3D labeling is according to the point cloud density, under a known dataset\\      
        5 & 3D & SJC & - & Evaluate how accurate the 3D labeling is according to the point cloud density (sparse and dense), under a unknown dataset\\ 
        \bottomrule
    \end{tabular}
    \label{training-explained}
    \FONTE{Author's production.}
\end{table}

\section{Image segmentation}
As mentioned in Section \ref{training-set}, 20\% of annotated images from each dataset were used to evaluate the model. The set consists of pairs of original and ground-truth images, which are not used during the training. The experiments carried out in this study were done individually -- where, the quality of the segmentation through the use of CNN for each dataset (following the sequence according to Table \ref{datasets}) was firstly discussed, followed by the impressions on the detection of objects, and in which situations it might have fail or still need attention. Then, the following section is presented an analysis of the geometry extraction and the quality of the 3D labeled model.

\subsection{CNN training and performance}
In the Figure \ref{training-result} is shown the neural network training results. Each line represent an online dataset, conducted in individual training processes, with 50 thousand (50k) iterations each. Each graph's row represent the metric used to analyze the CNN performance: accuracy, weight-loss and cross-entropy. The accuracy allows to measure how good the segmentation is according to the training progress. While the weight-loss is the CNN's error rate against reference images, and cross-entropy, the objective function defined.
\begin{figure}[H]
    \centering    
    \caption{Training performance for all online datasets.}
    \vspace{6mm}
    \includegraphics[width=0.8\textwidth]{\dropbox/phd/results/evaluation/new-evaluation-output.pdf}       
	\legenda{}
    \label{training-result}
\end{figure}

The evaluation is performed repeatedly, according to a certain number of iterations, where partial inferences (prediction) is obtained and the measures are calculated by comparing the result against the ground truth (references). These metrics are commonly adopted in the literature about classification using neural models. The accuracy is calculated according to a pixelwise analysis, in which first the success and error rates are measured through values of True Positive (TP), True Negative (TN), False Positive (FP) and False Negative (FN), given by:
\begin{align}
 Accuracy &= \frac{\sum_{i=1, j=i}^n M_{ij}}{N}~,
\end{align}
where $M$ is a confusion matrix\footnote{The confusion matrix is generated according to the same premises of TP, TN, FP and FN (more details in Annex \ref{annexB}). For example, considering the ground truth in $x$ and predicted in $y$, the TP is given by elements in diagonal, TN, sum of all elements except the diagonal, FP, the sum of elements in $x$ minus diagonal and, FN, the sum of elements in $y$ minus diagonal.}, $n$, the number of classes, and $N$, the number of samples used. The numerator consists in the sum of all elements in diagonal.

An optimization problem seeks to minimize a loss-function. The weight-loss consists of the error levels in which the neural network has according to an ideal of prediction that is given by the reference images. This ideal is an optimization problem which aims to be minimal, called then loss-function. In this case, the loss-function is given by the cross-entropy \cite{long2015}, a common and effective way to calculate the distance between multiple predicted and ground-truth classes, also known as multinominal logistic classification, given by:
\begin{align}
 Loss_{y'}(y) &= -\sum_{i}y_i'~\log~(y_i)~,
\end{align}
where $y_i$ is the predicted probability value for class $i$, and $y_i'$ is the ground-truth probability for that class (more about cross-entropy calculation and meaning can be found in Annex \ref{annexA}). 

Analyzing the graph, it is simple to notice a similar behavior for all training dataset, except for the weight-loss. The weight-loss decay is strongly related to the image dimension, which of course requires more iterations to learn the features. The demand for the learning of all features (generalization) is greater and varies among them. Accuracy and cross-entropy, on the other hand, had progressed mostly from 0 to 10k iterations, stabilizing near 90\% and 0.1 thereafter, respectively. 30k iterations were sufficient to reach similar results for all datasets (as shown later in the visual inspection). However, RueMonge2014, ENPC, and Graz still had high error rates, which means that not all classes could be detected or clearly delineated, even with 50k iterations.

%[colocar imagem da evolu��o da segmentacao] - 10k, 30k e 50k
\begin{figure}[H]
    \centering    
    \caption{Accuracy evolution according to the number of iterations. Example using a image from eTRIMS dataset.}
    \vspace{6mm}
    \includegraphics[width=1\textwidth]{\dropbox/phd/pics/evolution/evolution.pdf}       
	\legenda{}
    \label{training-evolution}
\end{figure}

\subsection{Inference over the online datasets (Experiment 1)}
In this section, each of the online facade sets and SJC will be evaluated in detail. In this evaluation was adopted the same metric used during the training (accuracy) with the addition of the F1-score, which expresses the harmonic mean of precision and recall, given by:
\begin{align}
 F1 &= 2 * \frac{precision * recall}{precision + recall}~, 
\end{align}
where $precision=TP/(TP+FP)$, and $recall=TP/(TP+FN)$. Then, both values reveal how good the segmentation was according to the correct assignment (accuracy) and object delineation (F1-score).

\subsubsection{RueMonge2014}
The Figure \ref{overview-result-ruemonge} shows the inferences from RueMonge2014 over the validation set. Instead of showing only a few example results, they were exposed as much as possible to allow the reader to better understand how the neural model behaves according to different situations. Here, it is positively highlighted two aspects. First, the robustness of the neural model in the detection of facade features even under shadow or occluded areas, such as in the presence of pedestrians or cars. This aspect has been one of the most difficult issue to overcome due to the respective obstacles being dynamic and difficult to deal with, especially by the use of pixelwise segmenters. The second aspect is that at 50k, all images presented fine class delineation. Only in a few situations the inferences were not satisfactory. 
\begin{figure}[H]
    \centering	   
    \caption{Results over RueMonge2014 dataset. The rows are splited respectively in original, segmented image, and both. These segmented images are the inferences under evaluation sets only. (a)--(j) Example of RueMonge2014 images, segmented by the neural model presented in Section \ref{neural-model-description}. In the first line, the original image, the second line, the result of the inference (segmentation), and the third and last line, the overlapping images.}
    \vspace{6mm}
    \subfigure[]{\label{overview-result-ruemongea}\includegraphics[width=0.09\textwidth]{/data/phd/results/facades-benchmark/ruemonge2014/merge1/IMG_5857.png}}
    \subfigure[]{\label{overview-result-ruemongeb}\includegraphics[width=0.09\textwidth]{/data/phd/results/facades-benchmark/ruemonge2014/merge1/IMG_5599.png}}
    \subfigure[]{\label{overview-result-ruemongec}\includegraphics[width=0.09\textwidth]{/data/phd/results/facades-benchmark/ruemonge2014/merge1/IMG_5779.png}}
    \subfigure[]{\label{overview-result-ruemonged}\includegraphics[width=0.09\textwidth]{/data/phd/results/facades-benchmark/ruemonge2014/merge1/IMG_5757.png}}
    \subfigure[]{\label{overview-result-ruemongee}\includegraphics[width=0.09\textwidth]{/data/phd/results/facades-benchmark/ruemonge2014/merge1/IMG_5760.png}}
    \subfigure[]{\label{overview-result-ruemongef}\includegraphics[width=0.09\textwidth]{/data/phd/results/facades-benchmark/ruemonge2014/merge1/IMG_5726.png}}
    \subfigure[]{\label{overview-result-ruemongeg}\includegraphics[width=0.09\textwidth]{/data/phd/results/facades-benchmark/ruemonge2014/merge1/IMG_5546.png}}
    \subfigure[]{\label{overview-result-ruemongeh}\includegraphics[width=0.09\textwidth]{/data/phd/results/facades-benchmark/ruemonge2014/merge1/IMG_5692.png}}
    \subfigure[]{\label{overview-result-ruemongei}\includegraphics[width=0.09\textwidth]{/data/phd/results/facades-benchmark/ruemonge2014/merge1/IMG_5643.png}}
    \subfigure[]{\label{overview-result-ruemongej}\includegraphics[width=0.09\textwidth]{/data/phd/results/facades-benchmark/ruemonge2014/merge1/IMG_5840.png}}    
    \vspace{2mm}
	\legenda{}
    \label{overview-result-ruemonge}
    \FONTE{Author's production.}
\end{figure}

During manual annotation production for RueMonge2014, the labels did not cover the entire scene. For example, sky, street intersections or background buildings (adjacent to the main facades), were partially annotated as background, sometimes, completely. This means that when presented to the CNN during training, all those features (sky, street intersection, etc.), are going to be trained as background as well. Therefore, whenever an intersection or sky appears, the neural model treats it as being background. The problem is that only half of the feature will be assign as background, which is not the case with the other half. The same behavior was visible in other classes. For instance, when an facade is fully annotated and appears only partially in the validation set, the neural model will act as it was not presented in the image, only part of it (clear on Figures \ref{overview-result-ruemongef}, \ref{overview-result-ruemongeh}, and \ref{overview-result-ruemongei}). Supervised neural model is strongly related to the context it has been trained. If a feature appears in the image, but only part it is detected, the segmentation will fail because of the incomplete context.

In addition, the confusion matrix over all validation set is presented in Table \ref{cm-ruemonge}. Pretty close to the visual inspection, the RueMonge2014 confusion matrix shows good prediction over all classes, with a tiny confusion between background and wall against the other classes (first and forth columns). It is explained by the fact that the respective classes are the overall classes in the entire prediction, which means it share boundaries with all other classes. As the object delineation is not always clear, the evaluated pixels on the edge is predicted as been background or wall instead the right class. Although it is minimal ($0.8624$ as F1-score), the prediction over edges seems to be the bottleneck of the adopted CNN architecture for all online datasets, as shown in the following sections. The labels in the bottom of the table, express the classes presented in the respective dataset.
\begin{table}[H]
    \renewcommand{\arraystretch}{1.2}
    \caption{Normalized confusion matrix for RueMonge2014 predictions.}
    \scriptsize \centering		
    \begin{tabular}{L{0.05cm}L{0.05cm}L{0.05cm}L{1.4cm}C{0.57cm}C{0.57cm}C{0.57cm}C{0.57cm}C{0.57cm}C{0.57cm}C{0.57cm}C{0.57cm}C{1.45cm}C{1.3cm}}
        \toprule        
        \multirow{2}{*}{} & \multirow{2}{*}{} & \multirow{2}{*}{} & \multirow{2}{*}{\textbf{Classes}} & \multicolumn{8}{c}{\textbf{Predicted}} & \multirow{2}{*}{\textbf{Scale}} & \multirow{2}{*}{\textbf{Evaluation}} \\ \cmidrule{5-12}
        & & & & \textbf{1} & \textbf{2} & \textbf{3} & \textbf{4} & \textbf{5} & \textbf{6} & \textbf{7} & \textbf{8} & & \\
        \toprule
        \multirow{8}{*}{\rotatebox[origin=c]{90}{\textbf{\textbf{Ground-Truth}}}} & 1 & \textcolor{black}{\faCircle} & Background & \multicolumn{8}{l}{\multirow{8}{*}{\includegraphics[width=0.52\textwidth, height=0.22\textwidth]{\dropbox/phd/results/evaluation/cm-normalized/RueMonge2014.png}}} & \multirow{8}{*}{\includegraphics[width=0.039\textwidth]{\dropbox/phd/results/evaluation/cm-normalized/scale.png}} & \\
        & 2 & \textcolor{blue}{\faCircle} & Roof & & &\\      
        & 3 & \textcolor{myCyan}{\faCircle} & Sky & & &\\      
        & 4 & \textcolor{yellow}{\faCircle} & Wall & & &\\      
        & 5 & \textcolor{myPurple}{\faCircle} & Balcony & & &\\      
        & 6 & \textcolor{red}{\faCircle} & Window & & &\\      
        & 7 & \textcolor{orange}{\faCircle} & Door & & &\\      
        & 8 & \textcolor{green}{\faCircle} & Shop & & &\\       
        \bottomrule
        & & & \multirow{2}{*}{\textbf{Rates:}} & \multicolumn{8}{l}{} & \textbf{Accuracy}: & \textbf{0.9563}\\ \cmidrule{13-14}
        & & & & \multicolumn{8}{l}{} & \textbf{F1-Score}: & \textbf{0.8624}\\     
        \bottomrule
    \end{tabular}
    \label{cm-ruemonge}
    \FONTE{Author's production.}
\end{table}

\subsubsection{CMP}
The CMP annotation set present labels beyond those already analyzed in this study. In the original CMP dataset, for example, there are annotations for decoration and pillars, which are sub-elements of the class wall and, for that reason, out of the scope in this study. Even so, still considered as being a single class: wall. For other sub-elements, all the labels not related to this study were equally ignored and had their annotations adapted to the problem. 

\begin{figure}[H]
    \centering	       
    \caption{Results over CMP dataset. (a)--(g) Example of CMP images, segmented by the neural model presented in Section \ref{neural-model-description}. The three different rows correspond to the same description as in Figure \ref{overview-result-ruemonge}.}
    \vspace{6mm}
    \subfigure[]{\label{overview-result-cmpa}\includegraphics[width=0.14\textwidth]{/data/phd/results/facades-benchmark/cmp/merge1/cmp_b0126.png}}
    \subfigure[]{\label{overview-result-cmpb}\includegraphics[width=0.085\textwidth]{/data/phd/results/facades-benchmark/cmp/merge1/cmp_b0101.png}}
    \subfigure[]{\label{overview-result-cmpc}\includegraphics[width=0.103\textwidth]{/data/phd/results/facades-benchmark/cmp/merge1/cmp_b0017.png}}
    \subfigure[]{\label{overview-result-cmpd}\includegraphics[width=0.116\textwidth]{/data/phd/results/facades-benchmark/cmp/merge1/cmp_b0073.png}}
    \subfigure[]{\label{overview-result-cmpe}\includegraphics[width=0.147\textwidth]{/data/phd/results/facades-benchmark/cmp/merge1/cmp_b0271.png}}
    \subfigure[]{\label{overview-result-cmpf}\includegraphics[width=0.132\textwidth]{/data/phd/results/facades-benchmark/cmp/merge1/cmp_b0169.png}}
    \subfigure[]{\label{overview-result-cmpg}\includegraphics[width=0.107\textwidth]{/data/phd/results/facades-benchmark/cmp/merge1/cmp_b0031.png}}        
    \vspace{2mm}
	\legenda{}
    \label{overview-result-cmp}
\end{figure}

The Figure \ref{overview-result-cmp} shows results reached at 50k iterations of training for CMP dataset. Once the architectural style has straight lines and facades whose texture is homogeneous, as presented in the CMP, the results of segmentation tend to be better precisely because the disturbance factors are minimal. CMP does not have annotations for background, sky, roof, and shop, and for these classes, therefore, the prediction is replaced by the labels wall, window, balcony or door, for example, in Figure \ref{overview-result-cmpd}, there are stores predicted as wall. As good results, the Figures \ref{overview-result-cmpa}, \ref{overview-result-cmpc}, \ref{overview-result-cmpe} and \ref{overview-result-cmpg} highlight the predictions that did not get much confusion. Facades that share different texture, as in Figure \ref{overview-result-cmpf}, presented a lot of confusion between the classes balcony and wall, specially under building decoration regions.

\begin{table}[H]
    \renewcommand{\arraystretch}{1.2}
    \caption{Normalized confusion matrix for CMP predictions.}
    \scriptsize \centering		
    \begin{tabular}{L{0.05cm}L{0.05cm}L{0.05cm}L{1.4cm}C{0.57cm}C{0.57cm}C{0.57cm}C{0.57cm}C{0.57cm}C{0.57cm}C{0.57cm}C{0.57cm}C{1.45cm}C{1.3cm}}
        \toprule        
        \multirow{2}{*}{} & \multirow{2}{*}{} & \multirow{2}{*}{} & \multirow{2}{*}{\textbf{Classes}} & \multicolumn{8}{c}{\textbf{Predicted}} & \multirow{2}{*}{\textbf{Scale}} & \multirow{2}{*}{\textbf{Evaluation}} \\ \cmidrule{5-12}
        & & & & \textbf{1} & \textbf{2} & \textbf{3} & \textbf{4} & \textbf{5} & \textbf{6} & \textbf{7} & \textbf{8} & & \\
        \toprule
        \multirow{8}{*}{\rotatebox[origin=c]{90}{\textbf{Ground-Truth}}} & 1 & \textcolor{gray!30}{\faCircleThin} & Background & \multicolumn{8}{l}{\multirow{8}{*}{\includegraphics[width=0.52\textwidth, height=0.22\textwidth]{\dropbox/phd/results/evaluation/cm-normalized/CMP.png}}} & \multirow{8}{*}{\includegraphics[width=0.039\textwidth]{\dropbox/phd/results/evaluation/cm-normalized/scale.png}} & \\
        & 2 & \textcolor{gray!30}{\faCircleThin} & Roof & & &\\      
        & 3 & \textcolor{gray!30}{\faCircleThin} & Sky & & &\\      
        & 4 & \textcolor{yellow}{\faCircle} & Wall & & &\\      
        & 5 & \textcolor{myPurple}{\faCircle} & Balcony & & &\\      
        & 6 & \textcolor{red}{\faCircle} & Window & & &\\      
        & 7 & \textcolor{orange}{\faCircle} & Door & & &\\      
        & 8 & \textcolor{gray!30}{\faCircleThin} & Shop & & &\\       
        \bottomrule
        & & & \multirow{2}{*}{\textbf{Rates:}} & \multicolumn{8}{l}{} & \textbf{Accuracy}: & \textbf{0.9357}\\ \cmidrule{13-14}
        & & & & \multicolumn{8}{l}{} & \textbf{F1-Score}: & \textbf{0.7418}\\     
        \bottomrule
    \end{tabular}
    \label{cm-cmp}
    \FONTE{Author's production.}
\end{table}

Among the four classes, only wall and window got good scores (TP), reaching up accuracy of $0.9357$ for the entire prediction. Balcony and door were often confused with wall, specially around the boundaries of the classes, what also explain the F1-score of $0.7418$.

\subsubsection{eTRIMS}
As in CMP, eTRIMS had more annotations than the ones adopted in this study. In addition to the classes not approached in this study, there were facade features where the annotation belonged to only one class, e.g. the roof in eTRIMS is annotated as being wall. For that reason, images with roof had it assigned as wall and, consequently, assumed as True Positive (TP). Among the 6 facade features of interest, only three in eTRIMS were considered: window, door and wall. 

As can seen in Figure \ref{overview-result-etrims}, the predictions over eTRIMS dataset reached the second best score among the online datasets, good accuracy (object location), $0.9632$, and F1-score (object delineation), $0.8291$. The eTRIMS consist of facade pictures of different styles, \textquotedblleft non-patterned\textquotedblright, non-rectified, and consequently, the closest facade layout to the SJC data. 
\begin{figure}[H]
    \centering	       
    \caption{Results over eTRIMS dataset. (a)--(h) Example of eTRIMS images, segmented by the neural model presented in Section \ref{neural-model-description}. The three different rows correspond to the same description as in Figure \ref{overview-result-ruemonge}.}
    \vspace{6mm}
    \subfigure[]{\label{overview-result-etrimsa}\includegraphics[width=0.15\textwidth]{/data/phd/results/facades-benchmark/etrims/merge1/basel_000080_mv0.png}}
    \subfigure[]{\label{overview-result-etrimsb}\includegraphics[width=0.15\textwidth]{/data/phd/results/facades-benchmark/etrims/merge1/basel_000074_mv0.png}}
    \subfigure[]{\label{overview-result-etrimsc}\includegraphics[width=0.15\textwidth]{/data/phd/results/facades-benchmark/etrims/merge1/basel_000070_mv0.png}}
    \subfigure[]{\label{overview-result-etrimsd}\includegraphics[width=0.067\textwidth]{/data/phd/results/facades-benchmark/etrims/merge1/basel_000004_mv0.png}}
    \subfigure[]{\label{overview-result-etrimse}\includegraphics[width=0.067\textwidth]{/data/phd/results/facades-benchmark/etrims/merge1/heidelberg_000035_mv0.png}}
    \subfigure[]{\label{overview-result-etrimsf}\includegraphics[width=0.067\textwidth]{/data/phd/results/facades-benchmark/etrims/merge1/basel_000052_mv0.png}}
    \subfigure[]{\label{overview-result-etrimsg}\includegraphics[width=0.067\textwidth]{/data/phd/results/facades-benchmark/etrims/merge1/basel_000010_mv0.png}}
    \subfigure[]{\label{overview-result-etrimsh}\includegraphics[width=0.15\textwidth]{/data/phd/results/facades-benchmark/etrims/merge1/basel_000073_mv0.png}}        
	\vspace{2mm}
    \legenda{}
    \label{overview-result-etrims}
\end{figure}       
Considering a system of mapping of walls, windows and doors, the predictions made on this dataset show the ability of the neural network to detect the urban elements accurately (Figure \ref{overview-result-etrimsb}). The eTRIMS was distinguished compare to others because fewer classes were used (neural network learns more), also, the quality of annotations is better. In any case, objects obstructing the facade are ignored, precisely because training data consider these objects as being another class. Here, considered as background and therefore not part of the facade. This lack of information, for example in Figures \ref{overview-result-etrimsc} and \ref{overview-result-etrimsc}, is the best scenario in the detection of features, still, it will need an algorithm to regularize these missing informations.

The confusion matrix of eTRIMS show a tiny confusion between window, door and wall. Once again, the FN rates regarding to this prediction are mainly related to the boundaries between one class and another.
\begin{table}[H]
    \renewcommand{\arraystretch}{1.2}
    \caption{Normalized confusion matrix for eTRIMS predictions.}
    \scriptsize \centering		
    \begin{tabular}{L{0.05cm}L{0.05cm}L{0.05cm}L{1.4cm}C{0.57cm}C{0.57cm}C{0.57cm}C{0.57cm}C{0.57cm}C{0.57cm}C{0.57cm}C{0.57cm}C{1.45cm}C{1.3cm}}
        \toprule        
        \multirow{2}{*}{} & \multirow{2}{*}{} & \multirow{2}{*}{} & \multirow{2}{*}{\textbf{Classes}} & \multicolumn{8}{c}{\textbf{Predicted}} & \multirow{2}{*}{\textbf{Scale}} & \multirow{2}{*}{\textbf{Evaluation}} \\ \cmidrule{5-12}
        & & & & \textbf{1} & \textbf{2} & \textbf{3} & \textbf{4} & \textbf{5} & \textbf{6} & \textbf{7} & \textbf{8} & & \\
        \toprule
        \multirow{8}{*}{\rotatebox[origin=c]{90}{\textbf{Ground-Truth}}} & 1 & \textcolor{black}{\faCircle} & Background & \multicolumn{8}{l}{\multirow{8}{*}{\includegraphics[width=0.52\textwidth, height=0.22\textwidth]{\dropbox/phd/results/evaluation/cm-normalized/eTRIMS.png}}} & \multirow{8}{*}{\includegraphics[width=0.039\textwidth]{\dropbox/phd/results/evaluation/cm-normalized/scale.png}} & \\
        & 2 & \textcolor{gray!30}{\faCircleThin} & Roof & & &\\      
        & 3 & \textcolor{gray!30}{\faCircleThin} & Sky & & &\\      
        & 4 & \textcolor{yellow}{\faCircle} & Wall & & &\\      
        & 5 & \textcolor{gray!30}{\faCircleThin} & Balcony & & &\\      
        & 6 & \textcolor{red}{\faCircle} & Window & & &\\      
        & 7 & \textcolor{orange}{\faCircle} & Door & & &\\      
        & 8 & \textcolor{gray!30}{\faCircleThin} & Shop & & &\\       
        \bottomrule
        & & & \multirow{2}{*}{\textbf{Rates:}} & \multicolumn{8}{l}{} & \textbf{Accuracy}: & \textbf{0.9632}\\ \cmidrule{13-14}
        & & & & \multicolumn{8}{l}{} & \textbf{F1-Score}: & \textbf{0.8291}\\     
        \bottomrule
    \end{tabular}
    \label{cm-etrims}
    \FONTE{Author's production.}
\end{table}

\subsubsection{ENPC}
The level of accuracy for all sets made the use of CNN the best of all alternatives. However, as well as noticed in eTRIMS, when an object such as trees appear right in front of the facade, they might add disturbances in the training phase. In case it was a tree, it could be annotated as either vegetation or part of the facade itself (for example, note the differences between Figures \ref{overview-result-etrimsb} and \ref{overview-result-enpcb}). 

It is understood that the lack of information in the first figure is the best inference, and in this case, the neural model is actually right: there is a facade with unknown object in front of it. But in cases such as in Figure \ref{overview-result-enpcb}, the facade inference is noisy or unreadable, which is not the case in Figure \ref{overview-result-enpch}, where the disturbance is minimal. 
\begin{figure}[H]
    \centering
    \caption{Results over ENPC dataset. (a)--(k) Example of ENPC images, segmented by the neural model presented in Section \ref{neural-model-description}. The three different rows correspond to the same description as in Figure \ref{overview-result-ruemonge}.}
    \vspace{6mm}
    \subfigure[]{\label{overview-result-enpca}\includegraphics[width=0.082\textwidth]{/data/phd/results/facades-benchmark/enpc/merge1/facade_18.png}}
    \subfigure[]{\label{overview-result-enpcb}\includegraphics[width=0.082\textwidth]{/data/phd/results/facades-benchmark/enpc/merge1/facade_46.png}}
    \subfigure[]{\label{overview-result-enpcc}\includegraphics[width=0.062\textwidth]{/data/phd/results/facades-benchmark/enpc/merge1/facade_53.png}}
    \subfigure[]{\label{overview-result-enpcd}\includegraphics[width=0.062\textwidth]{/data/phd/results/facades-benchmark/enpc/merge1/facade_73.png}}
    \subfigure[]{\label{overview-result-enpce}\includegraphics[width=0.054\textwidth]{/data/phd/results/facades-benchmark/enpc/merge1/facade_54.png}}
    \subfigure[]{\label{overview-result-enpcf}\includegraphics[width=0.062\textwidth]{/data/phd/results/facades-benchmark/enpc/merge1/facade_65.png}}
    \subfigure[]{\label{overview-result-enpcg}\includegraphics[width=0.062\textwidth]{/data/phd/results/facades-benchmark/enpc/merge1/facade_66.png}}
    \subfigure[]{\label{overview-result-enpch}\includegraphics[width=0.068\textwidth]{/data/phd/results/facades-benchmark/enpc/merge1/facade_13.png}}
    \subfigure[]{\label{overview-result-enpci}\includegraphics[width=0.063\textwidth]{/data/phd/results/facades-benchmark/enpc/merge1/facade_28.png}}    
    \subfigure[]{\label{overview-result-enpcj}\includegraphics[width=0.078\textwidth]{/data/phd/results/facades-benchmark/enpc/merge1/facade_71.png}}  
    \subfigure[]{\label{overview-result-enpcl}\includegraphics[width=0.037\textwidth]{/data/phd/results/facades-benchmark/enpc/merge1/facade_12.png}}    
    \vspace{2mm}
    \legenda{}
    \label{overview-result-enpc}
\end{figure}

In the confusion matrix for ENPC predictions, it is revealed that all classes has been assigned correctly (accuracy $0.9636$ and F1-score $0.7655$), except for background. Once this dataset only have rectified images (facade covering the entire image), the class background actually should not exist. But once defined, the assignment to this class is splited over other options: wall, balcony, and window (mostly). Roof and door were the classes that got lower rates of TP, frequent confusing with sky and wall classes.
\begin{table}[H]
    \renewcommand{\arraystretch}{1.2}
    \caption{Normalized confusion matrix for ENPC predictions.}
    \scriptsize \centering		
    \begin{tabular}{L{0.05cm}L{0.05cm}L{0.05cm}L{1.4cm}C{0.57cm}C{0.57cm}C{0.57cm}C{0.57cm}C{0.57cm}C{0.57cm}C{0.57cm}C{0.57cm}C{1.45cm}C{1.3cm}}
        \toprule        
        \multirow{2}{*}{} & \multirow{2}{*}{} & \multirow{2}{*}{} & \multirow{2}{*}{\textbf{Classes}} & \multicolumn{8}{c}{\textbf{Predicted}} & \multirow{2}{*}{\textbf{Scale}} & \multirow{2}{*}{\textbf{Evaluation}} \\ \cmidrule{5-12}
        & & & & \textbf{1} & \textbf{2} & \textbf{3} & \textbf{4} & \textbf{5} & \textbf{6} & \textbf{7} & \textbf{8} & & \\
        \toprule
        \multirow{8}{*}{\rotatebox[origin=c]{90}{\textbf{Ground-Truth}}} & 1 & \textcolor{black}{\faCircle} & Background & \multicolumn{8}{l}{\multirow{8}{*}{\includegraphics[width=0.52\textwidth, height=0.22\textwidth]{\dropbox/phd/results/evaluation/cm-normalized/ENPC.png}}} & \multirow{8}{*}{\includegraphics[width=0.039\textwidth]{\dropbox/phd/results/evaluation/cm-normalized/scale.png}} & \\
        & 2 & \textcolor{blue}{\faCircle} & Roof & & &\\      
        & 3 & \textcolor{myCyan}{\faCircle} & Sky & & &\\      
        & 4 & \textcolor{yellow}{\faCircle} & Wall & & &\\      
        & 5 & \textcolor{myPurple}{\faCircle} & Balcony & & &\\      
        & 6 & \textcolor{red}{\faCircle} & Window & & &\\      
        & 7 & \textcolor{orange}{\faCircle} & Door & & &\\      
        & 8 & \textcolor{green}{\faCircle} & Shop & & &\\       
        \bottomrule
        & & & \multirow{2}{*}{\textbf{Rates:}} & \multicolumn{8}{l}{} & \textbf{Accuracy}: & \textbf{0.9636}\\ \cmidrule{13-14}
        & & & & \multicolumn{8}{l}{} & \textbf{F1-Score}: & \textbf{0.7655}\\     
        \bottomrule
    \end{tabular}
    \label{cm-enpc}
    \FONTE{Author's production.}
\end{table}

\subsubsection{ECP}
All online datasets does not have any certificate of quality. When checking the annotated images of some of them, there is a high degree of inconsistency between the annotations. This implies incorrect segmentation (see overlapping images - detail on the roofs) according to the real scenario -- visual inspection, but not to the validation set. It means the validation metrics might present some inconsistency, since they are calculated according to the validation (annotated) images. 

ECP also presented inconsistencies in some of its annotations. The missing roof-parts in Figures \ref{overview-result-ecpa} to \ref{overview-result-ecpf} are expected behaviors since the annotations from the training sets do not consider these objects as being part of the roof. However, the learning happens for most of the features and should not be a problem since the neural model will identify the main content in the image.

\begin{figure}[H]
    \centering
    \caption{Results over ECP dataset. (a)--(l) Example of ECP images, segmented by the neural model presented in Section \ref{neural-model-description}. The three different rows correspond to the same description as in Figure \ref{overview-result-ruemonge}.}
    \vspace{6mm}
    \subfigure[]{\label{overview-result-ecpa}\includegraphics[width=0.056\textwidth]{/data/phd/results/facades-benchmark/ecp/merge1/monge_105.png}}
    \subfigure[]{\label{overview-result-ecpb}\includegraphics[width=0.041\textwidth]{/data/phd/results/facades-benchmark/ecp/merge1/monge_12.png}}    
    \subfigure[]{\label{overview-result-ecpc}\includegraphics[width=0.059\textwidth]{/data/phd/results/facades-benchmark/ecp/merge1/monge_72.png}}    
    \subfigure[]{\label{overview-result-ecpd}\includegraphics[width=0.088\textwidth]{/data/phd/results/facades-benchmark/ecp/merge1/monge_24.png}}    
    \subfigure[]{\label{overview-result-ecpe}\includegraphics[width=0.059\textwidth]{/data/phd/results/facades-benchmark/ecp/merge1/monge_68.png}}    
    \subfigure[]{\label{overview-result-ecpf}\includegraphics[width=0.066\textwidth]{/data/phd/results/facades-benchmark/ecp/merge1/monge_53.png}}    
    \subfigure[]{\label{overview-result-ecpg}\includegraphics[width=0.089\textwidth]{/data/phd/results/facades-benchmark/ecp/merge1/monge_29.png}}    
    \subfigure[]{\label{overview-result-ecph}\includegraphics[width=0.083\textwidth]{/data/phd/results/facades-benchmark/ecp/merge1/monge_25.png}}        
    \subfigure[]{\label{overview-result-ecpi}\includegraphics[width=0.051\textwidth]{/data/phd/results/facades-benchmark/ecp/merge1/monge_66.png}}    
    \subfigure[]{\label{overview-result-ecpj}\includegraphics[width=0.044\textwidth]{/data/phd/results/facades-benchmark/ecp/merge1/monge_5.png}}    
    \subfigure[]{\label{overview-result-ecpl}\includegraphics[width=0.058\textwidth]{/data/phd/results/facades-benchmark/ecp/merge1/monge_69.png}}    
    \subfigure[]{\label{overview-result-ecpm}\includegraphics[width=0.042\textwidth]{/data/phd/results/facades-benchmark/ecp/merge1/monge_103.png}} 
    \vspace{2mm}
    \legenda{}
    \label{overview-result-ecp}
\end{figure}

According to the accuracy and F1-Score metrics, ECP got the best scores among the other, with $0.9762$ and $0.8946$, respectively. The lower TP was $0.841$ for class window, which is already considered a good mark since the other classes got better scores. A tiny confusion between balcony, window, and wall, is highlighted, as well as for door and shop.   
\begin{table}[H]
    \renewcommand{\arraystretch}{1.2}
    \caption{Normalized confusion matrix for ECP predictions.}
    \scriptsize \centering		
    \begin{tabular}{L{0.05cm}L{0.05cm}L{0.05cm}L{1.4cm}C{0.57cm}C{0.57cm}C{0.57cm}C{0.57cm}C{0.57cm}C{0.57cm}C{0.57cm}C{0.57cm}C{1.45cm}C{1.3cm}}
        \toprule        
        \multirow{2}{*}{} & \multirow{2}{*}{} & \multirow{2}{*}{} & \multirow{2}{*}{\textbf{Classes}} & \multicolumn{8}{c}{\textbf{Predicted}} & \multirow{2}{*}{\textbf{Scale}} & \multirow{2}{*}{\textbf{Evaluation}} \\ \cmidrule{5-12}
        & & & & \textbf{1} & \textbf{2} & \textbf{3} & \textbf{4} & \textbf{5} & \textbf{6} & \textbf{7} & \textbf{8} & & \\
        \toprule
        \multirow{8}{*}{\rotatebox[origin=c]{90}{\textbf{Ground-Truth}}} & 1 & \textcolor{gray!30}{\faCircleThin} & Background & \multicolumn{8}{l}{\multirow{8}{*}{\includegraphics[width=0.52\textwidth, height=0.22\textwidth]{\dropbox/phd/results/evaluation/cm-normalized/ECP.png}}} & \multirow{8}{*}{\includegraphics[width=0.039\textwidth]{\dropbox/phd/results/evaluation/cm-normalized/scale.png}} & \\
        & 2 & \textcolor{blue}{\faCircle} & Roof & & &\\      
        & 3 & \textcolor{myCyan}{\faCircle} & Sky & & &\\      
        & 4 & \textcolor{yellow}{\faCircle} & Wall & & &\\      
        & 5 & \textcolor{myPurple}{\faCircle} & Balcony & & &\\      
        & 6 & \textcolor{red}{\faCircle} & Window & & &\\      
        & 7 & \textcolor{orange}{\faCircle} & Door & & &\\      
        & 8 & \textcolor{green}{\faCircle} & Shop & & &\\       
        \bottomrule
        & & & \multirow{2}{*}{\textbf{Rates:}} & \multicolumn{8}{l}{} & \textbf{Accuracy}: & \textbf{0.9762}\\ \cmidrule{13-14}
        & & & & \multicolumn{8}{l}{} & \textbf{F1-Score}: & \textbf{0.8946}\\     
        \bottomrule
    \end{tabular}
    \label{cm-ecp}
    \FONTE{Author's production.}
\end{table}

\subsubsection{Graz}
Among the inputs, Graz has the smallest number of images, but the spectral variability is clearly greater when compared to the others. The symmetry between windows, however, was pretty much the same in CMP, ECP, and ENPC. It is noticed, then, that the results for Graz (Figure \ref{overview-result-graz}) did not change much from what was seen in other datasets.
\begin{figure}[H]
    \centering
    \caption{Results over Graz dataset. (a)--(h) Example of Graz images, segmented by the neural model presented in Section \ref{neural-model-description}. The three different rows correspond to the same description as in Figure \ref{overview-result-ruemonge}.}
    \vspace{6mm}
    \subfigure[]{\includegraphics[width=0.1\textwidth]{/data/phd/results/facades-benchmark/graz/merge1/facade_0_0081240_0081479.png}}
    \subfigure[]{\includegraphics[width=0.123\textwidth]{/data/phd/results/facades-benchmark/graz/merge1/facade_0_0102661_0102921.png}}
    \subfigure[]{\includegraphics[width=0.103\textwidth]{/data/phd/results/facades-benchmark/graz/merge1/facade_0_0099285_0099509.png}}
    \subfigure[]{\includegraphics[width=0.088\textwidth]{/data/phd/results/facades-benchmark/graz/merge1/facade_0_0084655_0084815.png}}
    \subfigure[]{\includegraphics[width=0.115\textwidth]{/data/phd/results/facades-benchmark/graz/merge1/facade_0_0101200_0101389.png}}
    \subfigure[]{\includegraphics[width=0.145\textwidth]{/data/phd/results/facades-benchmark/graz/merge1/facade_0_0082229_0082431.png}}
    \subfigure[]{\includegraphics[width=0.11\textwidth]{/data/phd/results/facades-benchmark/graz/merge1/facade_0_0078702_0078926.png}}
    \subfigure[]{\includegraphics[width=0.107\textwidth]{/data/phd/results/facades-benchmark/graz/merge1/facade_1_0056345_0056536.png}} 
    \vspace{2mm}
    \legenda{}
    \label{overview-result-graz}
\end{figure}

Even the predictions has reached accuracy of $0.9368$ and F1-score of $0.7698$, a lot of confusion involving the class wall can be notice in Table \ref{cm-graz}, specially for balcony and sky, where the predictions were almost null. 
\begin{table}[H]
    \renewcommand{\arraystretch}{1.2}
    \caption{Normalized confusion matrix for Graz predictions.}
    \scriptsize \centering		
    \begin{tabular}{L{0.05cm}L{0.05cm}L{0.05cm}L{1.4cm}C{0.57cm}C{0.57cm}C{0.57cm}C{0.57cm}C{0.57cm}C{0.57cm}C{0.57cm}C{0.57cm}C{1.45cm}C{1.3cm}}
        \toprule        
        \multirow{2}{*}{} & \multirow{2}{*}{} & \multirow{2}{*}{} & \multirow{2}{*}{\textbf{Classes}} & \multicolumn{8}{c}{\textbf{Predicted}} & \multirow{2}{*}{\textbf{Scale}} & \multirow{2}{*}{\textbf{Evaluation}} \\ \cmidrule{5-12}
        & & & & \textbf{1} & \textbf{2} & \textbf{3} & \textbf{4} & \textbf{5} & \textbf{6} & \textbf{7} & \textbf{8} & & \\
        \toprule
        \multirow{8}{*}{\rotatebox[origin=c]{90}{\textbf{Ground-Truth}}} & 1 & \textcolor{black}{\faCircle} & Background & \multicolumn{8}{l}{\multirow{8}{*}{\includegraphics[width=0.52\textwidth, height=0.22\textwidth]{\dropbox/phd/results/evaluation/cm-normalized/Graz.png}}} & \multirow{8}{*}{\includegraphics[width=0.039\textwidth]{\dropbox/phd/results/evaluation/cm-normalized/scale.png}} & \\
        & 2 & \textcolor{blue}{\faCircle} & Roof & & &\\      
        & 3 & \textcolor{myCyan}{\faCircle} & Sky & & &\\      
        & 4 & \textcolor{yellow}{\faCircle} & Wall & & &\\      
        & 5 & \textcolor{myPurple}{\faCircle} & Balcony & & &\\      
        & 6 & \textcolor{red}{\faCircle} & Window & & &\\      
        & 7 & \textcolor{orange}{\faCircle} & Door & & &\\      
        & 8 & \textcolor{green}{\faCircle} & Shop & & &\\       
        \bottomrule
        & & & \multirow{2}{*}{\textbf{Rates:}} & \multicolumn{8}{l}{} & \textbf{Accuracy}: & \textbf{0.9368}\\ \cmidrule{13-14}
        & & & & \multicolumn{8}{l}{} & \textbf{F1-Score}: & \textbf{0.7698}\\     
        \bottomrule
    \end{tabular}
    \label{cm-graz}
    \FONTE{Author's production.}
\end{table}

In Table \ref{training-validation-independent}, it is summarized the accuracies and F1-scores for each online dataset. ECP presented the best results among the datasets, not only the accuracy, but also in precision (F1-score). Datasets which accuracy is high but F1-score inferior, demonstrated an excellent inference in which region the object was found on the image, but not equally efficient regarding its delineation, that was the case of CMP, ENPC and Graz. The predictions on RueMonge2014 and eTRIMS datasets presented better quality in both metrics. 

As the evaluation was performed under multiple validation images, the columns in red and green represent the variance and standard deviation, respectively. None of them have reached significant variations.
\begin{table}[H]
    \renewcommand{\arraystretch}{1.4}
    \caption{Inference accuracy over the online datasets. Var. (Variance) and StD. (Standard Deviation) stands for the inferences over different images. The values in bold, expose the best datasets according to the Accuracy and F1-Score metrics.}
    \scriptsize \centering		
    \begin{tabular}{L{2.8cm}C{2.4cm}C{1.1cm}C{1.1cm}C{2.4cm}C{1.1cm}C{1.1cm}}
        \toprule
        \textbf{Dataset} & \textbf{Accuracy} & \cellcolor{red!20}\textbf{Var.} & \cellcolor{green!20}\textbf{StD.} & \textbf{F1-score} & \cellcolor{red!20}\textbf{Var.} & \cellcolor{green!20}\textbf{StD.}\\ 
        \toprule
        RueMonge2014 & 0.9563 & \cellcolor{red!20}0.008 & \cellcolor{green!20}0.090 & 0.8624 & \cellcolor{red!20}0.000 & \cellcolor{green!20}0.027 \\      
        CMP & 0.9357 & \cellcolor{red!20}0.005 & \cellcolor{green!20}0.073 & 0.7418 & \cellcolor{red!20}0.001 & \cellcolor{green!20}0.043\\
        eTRIMS & 0.9632 & \cellcolor{red!20}0.000 & \cellcolor{green!20}0.027 & 0.8291 & \cellcolor{red!20}0.000 & \cellcolor{green!20}0.017\\
        ENPC & 0.9636 & \cellcolor{red!20}0.001 & \cellcolor{green!20}0.031 & 0.7655 & \cellcolor{red!20}0.000 & \cellcolor{green!20}0.009\\
        ECP & \textbf{0.9762} & \cellcolor{red!20}0.000 & \cellcolor{green!20}0.021 & \textbf{0.8946} & \cellcolor{red!20}0.000 & \cellcolor{green!20}0.014\\
        Graz & 0.9368 & \cellcolor{red!20}0.014 & \cellcolor{green!20}0.117 & 0.7698 & \cellcolor{red!20}0.000 & \cellcolor{green!20}0.023\\
        \bottomrule
    \end{tabular}
    \label{training-validation-independent}
    \FONTE{Author's production.}
\end{table}

\subsection{Inference over the SJC dataset (Experiments 2 and 3)}
The idea behind the usage of the SJC dataset was simple: to observe how the neural network reacts to an unknown architectural style after being trained with different ones. The outcome could then provide insights on how the training set should look like for the detection of facades of any kind. Figure \ref{overview-result-all-together} shows the results after presenting SJC images to a different version of training data (knowledge).    
\begin{figure}[H]
    \centering
    \caption{Segmented image from SJC dataset. Inferences between individual training knowledges. (a) Result using RueMonge2014 knowledge, (b) CMP, (c) eTRIMS, (d) ECP, (e) ENPC, (f) and Graz. (g) result using all knowledge.}
    \vspace{6mm}
    \subfigure[]{\label{overview-result-all-togethera}\includegraphics[width=0.15\textwidth]{/data/phd/results/facades-benchmark/sjc/inferences/ruemonge-knowledge/merge1/IMG_7754.png}}    
    \subfigure[]{\label{overview-result-all-togetherc}\includegraphics[width=0.15\textwidth]{/data/phd/results/facades-benchmark/sjc/inferences/cmp-knowledge/merge1/IMG_7754.png}}
    \subfigure[]{\label{overview-result-all-togetherb}\includegraphics[width=0.15\textwidth]{/data/phd/results/facades-benchmark/sjc/inferences/etrims-knowledge/merge1/IMG_7754.png}}
    \subfigure[]{\label{overview-result-all-togetherd}\includegraphics[width=0.15\textwidth]{/data/phd/results/facades-benchmark/sjc/inferences/ecp-knowledge/merge1/IMG_7754.png}}
    \subfigure[]{\label{overview-result-all-togethere}\includegraphics[width=0.15\textwidth]{/data/phd/results/facades-benchmark/sjc/inferences/enpc-knowledge/merge1/IMG_7754.png}}
    \subfigure[]{\label{overview-result-all-togetherf}\includegraphics[width=0.15\textwidth]{/data/phd/results/facades-benchmark/sjc/inferences/graz-knowledge/merge1/IMG_7754.png}}
    \subfigure[]{\label{overview-result-all-togetherg}\includegraphics[width=1\textwidth]{/data/phd/results/facades-benchmark/sjc/inferences/all-together-knowledge/IMG_7754.png}}
    \vspace{2mm}
    \legenda{}    
    \label{overview-result-all-together}
    \FONTE{Author's production.}
\end{figure}

The Figures \ref{overview-result-all-togethera} to \ref{overview-result-all-togetherf} are the respective results from datasets listed in Table \ref{datasets} (online). When looking at these results as seen in the figures, it is safely concluded that these are incorrect and inaccurate segmentations. The fact is that in environments whose diversity of objects, any other segmentation and classification methods would add a certain imprecision on it. The operation of a CNN is not to perfectly delineate an object, but to provide hints (as close as possible) to where a given object is located in the image. This shows us that in order to extract precise parameters, such as height and area of a feature, a post-processing phase should certainly be conducted on the CNN beforehand.

Going through one problem at a time, it is noticed that, firstly, there is the need to define a background class in supervised approaches. Using RueMonge2014 knowledge, the inference process was not able to segment properly even under the most common feature: wall. Unlikely, some knowledge generalized it to sky, sidewalk and street, especially when there is no general class that represents too many objects in the scene (Figure \ref{overview-result-all-togetherc}). When the class is annotated correctly such as sky, a proper segmentation can be seen: that is the case with ECP (Figure \ref{overview-result-all-togetherd}), ENPC (Figure \ref{overview-result-all-togethere}), and Graz (Figure \ref{overview-result-all-togetherf}). When it is not, the inference is poor or average: which was the case with RueMonge2014 (Figure \ref{overview-result-all-togethera}). Features in RueMonge2014 were pretty much dependent on local architectural style. In general, eTRIMS is the dataset with the most similar features for SJC. Despite being trained with only 4 classes, including background, the results have shown a certain level of intelligence in detecting sidewalks and street as being part of the background, as well as for sky and vegetation.

Therefore, when using a supervised neural network, it is evident that the arrangement of the annotations can affect the inference, either positively or negatively. For instance, annotations for all the sky coverage instead of only part of it, or background annotations for sidewalks and street, in cases that it is not a desired feature. The confusion matrix of all-together knowledge applied in SJC data is shown in Table \ref{cm-sjc}.

The confusion can be mainly described between the classes roof, sky, balcony and door. Only classes background, wall and window were reasonably assigned, most influenced by the knowledge from eTRIMS and Graz (Figures \ref{overview-result-all-togetherb} and \ref{overview-result-all-togetherf}).
\begin{table}[H]
    \renewcommand{\arraystretch}{1.2}
    \caption{Normalized confusion matrix for SJC (all-together) predictions.}
    \scriptsize \centering		
    \begin{tabular}{L{0.05cm}L{0.05cm}L{0.05cm}L{1.4cm}C{0.57cm}C{0.57cm}C{0.57cm}C{0.57cm}C{0.57cm}C{0.57cm}C{0.57cm}C{0.57cm}C{1.45cm}C{1.3cm}}
        \toprule        
        \multirow{2}{*}{} & \multirow{2}{*}{} & \multirow{2}{*}{} & \multirow{2}{*}{\textbf{Classes}} & \multicolumn{8}{c}{\textbf{Predicted}} & \multirow{2}{*}{\textbf{Scale}} & \multirow{2}{*}{\textbf{Evaluation}} \\ \cmidrule{5-12}
        & & & & \textbf{1} & \textbf{2} & \textbf{3} & \textbf{4} & \textbf{5} & \textbf{6} & \textbf{7} & \textbf{8} & & \\
        \toprule
        \multirow{8}{*}{\rotatebox[origin=c]{90}{\textbf{Ground-Truth}}} & 1 & \textcolor{black}{\faCircle} & Background & \multicolumn{8}{l}{\multirow{8}{*}{\includegraphics[width=0.52\textwidth, height=0.22\textwidth]{\dropbox/phd/results/evaluation/cm-normalized/all-together.png}}} & \multirow{8}{*}{\includegraphics[width=0.039\textwidth]{\dropbox/phd/results/evaluation/cm-normalized/scale.png}} & \\
        & 2 & \textcolor{blue}{\faCircle} & Roof & & &\\      
        & 3 & \textcolor{myCyan}{\faCircle} & Sky & & &\\      
        & 4 & \textcolor{yellow}{\faCircle} & Wall & & &\\      
        & 5 & \textcolor{myPurple}{\faCircle} & Balcony & & &\\      
        & 6 & \textcolor{red}{\faCircle} & Window & & &\\      
        & 7 & \textcolor{orange}{\faCircle} & Door & & &\\      
        & 8 & \textcolor{green}{\faCircle} & Shop & & &\\       
        \bottomrule
        & & & \multirow{2}{*}{\textbf{Rates:}} & \multicolumn{8}{l}{\textcolor{black}{\faCircle}~\textcolor{myCyan}{\faCircle}~\textcolor{blue}{\faCircle}~\textcolor{yellow}{\faCircle}~\textcolor{red}{\faCircle}~\textcolor{myPurple}{\faCircle}~\textcolor{orange}{\faCircle}} & \textbf{Accuracy}: & \textbf{0.8591}\\ \cmidrule{13-14}
        & & & & \multicolumn{8}{l}{} & \textbf{F1-Score}: & \textbf{0.4706}\\     
        \bottomrule
    \end{tabular}
    \label{cm-sjc}
    \FONTE{Author's production.}
\end{table}

In Figure \ref{overview-result-all-togetherg}, a summarized contributions of each dataset is presented. For instance, eTRIMS was the only one sensitive to sidewalk and street, balconies were only detected in CMP, even though this was incorrectly segmented. Meanwhile, the results for unknown features were understandable and expected. With the addition of more classes (e.g. gate), improvements to the annotation process and an increase in the number of training epochs, the better the results of the inferences would be. Table \ref{training-validation} below shows the accuracy overview for each individual learned feature (knowledge) over the SJC dataset.  
\begin{table}[H]
    \renewcommand{\arraystretch}{1.4}
    \caption{Inference accuracy over SJC data. Last row corresponds to the accuracy with the knowledge of all training together. The values in bold, expose the best datasets according to the Accuracy and F1-Score metrics. When together, the quality metrics increased due to the better generalization of the neural network, as it has received a bigger amount of images.}
    \scriptsize \centering		
    \begin{tabular}{L{2.8cm}C{2.4cm}C{1.1cm}C{1.1cm}C{2.4cm}C{1.1cm}C{1.1cm}}
        \toprule
        \textbf{Knowledge from...} & \textbf{Accuracy} & \cellcolor{red!20}\textbf{Var.} & \cellcolor{green!20}\textbf{StD.} & \textbf{F1-score} & \cellcolor{red!20}\textbf{Var.} & \cellcolor{green!20}\textbf{StD.}\\ 
        \toprule
        RueMonge2014 & 0.7009 & \cellcolor{red!20}0.003 & \cellcolor{green!20}0.054 & 0.1612 & \cellcolor{red!20}0.000 & \cellcolor{green!20}0.014 \\      
        CMP & 0.7907 & \cellcolor{red!20}0.006 & \cellcolor{green!20}0.080 & 0.3763 & \cellcolor{red!20}0.001 & \cellcolor{green!20}0.043\\
        eTRIMS & 0.7726 & \cellcolor{red!20}0.001 & \cellcolor{green!20}0.037 & \textbf{0.3915} & \cellcolor{red!20}0.000 & \cellcolor{green!20}0.017\\
        ENPC & 0.8123 & \cellcolor{red!20}0.001 & \cellcolor{green!20}0.031 & 0.2394 & \cellcolor{red!20}0.000 & \cellcolor{green!20}0.009\\
        ECP & \textbf{0.8610} & \cellcolor{red!20}0.012 & \cellcolor{green!20}0.016 & 0.2225 & \cellcolor{red!20}0.000 & \cellcolor{green!20}0.014\\
        Graz & 0.8011 & \cellcolor{red!20}0.006 & \cellcolor{green!20}0.078 & 0.2669 & \cellcolor{red!20}0.000 & \cellcolor{green!20}0.023\\ \hline  
        All together & \textbf{0.8591} & \cellcolor{red!20}0.011 & \cellcolor{green!20}0.107 & \textbf{0.4706} & \cellcolor{red!20}0.000 & \cellcolor{green!20}0.020\\ 
        \bottomrule
    \end{tabular}
    \label{training-validation}
    \FONTE{Author's production.}
\end{table}

Both visually (Figure \ref{overview-result-etrims}) and in the table, it is evident that eTRIMS was better suited to deal with the architectural style seen in the SJC dataset. It was expected however, that the values for accuracy and precision would be low, due to the characteristics (similarities) of the SJC dataset. eTRIMS consists of non-rectified facades, lacks symmetry between doors and windows and presents specific architectural styles, characteristics that have similarity with the SJC images. Once the entire online collection has been merged (all-together dataset), the accuracy was increased, but without improvements to the correct delineation of the objects (low F1-score).

\section{3D labeling - Experiments 4 and 5}  
\subsection{RueMonge2014}
The quality of the reconstructed surface (mesh) is highly dependent on the density of the point cloud and the method of reconstruction. Very sparse point clouds can generalize feature volumetry too much, while very dense point clouds can represent it faithfully, but the associated computational cost will also increase. Therefore, there is a limit between the quality of the 3D labeled model and the point cloud density, which falls on the question: how many points it is needed to fairly represent a specific feature? 

Features that are segmented in 2D domain might perfectly align with their geometry, but imprecisions between the geometric edges and the classification may occur. Despite of that, the segmentation alignment onto the mesh is also related to the estimated camera parameters, which are used during ray-tracing. These impressions are directly related to the mesh quality. 

Table \ref{3d-validation} shows how the ray-tracing procedure performed. It was responsible for connecting each segmented feature onto its respective geometry. The 3D reconstruction of both scene had average performance and was not critical since the density was not the highest. RueMonge2014 dense point cloud has around 9 million points, reconstructed in 46.4 minutes. Using a proper computer, equipped with GPU, this rate could certainly be optimized, as well as the other reconstructions.
\begin{table}[H]
    \renewcommand{\arraystretch}{1.4}
    \caption{Ray-tracing performance for geometry classification.}
    \scriptsize \centering		
    \begin{tabular}{L{2.2cm}C{2.2cm}C{2.2cm}C{2.6cm}C{2.4cm}}
        \toprule
        \textbf{Dataset} & \textbf{Point Cloud density} & \textbf{Num. faces (triangles)} & \textbf{3D reconstruction - SfM/MVS (min)} & \textbf{Ray-tracing (sec)}  \\ 
        \toprule
        RueMonge2014 & Sparse & 1,072,646 & 21.4 & 12.42\\
        RueMonge2014 & Dense & 9,653,679 & 46.4 & 27.13\\ \hline
        SJC & Sparse & 800,000 & 13.6 & 20.89\\
        SJC & Dense & 3,058,329 & 35.5 & 41.12\\
        \bottomrule
    \end{tabular}
    \label{3d-validation}
    \FONTE{Author's production.}
\end{table}

In order to illustrate the influences of the point cloud density on the quality of 3D labeling, Figure \ref{mesh-result-ruemonge} shows the result for the RueMonge2014 dataset. Only sparse and dense point clouds were tested. However, to explore the limits between the number of points and the geometric accuracy is some of the issues to approach in a future work.
\begin{figure}[H]
    \centering   
    \caption{3D labeled model of RueMonge2014. (a) Wide view and of the street. (b) Details of facade geometry by a sparse point cloud, (c) its labels after ray-tracing analysis, and (d) close-look of 3D window labels. (e) The facade geometry by a dense point cloud and (f) its labels, and (g) close-look of 3D window labels.}
    \vspace{6mm}    
    \subfigure[]{\label{mesh-result-ruemongea}\includegraphics[width=1\textwidth]{\dropbox/phd/pics/ruemonge/ruemonge2014-dense-50k.png}}    
    \subfigure[]{\label{mesh-result-ruemongeb}\includegraphics[width=0.4\textwidth]{\dropbox/phd/pics/ruemonge/mesh-sparse.png}}
    \subfigure[]{\label{mesh-result-ruemongec}\includegraphics[width=0.4\textwidth]{\dropbox/phd/pics/ruemonge/mesh-color-sparse.png}}
    \subfigure[]{\label{mesh-result-ruemonged}\includegraphics[width=0.09\textwidth]{\dropbox/phd/pics/ruemonge/window-sparse.png}}
    \subfigure[]{\label{mesh-result-ruemongee}\includegraphics[width=0.4\textwidth]{\dropbox/phd/pics/ruemonge/mesh-dense.png}}
    \subfigure[]{\label{mesh-result-ruemongef}\includegraphics[width=0.4\textwidth]{\dropbox/phd/pics/ruemonge/mesh-color-dense.png}}
    \subfigure[]{\label{mesh-result-ruemongeg}\includegraphics[width=0.09\textwidth]{\dropbox/phd/pics/ruemonge/window-dense.png}}    
    \vspace{2mm}
    \legenda{}
    \label{mesh-result-ruemonge}
    \FONTE{Author's production.}
\end{figure}

In Figures \ref{mesh-result-ruemonged} and \ref{mesh-result-ruemongeg}, it is highlighted how well the point cloud density could represent a labeled 3D model. Assuming a hypothetical situation where area or window height information are required in a building inspection to estimate luminosity (indoor and outdoor), the estimation of these parameters should be as close as possible to reality. Therefore, the height and area obtained from the mesh, as in the respective figures, may be inaccurate to this kind of need. \citeonline{martinovic2015} and \citeonline{boulch2013} propose a post-processing procedure, in which the facade has its features simplified by the so called Parsing, where most of the time, Grammar-based approaches \cite{stiny1971} are used. Perhaps the post-processing phase is essential in applications where precise geometric information is required, but it has to ensure that the geometric accuracy does not be penalized once the Grammar can over-generalize the facade feature shape.

In the Figure \ref{zoom-ruemonge}, is highlighted four details about the labeling. In \textquotedblleft A\textquotedblright and \textquotedblleft B\textquotedblright, details on the bottom region of the building, where the store and door class mainly appear. This part of the building, as well as the top, are the critical regions during the reconstruction. 
\begin{figure}[H]
    \centering	   
    \caption{Zoom-in details of the 3D labeled RueMonge2014 reconstruction. (A) and (B) Building's bottom-part: objects that impair the reconstruction and labeling. (C) and (D) Building's upper-part: here, the reconstruction and labeling are impaired by the view that the photos have been taken.}
    \vspace{6mm}    
    \includegraphics[width=1\textwidth]{\dropbox/phd/pics/zoom/zoom-in-result-3D-ruemonge.png}
    \vspace{2mm}
	\legenda{}
    \label{zoom-ruemonge}
    \FONTE{Author's production.}
\end{figure}

At the bottom, the presence of pedestrians, cars, light poles not only hinder the reconstruction, but also become part of the incorrect classification during the projection of pixels on the mesh. In detail in \textquotedblleft A\textquotedblright, however, it may be noted that, despite the presence of vehicles, the classification of stores (in green) is not much impaired. In the illustrations in \textquotedblleft C\textquotedblright~and \textquotedblleft D\textquotedblright, features are highlighted at the top of the building. In this case, the classification is impaired by the projection center. In some regions, as in \textquotedblleft D\textquotedblright, the balconies often open without background. This absence, however, is a consequence of the type of acquisition, not necessarily of a poor labeling, like the information from roof in \textquotedblleft C\textquotedblright. By terrestrial acquisition only, it is not possible to observe much about this structure (as seen in Figure \ref{building-field-of-view-1} and \ref{building-field-of-view-2}, Section \ref{urban-environment}).

\subsection{SJC}
The 3D reconstruction performed by SfM is based on the identification of corners and image analysis, with the purpose of checking for correspondence and acquiring overlapping pairs of images. For this reason, the spectral properties from the urban elements influence the reconstruction process directly. For example, surfaces where the texture is too homogeneous or specular, these will be potential problems and will not be detected by the algorithm. Once not detected, the 3D model will end up with a rough representation, with gaps instead of mapped features. Similarly, the MVS technique (responsible for dense 3D reconstruction) is equally dependent on the homogeneity of the objects.

Unfortunately, many of these spectral properties over SJC facades can be seen. The texture related to walls are often uniform, with windows completed in glasses. Besides, the geometry of acquisition did not contribute in this case. As seen in Figures \ref{mesh-result-sjc2b} to \ref{mesh-result-sjc2d}, all over the street, there are always gaps between the gate and the facade itself. These gaps often imposes problems during and after the reconstruction. As a consequence, a lot of them among important artifacts that could be determinant when trying to identify features in a semantic system. Hence, in order to fully map buildings through the use of SfM/MVS, the imaging of these areas, at least in Brazil, should be complemented by aerial imagery with the aim of targeting these areas (as presented and discussed in Section \ref{facade-features}, Figure \ref{building-field-of-view-2}). The final 3D reconstruction, however, was moderate as the segmentation in 2D domain. 
\begin{figure}[H]
    \centering   
    \caption{3D model of SJC. (a), (b), and (c) Example of gap between the gate and the facade, often present in this specific architectural style. The picture (c) represents the point of view in (b), and vice-versa.}
    \vspace{6mm}
    \subfigure[]{\label{mesh-result-sjc2b}\includegraphics[width=0.3\textwidth]{\dropbox/phd/pics/sjc/realpicture.JPG}}
    \subfigure[]{\label{mesh-result-sjc2c}\includegraphics[width=0.3\textwidth]{\dropbox/phd/pics/sjc/sjc-01.jpg}}
    \subfigure[]{\label{mesh-result-sjc2d}\includegraphics[width=0.3\textwidth]{\dropbox/phd/pics/sjc/sjc-02.jpg}}
    \vspace{2mm}
    \legenda{}
    \label{mesh-result-sjc2}
    \FONTE{Author's production.}
\end{figure}

The Figures \ref{mesh-result-sjcb} (overview), \ref{mesh-result-sjcd} (reconstruction from sparse point cloud), and \ref{mesh-result-sjce} (reconstruction from dense point cloud), are the same residential building as the previous picture, whose geometry is characterized by high walls and gates. Trees and cars appear in most of the images. These objects serve as obstacles, especially in terrestrial and optical campaigns. Of course, this depends solely on the imaged region. In case of RueMonge2014, for example, while pedestrians, cars and vegetation act negatively in the reconstruction, the texture of the facade contributes positively. This makes the final 3D model be penalized, but still, it is an acceptable product. As can be seen in Figures \ref{mesh-result-sjc2} and \ref{mesh-result-sjc}, however, not only the texture, but also the houses geometry and the frequent presence of obstructing objects, negatively affected the reconstruction. In Figure \ref{mesh-result-sjce}, an different area of very poor labeling. Although the gate has been assigned partially correct, the features here are mostly unreadable. 
\begin{figure}[H]
    \centering   
    \caption{3D labeled model of SJC. On top, a wide view of the street. (a) Same view of Figure \ref{mesh-result-sjc2c}, after 3D labeling procedure. (b) Region with spurious labeling - most of the 3D street model was spurious due to the image segmentation quality. (c) Close-look at features details reconstructed by sparse point cloud. (d) Example of the same area using a dense point cloud reconstruction.}
    \vspace{6mm}
    \label{mesh-result-sjca}\includegraphics[width=1\textwidth]{\dropbox/phd/pics/sjc/labeled-street-sjc.png}
    \subfigure[]{\label{mesh-result-sjcb}\includegraphics[width=0.38\textwidth]{\dropbox/phd/pics/sjc/snapshot-200.png}}\hspace{1cm}
    \subfigure[]{\label{mesh-result-sjcc}\includegraphics[width=0.38\textwidth]{\dropbox/phd/pics/sjc/snapshot.png}}    
    \subfigure[]{\label{mesh-result-sjcd}\includegraphics[width=0.38\textwidth]{\dropbox/phd/pics/sjc/sparse-window.png}}\hspace{1cm}    
    \subfigure[]{\label{mesh-result-sjce}\includegraphics[width=0.38\textwidth]{\dropbox/phd/pics/sjc/dense-window.png}}
    \legenda{}
    \label{mesh-result-sjc}
    \FONTE{Author's production.}
\end{figure}

In Figure \ref{zoom-sjc}, some peculiarities are highlighted in the reconstruction and classification of the SJC dataset. In the illustration in \textquotedblleft A\textquotedblright, it is possible to notice details of the roof geometry (high resolution reconstruction - dense cloud), however, the classification of the roof is labeled as a wall (in yellow), and partially correct in the region of the gate (in orange). In \textquotedblleft B\textquotedblright, as previously highlighted, windows with features very similar to the online datasets were correctly identified. However, there were very few occurrences. The architectural style of SJC has its own style and very different from the styles used during CNN training (In \textquotedblleft C\textquotedblright, details of the gap between the gate and the wall itself). Even so, it is concluded that it is possible to identify features of interest using generic training data. The highlight in D, is shown the poor classification in regions whose gates with \textquotedblleft grid\textquotedblright~aspects, confused the prediction with the window feature.
\begin{figure}[H]
    \centering	   
    \caption{Zoom-in details of the 3D labeled SJC reconstruction. (A) Detail of the gate partially segmented. (B) Few windows and balconies were detected, this is an example of a properly detected feature. (C) Gaps between the gate and house (facade). (D) Confusion under gates with \textquotedblleft grid\textquotedblright~characteristics.}
    \vspace{6mm}    
    \includegraphics[width=1\textwidth]{\dropbox/phd/pics/zoom/zoom-in-result-3D-sjc.pdf}
    \vspace{2mm}
	\legenda{}
    \label{zoom-sjc}
    \FONTE{Author's production.}
\end{figure}

    \chapter{CONCLUSION}\label{chapter5}
\section{General conclusions}
Increasingly, the research regarding the facade feature extraction from complex structures, under a dynamic and hard-to-work environment (crowded cities) represent a new branch of research, with perspectives to the areas of technology, such as the concept of smart cities, as well as the areas of Cartography, toward to more detailed maps and semantized systems. In this study, an overview of the most common techniques was presented, as well as an introduction of instruments and ways of observing structural information through remote sensing data. Besides, was also presented a methodology to detect facade features by the use of a CNN, incorporating this detection to its respective geometry through the application of a SfM pipeline and ray-tracing analysis. 

The experiments are mainly focused in aspects such those aforementioned techniques and their computational capability in detecting facade features, regardless of architectural style, location, scale, orientation or color variation. All the images used in the training procedures underwent no preprocessing whatsoever, keeping the study area as close as possible as to what would be a common-user dataset (photos taken from the streets). 

In this sense, the edges of the acquired delineated features show the robustness of the CNN technique in segmenting any kind of material, in any level of brightness (shadow and occluded areas), orientation, or presence of pedestrians and cars. Considering that the values achieved for the individual datasets were above 90\%, it is concluded that the CNNs can provide good results for image segmentation in many situations. However, being a supervised architecture, the network has to pass through a huge training set, with no guarantees of good inputs, in order to get reliable inferences. When applied over unknown data, such as the experiment on the SJC data, it was noted that the neural network failed, except in regions where the facade features share similar characteristics, though such occasions were rare.

\subsection{Brazilian architectural styles and unreachable areas}
In Section \ref{map-brazil}, it is shown some of the difficulties encountered specifically in Brazil. In addition, in Chapter \ref{chapter4}, experiments were carried out on the extraction and reconstruction of features on Brazilian facades. As mentioned before, factors such as local culture, number of inhabitants and economy are determinant in local geometry issues, especially in poor regions whose civil construction is often made irregularly, over equally irregular and difficult access areas. Not only that, but the factors can also determine, for example, the geometry of city-centers, such as the presence of skyscrapers (e.g. S�o Paulo, Brazil; Changhai, China; and New York, United States) or commercial areas with buildings with a maximum of four floors (e.g. London, Germany, and France). Thus, studies involving the measurement of these factors could contribute as input for the development of 3D reconstruction techniques as presented in this document.

\section{Source-code: Usability, Licenses and Extension}\label{code-usability-and-extension}
References to the codes and experiments carried out in this study can be found in the Table \ref{tools-used}, in Appendix \ref{apendiceA}. Since the beginning of this work, libraries such as CGAL, PCL, VisualSfM, COLMAP, OpenCV, among others, have been used for tests and experiments throughout development. Of course, some of these libraries and software were discarded despite having a key role in refining the final methodology. Therefore, listing the technologies used, their sources, and other characteristics, is a way of guiding other works and contributing with related researches.

\subsection{Difficulties}
This work was entirely carried out with the aid of Open-Source tools, for the most part, under MIT and GPL licenses\footnote{Licenses of a public domain that could be extended and, in case of GPL, must be shared.}, which includes the non-commercial extension and free-to-use code. In the table \ref{tools-used}, in the Appendix \ref{apendiceA}, the tools used and their relation with the respective methodological step are presented.

In addition to the results archived in this study, every implementation was carried out in order to be expandable, interoperable, understandable, cohesive, public and easily accessible. Therefore, all methodological steps are available for consultation and public use at the GitHub\texttrademark~\cite{github}, where the links and respective explanations are available in Table \ref{tools-developed}. All the datasets used for this research were provided with any cost. In the same way, private tools that were essential for the development of the research were difficult to access or too expensive to acquire, most of the technologies used in this study were free, for example, the tools mentioned in Section \ref{code-usability-and-extension}.

Due to the technology approached in this study covering the state-of-the-art of classification and urban mapping solutions, the supply of training and courses is scarce. Likewise, in Brazil, there was a great difficulty in learning how to process LiDAR data during the first phases of this study. As we progressed, the operations involving SfM/MVS became more consistent, mainly, with the involvement and cooperation of multiple institutions (academic, military - national and international). For example, the DSG, in the revision of text regarding Brazilian standards and specifications for 3D geographic mapping in Brazil. The Department of Photogrammetry (IfP), at the University of Stuttgart, Germany, by offering disciplines of Computer Vision and Pattern Recognition. The Federal University of Paran� (UFPR), by their methodic assistance regarding the use of LiDAR in urban areas, among others. 

\section{Future prospects}
Identifying facade features under a great variety and arrangement make up these tasks still a scientific challenge whose tendency is to expand. The technologies to observe cities, such as sophisticated sensors, reconstruction and classification techniques, evolve as the numerous architectural styles change according to local culture and way of life. Moreover, it is essential to think that the multiplicity of architectural styles is not the only problem. Studies, such as those carried out at the MIT Center for Art, Science and Technology (CAST), Massachusetts Institute of Technology (MIT)\footnote{Video available at https://www.youtube.com/watch?v=vRfNbhyPPKs. Accessed \today.} and at Eidgen�ssische Technische Hochschule (ETH) Z�rich \cite{adriaenssens2016}, show, for example, that materials used in construction might become dynamic and therefore do not present a single static structure of a building. Urban occupation tends to evolve, which also demands that mapping techniques must both to follow the current architectural structures, as well as their eminent evolution.

As future prospects, to explore aspects such as the use of non-supervised models, separate tasks such as pre-classification of architectural styles, and mix different DL techniques to deal with specific scenarios, such as the chaotic arrangement of urban elements. Even it is the first study case, the methodology presented is highly dependent of the quality and number of images for training. The power of generalization in a neural network occurs as soon as the training set is large enough, as well as their resolutions. Besides, once this study has shown the robustness of CNN over complicated situations, we believe that efforts directed towards post-processing techniques could make the final 3D labeled model even more accurate.

    % -------------------------------------------------------------------------------------------
    
    % -- BIB  -----------------------------------------------------------------------------------
    \bibliography{\bibpath}
    % -------------------------------------------------------------------------------------------
    
    % -- APENDICES  -----------------------------------------------------------------------------
    \inicioApendice
    \chapter{APPENDIX A - MATERIAL DETAILS}\label{apendiceA}	
    \section{Tools and modules used along the study}\label{apendiceA-tools}
    \begin{table}[!htb]
      \renewcommand{\arraystretch}{1.6}
      \caption{Sources-code, softwares, libraries and their public links on Github.}
      \scriptsize \centering	
      \rowcolors{2}{white}{gray!25}
      \begin{tabular}{L{1.8cm}|L{3.2cm}|C{1.8cm}|C{0.9cm}|C{1.4cm}|C{0.8cm}|L{2.5cm}}
	      \toprule
	      \textbf{Name} & \textbf{Description}  & \textbf{Type} & \textbf{Version} & \textbf{Purpose} & \textbf{URL} & \textbf{Reference} \\ 
	      \toprule
	      scipy & Mathematics, science, and engineering & Python module & & Image Processing & \href{https://www.scipy.org}{\faLink} & - \\ 
	      numpy & Scientific computing & SciPy module & & Image Processing & \href{http://www.numpy.org}{\faLink} & - \\
	      matplotlib & Comprehensive 2D Plotting & SciPy module & & Image Processing & \href{http://matlibplot.sourceforce.net}{\faLink} & \cite{hunter2007} \\ 
	      sklearn & Machine Learning & Python module & 0.18.1 & Image Processing & \href{http://scikit-learn.org}{\faLink} & - \\ 
	      pylsd & Line Segment Detector (LSD) & Python Library & - & Image Processing & \href{https://github.com/primetang/pylsd}{\faGithub}$~~$\href{http://www.ipol.im/pub/art/2012/gjmr-lsd/}{\faLink} & - \\ 
	      skimage & Image processing & Python module & 0.14 & Image Processing & \href{http://scikit-image.org}{\faLink} & - \\ 
	      quickshift & Image segmentation & SkImage module & - & Image Processing & \href{http://scikit-image.org/docs/dev/api/skimage.segmentation.html}{\faLink} & \cite{vedaldi2008} \\ 
	      cv2 & OpenCV Image Processing & Python Library & 3.2.0.8 & Image Processing & \href{https://github.com/opencv/opencv}{\faGithub}$~~$\href{http://opencv.org}{\faLink} & - \\ \hline
	      tensorflow & Tensorflow & Python Library & 1.5 & Image Processing & \href{https://github.com/tensorflow/tensorflow}{\faGithub}$~~$\href{https://github.com/tensorflow}{\faLink} & -\\ \hline	      
	      embree & High Performance Ray Tracing Kernels & C++ Library & 2.0 & Ray-tracing & \href{https://github.com/embree}{\faGithub}$~~$\href{https://embree.github.io/}{\faLink} & -\\ 
	      vtk & The Visualization Toolkit & C++ Library & 7.1.1 & Ray-tracing & \href{https://github.com/Kitware/VTK}{\faGithub}$~~$\href{https://www.vtk.org/Wiki/VTK}{\faLink} & - \\ \hline
	      cgal & Computational Geometry Algorithms (CGAL) & C++ Library & 4.10 & Point cloud processing & \href{https://github.com/CGAL/cgal}{\faGithub}$~~$\href{http://www.cgal.org}{\faLink} & \cite{cgal410} \\ 
	      pcl & Point Cloud Library (PCL) & C++ Library & 4.7 & Point cloud processing & \href{https://github.com/PointCloudLibrary}{\faGithub}$~~$\href{http://pointclouds.org}{\faLink} & \cite{pcl2011} \\
	      lastools & LAStools & C++ Module & - & Point cloud processing & \href{https://github.com/LAStools/LAStools}{\faGithub}$~~$\href{https://rapidlasso.com}{\faLink} & \cite{rapidlasso} \\
	      libigl & C++ geometry processing & C++ Library & 1.0 & Visualization & \href{https://github.com/libigl/libigl}{\faGithub}$~~$\href{http://libigl.github.io/libigl/tutorial/}{\faLink} & \cite{libigl} \\
	      CloudCompare & 3D point cloud and mesh processing & GUI & 2.9 & Visualization & \href{https://github.com/CloudCompare/CloudCompare}{\faGithub}$~~$\href{http://www.cloudcompare.org}{\faLink} & - \\ 
	      MeshLab & 3D point cloud and mesh processing & GUI & 2016.12 & Visualization & \href{https://github.com/cnr-isti-vclab/meshlab}{\faGithub}$~~$\href{http://www.meshlab.net}{\faLink} & \cite{meshlab} \\  
	      \bottomrule
      \end{tabular}
      \label{tools-used}
      \FONTE{Author's production.}      
    \end{table}

    \section{Author's production along this study}\label{apendiceA-development}
    \begin{table}[H]
      \renewcommand{\arraystretch}{1.6}
      \caption{Sources-code produced by the author throughout the study. The usability and further explanations, can be found at the respective Github links.}
      \scriptsize \centering	
      \rowcolors{2}{white}{gray!25}
      \begin{tabular}{L{1.3cm}|L{4cm}|C{2.2cm}|C{1.2cm}|C{2.5cm}|C{0.7cm}|C{0.7cm}}
	      \toprule
	      \textbf{Name} & \textbf{Description} & \textbf{Methodological stage} & \textbf{Language} & \textbf{Github} & \textbf{Type} & \textbf{URL}\\
	      \toprule
	      Image processing & Preprocessing on images to apply the Superpixel routine. The preprocessing procedures here include filtering, colormap conversion (i.e. HSV, Lab, Local Entropy), etc & Detection & C++ & rodolfolotte/image & - & \href{https://github.com/rodolfolotte/image/tree/master/processing}{\faGithub}\\	      
	      Superpixel & Segmentation routine with numerous versions of Superpixels, for example, SLIC, SLICO, LSD and SEEDS. The routine was programmed to perform experiments involving classification of facades & Detection & C++ & rodolfolotte/gdal-segment & \faCodeFork & \href{https://github.com/rodolfolotte/gdal-segment}{\faGithub}\\ 	      
	      Point cloud preprocessing & Includes point cloud preprocessing routines, such as simplification, filtering, CRF classification, etc & Reconstruction & C++ & rodolfolotte/point-cloud & - & \href{https://github.com/rodolfolotte/point-cloud}{\faGithub}\\	      
          Inputs preparation & Preparation code for neural network inputs. The code includes preparation of annotated images and creation of text files containing the paths of the images & Detection & Python & rodolfolotte/deep-learning & - & \href{https://github.com/rodolfolotte/deep-learning/tree/master/cnn-projects/inputs}{\faGithub}\\	      
          CNN & Routines responsible for the Convolutional Neural Network training, as well as the routines for the inferences & Detection & Python & rodolfolotte/deep-learning & \faCodeFork & \href{https://github.com/rodolfolotte/KittiSeg}{\faGithub}\\	      
	      Ray-tracing & Read the camera parameters, and by the configuration of image blocks, project the segmented image onto the mesh, in a way the triangles mesh are assigned according to each class & Ray-Tracing & C++ & rodolfolotte/3d-reconstruction & - & \href{https://github.com/rodolfolotte/3d-reconstruction}{\faGithub}\\	      
	      Evaluation & Evaluates the segmented images against ground-truth. Here, all evaluation metrics and plots can be setted up & Detection & Python & rodolfolotte/image & - & \href{https://github.com/rodolfolotte/image/tree/master/evaluation}{\faGithub}\\	      
	      PC vizualization & Routines to visualize meshes and customized point of views, colors, etc & Reconstruction & C++ & rodolfolotte/3d-reconstruction & - & \href{https://github.com/rodolfolotte/3d-reconstruction}{\faGithub}\\ \hline
	      Thesis & \LaTeX version of this thesis & Documentation & \LaTeX\ & rodolfolotte/doc & - & \href{https://github.com/rodolfolotte/documentation}{\faGithub}\\
	      \bottomrule	
      \end{tabular}
      \label{tools-developed}
      \FONTE{Author's production.}
    \end{table}
    %%exemplo: https://books.google.de/books?id=AQjmCgAAQBAJ&pg=PA36&lpg=PA36&dq=python+segmentation+statistics+parameters&source=bl&ots=o4kqBclAqV&sig=xX8oiOdu__38B8pO-fhsRNMiWR4&hl=pt-BR&sa=X&ved=0ahUKEwiVvI-C4sTVAhUCsxQKHUogDVwQ6AEIQTAD#v=onepage&q=python%20segmentation%20statistics%20parameters&f=false

    % The CGAL library

    %technologies that have appeared over the past few years include the stabilization of numeric arrays and processing (NumPy), the advancement continuing stabilization of a broad base of scientific algorithms (SciPy), the development of a robust interface to the R statistical modeling package (RPy), and array and volume visualization (Matplotlib). 

    %...we have found that the use of Python as the core programming language for our methodology provides significantly better control over most aspects of an experiment than is possible with existing packages, including commercial packages.

    
    

    %Hough_normal_estimation
    %Hough_normal
    % @article{Boulch:2012:FRN:2346796.2346811,
    %  author = {Boulch, Alexandre and Marlet, Renaud},
    %  title = {Fast and Robust Normal Estimation for Point Clouds with Sharp Features},
    %  journal = {Comp. Graph. Forum},
    %  issue_date = {August 2012},
    %  volume = {31},
    %  number = {5},
    %  month = aug,
    %  year = {2012},
    %  issn = {0167-7055},
    %  pages = {1765--1774},
    %  numpages = {10},
    %  url = {http://dx.doi.org/10.1111/j.1467-8659.2012.03181.x},
    %  doi = {10.1111/j.1467-8659.2012.03181.x},
    %  acmid = {2346811},
    %  publisher = {John Wiley \& Sons, Inc.},
    %  address = {New York, NY, USA},
    %  keywords = {I.3.5 [Computer Graphics]: Computational Geometry and Object Modeling\&\#x2014;}

    % nanoflann
    % @misc{blanco2014nanoflann,
    %   title        = {nanoflann: a {C}++ header-only fork of {FLANN}, a library for Nearest Neighbor ({NN}) wih KD-trees},
    %   author       = {Blanco, Jose Luis and Rai, Pranjal Kumar},
    %   howpublished = {\url{https://github.com/jlblancoc/nanoflann}},
    %   year         = {2014}
    % }

    % \section{Source-code related to this thesis}
    % As we were benefit with a lot of open-source and free implementations from inumerous institutes, faculties, and foruns, we have also a pleasure to describe and share each piece of our work. In order to do that, we have use the Github platform \cite{github}, which easily provide all support to make it public, shareable, \footnote{Too see how to cite a respective source-code, we recommend the follow reading: https://guides.github.com/activities/citable-code/.}. Following the workflow presented in the Figure \ref{methodology}, the Table \ref{source-code} gives the source-code links of each respective step. 
    % \begin{table}\label{source-code}
    %   \renewcommand{\arraystretch}{1.1}
    %   \caption{Sources-code methodology and its public links on Github.}
    %   \footnotesize \centering		
    %    \begin{tabular}{l|C{2.6cm}|C{2.0cm}|C{0.6cm}|C{1.5cm}|C{1.8cm}|C{0.5cm}}
    % 	  \hline \hline
    % 	  \textbf{Name} & \textbf{Description}  & \textbf{Type} & \textbf{Version} & \textbf{Purpose} & \textbf{URL} & \textbf{Reference} \\ \hline
    % 	  SciPy & Python-based for mathematics, science, and engineering & Python module & & - & https://www.scipy.org/ & - \\ \hline
    % 	  NumPy & Scientific computing & SciPy module & & - & http://www.numpy.org/ & - \\ \hline	  			
    % 	  Matplotlib & Comprehensive 2D Plotting & SciPy module & & - & http://matlibplot.sourceforce.net & \cite{hunter2007} \\ \hline
    % 	  SkLearn & Machine Learning & Python module & 0.18.1 & - & http://scikit-learn.org/stable/ & - \\ \hline
    % 	  PyLSD & Line Segment Detector & Library & - & - & https://github.com/primetang/pylsd & - \\ \hline
    % 	  cv2 & OpenCV Image Processing & Library & 3.2.0.8 & & http://opencv.org/ & -\\ \hline	
    % 	  CGAL &  Computational Geometry Algorithms & C++ Library & 4.10 & & http://www.cgal.org/ & \cite{cgal410} \\ \hline
    % 	  CloudCompare & 3D point cloud and mesh processing & IDE & 2.9 & Visual & http://www.cloudcompare.org/ & - \\ \hline
    % 	  MeshLab & 3D point cloud and mesh processing & IDE & 2.9 & Visual & http://www.meshlab.net/ & \cite{meshlab} \\ \hline 	  
    % 	  \hline			
    %   \end{tabular}			
    % \end{table}

    \chapter{APPENDIX B - 3D MAPPING IN BRAZIL}\label{apendiceC}    
  \section{Poll about the use of 3D maps in Brazil}\label{apendiceC-poll}
  The questionary was made by email, in some cases, was not possible to reach some of the capitals either by email or phone. The detailed answers from Table \ref{poll-answers} can be seen in Table \ref{poll-answers2}.
  \begin{table}[!H]
    \renewcommand{\arraystretch}{1.6}
    \caption{Poll answers from Table \ref{poll-answers} in details.}
    \tiny \centering
    \rowcolors{2}{white}{gray!25}
    \begin{tabular}{lcp{11.5cm}}    
        %     \begin{tabular}L{1cm}L{0.4cm}L{4.6cm}L{2.3cm}L{4cm}}  
        \toprule
        \textbf{Capital} & \textbf{State}  & \textbf{Answer} \\ 
        \toprule
        Fortaleza & CE & \textbf{a)} The City Hall of Fortaleza is acquiring the 3D mapping by laser profiling. Currently, department does not use this data for studies.\par \textbf{d)} Planning is performed without any 3D map.\par \textbf{e)} Studies using the Z component.\\% & ouvidoria.seuma@fortaleza.ce.gov.br\\
        
        Vit�ria & ES & \textbf{a)} No.\par \textbf{d)} By 2D thematic maps, with additional database information, such as occupation, use, height, feedback, distribution of activities, commerce and urban services, public equipment, socioeconomic characteristics, among others.\par \textbf{e)} For example, Vit�ria has a map on the protection of the natural heritage of the municipality and the landscape. A 3D map would aid in reconciling territory occupation policies with the protection policy of this heritage, as well as advanced studies such as volumetric tests, among others. It can be an important tool in the studies of protection of the environmental patrimony, mangroves, hills and conservation units, allowing a reading of the constructed object of the city and its relation with the environmentally protected areas.\\% & clivia.leite@correio1.vitoria.es.gov.br\\
        
        Porto Alegre & RS & \textbf{a)} Yes, Digital Surface Model (DSM) and Digital Terrain Model (DTM).\par \textbf{b)} At the moment, in the identification of areas susceptible to geological, geotechnical and hydrological risks; in the estimation of the height of the buildings, for tax purposes. GIS system in Web platform, for internal queries.\par \textbf{e)} Support for decision making, confronting the planned environment with the built environment, impact and shading simulations, volumetric simulations. Volumetric studies.\\% & rodrigoml@smurb.prefpoa.com.br\\
        
        S�o Paulo & SP & \textbf{a)} Yes. DSM and DTM. In 2017, a new campaign using LiDAR sensors was performed and new 3D models will be produced.\par \textbf{b)} Workings such as the Strategic Master Plan, Zoning, Drainage Plan, among others. \par \textbf{c)} The DTM is used as a reference for other surveys, including planning and execution of some current aerial imaging stages. However, in projects and infrastructure, the effective use of these 3D digital technologies is still in the process of being disseminated to the technical staff.\par \textbf{d)} Although the 3D models are available, 2D is still predominantly used in urban planning activities.\par \textbf{e)} They incorporate technical features that are extremely useful for urban management.\\% & amlaurenza@prefeitura.sp.gov.br\\
        
        Belo Horizonte & MG & \textbf{a)} Yes. The execution of 3D modeling is carried out on demand in projects or urban plans that require analysis of modifications in the urban landscape.\par \textbf{b)} The 3D maps help in the visualization of impacts, in the increase of the constructive density, in the generation of obstructions of sightings, in the relations between equipment and urban infrastructures, among others.\par \textbf{c)} Volumetry studies are carried out for specific projects, such as regions of Urban Operations, Plans in Areas of Social Interest, Plans of Cultural Regions.\par \textbf{d)} Planning is already done with the eventual use of 3D modeling of the terrain and buildings.\par \textbf{e)} The 3D maps help in diagnosing the situation before and after the implementation of projects to evaluate the urban impacts of density and volumetry in parameters established in the plans.\\% & guilherme.vargas@pbh.gov.br\\
        
        Rio de Janeiro & RJ & \textbf{a)} The buildings are represented in two dimensions with possibility of elevation in 3D. Calculated density calculations for proposed zoning alterations, urban parameters and studies of impacts on the landscape.\par \textbf{b)} 3D simulations allow for technical discussion, improvement of proposals and better communication with managers, city councilors and the general population.\par \textbf{c)} Most of urban law plans made in the last five years have included 3D maps on strategic areas.\par \textbf{d)} In addition to the 3D studies, the urban planning uses information produced by both federal and state bodies such as IBGE, ISPE and also municipal, produced by the municipal authorities.\par \textbf{e)} Associating the relief with the buildings, it contributes to studies of insolation, heat islands, aimed at the natural and cultural landscape, simulations and urban parameters such as building heights and their distances to the other buildings and impacts on the environment.\\% & valeriahazan.pcrj@gmail.com\\
        
        Curitiba & PR & \textbf{a)} Yes. Partial maps, that is, not of the whole city, but of parts of it.\par \textbf{b)} Providing visualization of occupancy scenarios (for example, using all basic building potential or with extra potential acquisition); in studies of insolation and microclimate. Answered in the previous question.\par \textbf{e)} In addition to the current applications, these maps could be used, for example, in Neighborhood Impact Studies and project detailing.\\% & geoprocessamento@ippuc.org.br\\
        
        Recife & PE & \textbf{a)} Yes, the Urban Planning Dept. has georeferenced databased with vector and raster data from LiDAR campaings. Another tool that it is used is Google Earth Pro, which has the 3D model of the entire municipality of Recife.\par \textbf{b)} A 3D model allows us to manipulate, take measurements, visualize any interventions already built, and also simulate transformations from anywhere.\par \textbf{c)} Any analysis of morphology and occupation of the territory. These activities include analyzes of impact projects, simulation of changes in legislation and simulations, and studies of interventions. \\%& icps@recife.pe.gov.br\\
        \bottomrule
    \end{tabular}    
    \label{poll-answers2}
    \FONTE{Author's production.}
  \end{table}
    
  The contact of each professional responsible for the answers, can be found in Table \ref{contacts}.  
  \begin{table}[!ht]
  \renewcommand{\arraystretch}{1.6}
  \caption{Responsibles for the 3D mapping poll answers, sent to the Brazilian capitals infraestructure department (Table \ref{poll-answers2}, in Section \ref{3d-urban-brazil}).}
  \scriptsize \centering
  \rowcolors{2}{white}{gray!25}
    \begin{tabular}{L{2.2cm}L{1cm}L{4.8cm}L{4.7cm}}  
      \toprule
      \textbf{Capital} & \textbf{State}  & \textbf{Name} & \textbf{Email} \\ 
      \toprule
      Fortaleza & CE & Ouvidoria & ouvidoria.seuma@fortaleza.ce.gov.br\\      
      Vit�ria & ES & Clivia Leite Mendon�a & clivia.leite@correio1.vitoria.es.gov.br\\      
      Porto Alegre & RS & Rodrigo Marsillac Linn & rodrigoml@smurb.prefpoa.com.br\\      
      S�o Paulo & SP & Ana Maria A. Laurenza and Silvio C. L. Ribeiro & amlaurenza@prefeitura.sp.gov.br\\      
      Belo Horizonte & MG & Guilherme Pereira de Vargas & guilherme.vargas@pbh.gov.br\\            
      Rio de Janeiro & RJ & Val�ria Hazan & valeriahazan.pcrj@gmail.com\\      
      Curitiba & PR & Oscar Ricardo M. Schmeiske and Alessandro Dias & geoprocessamento@ippuc.org.br\\      
      Recife & PE & Tiago Henrique & icps@recife.pe.gov.br\\
      \bottomrule
    \end{tabular}
    \label{contacts}
    \FONTE{Author's production.}
  \end{table}

    % -------------------------------------------------------------------------------------------
    
    % -- GLOSSARIO  -----------------------------------------------------------------------------
    \begin{glossario}
    \begin{description}
    \item [3D representation] A digital 3D version of a real life object.
    \item [3D reconstruction] A real life object that has been created digitally through an automatic or semi-automatic computational procedure.
    \item [3D generation] A real life object that has been created digitally through a manual edition, or specialist in 3D designs.
    \item [3D model] Same as 3D representation.
    \item [3D labeling] The procedure of classifying a 3D representation.
    \item [Image annotation] Images that carries the classified features, or, what each pixel corresponds to. These images are used as ground-truth during CNN training.
    \item [Street-side images] Photos taken from the streets, from the ground and pointed toward the facade (perpendicular).
    \item [Facade features] Individual elements essential to urban mapping. These elements could be, for example, decisive in the application and execution of urban and supervisory laws.
    \item [Openings] Same as Facade Features. Openings are all those related to window, door, gates, entrances, etc.
    \end{description}
\end{glossario}
    
    % -------------------------------------------------------------------------------------------
    
    % -- ANEXOS  --------------------------------------------------------------------------------
    \inicioAnexo
    \chapter{ANNEX A - CROSS-ENTROPY}\label{annexA} 
  % https://rdipietro.github.io/friendly-intro-to-cross-entropy-loss/#cross-entropy
  % https://datascience.stackexchange.com/questions/9302/the-cross-entropy-error-function-in-neural-networks
  \section{Example of cross-entropy calculation}  
  The main goal when developing a probabilistic classificator is to map inputs to probabilistic predictions, incrementally adjusting the model's parameters as the amount of erros are observed. Thus, that prediction get closer and closer to ground-truth probabilities. One way to interpret cross-entropy is to see it as a negative log-likelihood for the ground-truth ($y_i'$), under their prediction ($y_i$). Then, \textquotedblleft get closer\textquotedblright~means that when distance between $y_i'$ and $y_i$ is minimal.
  
  Taking an example from internet\footnote{Available at https://rdipietro.github.io/friendly-intro-to-cross-entropy-loss/\#cross-entropy. Accessed \today.}, supposing a fixed model (hypothesis) that predicts for $n$ classes ${1,2,...,n}$ their hypothetical occurrence probabilities $y_1,y_2,...,y_n$. Suppose that now observing (in reality) $k_1$ instances of class 1, $k_2$ instances of class 2, $k_n$ instances of class $n$, so on. According to this model, the likelihood of this happening is:
  \begin{align}
   P[data|model] &= y^{k_1}_1~y^{k_2}_2~...~y^{k_n}_n~,  
  \end{align}
  taking the logarithm and changing the sign:
  \begin{align}
   -log~P[data|model] &= -k_1~\log~y_1~-k_2~\log~y_2~...~-k_n~\log~y_n\\
   &= -\sum_ik_i~log~y_i~,
  \end{align}
    
  dividing the right-hand sum by the number of observations $N=k_1+k_2+...+k_n$, and denote the empirical probabilities as $y_i'=k_i/N$, the cross-entropy is then obtained:
  \begin{align}
   -\frac{1}{N}log~P[data|model] &= -\frac{1}{N}\sum_ik_i~\log~y_i\\
   &= -\sum_iy_i'~\log~y_i\\
   &= Loss_{y'}(y)~.
  \end{align}

    \chapter{ANNEX B - 2D EVALUATION}\label{annexB} 
For validation, reference and prediction images (of the same dimension and color map) are essentially required. Once in agreement, the images are confronted so that an $N$ number of randomly selected samples are compared according to their color properties. The resulting success and error count is acquired in the form of a matrix, the confusion matrix. One of the axes denotes the reference classes, and another, the predicted classes. The resulting matrix as well as the success and error metrics ($TP$, $TN$, $FP$, and $FN$), therefore, is illustrated in Figure \ref{cm-representation}.
\begin{figure}[H]
    \centering    
    \caption{Multiclass confusion matrix and the respective success and error rates.}
    \vspace{6mm}
    \subfigure[TP]{\includegraphics[width=0.22\textwidth]{\dropbox/phd/pics/cm/tp.png}}
    \subfigure[FN]{\includegraphics[width=0.24\textwidth]{\dropbox/phd/pics/cm/fn.png}}      
    \subfigure[FP]{\includegraphics[width=0.205\textwidth]{\dropbox/phd/pics/cm/fp.png}}       
    \subfigure[TN]{\includegraphics[width=0.215\textwidth]{\dropbox/phd/pics/cm/tn.png}}
	\legenda{}
    \label{cm-representation}
    \FONTE{Author's production.}
\end{figure}

Then, considering $M$ as the confusion matrix, $C$ number of classes, $i$ (predicted) and $j$ (reference) as the axes, the metrics are given by:
\begin{equation}
 M_{ij}=
 \begin{cases}
  TP = {\forall~m~\in~M:~i=j,m~\Re>0}~,\\  
  FP = {\sum_{j=1}^CM_{ij}-TP:~j={1,...,C}}~,\\
  FN = {\sum_{i=1}^CM_{ij}-TP:~i={1,...,C}}~,\\
  TN = {\sum_{i=1,j=1}^CM_{ij} - TP - FP - FN:~i={1,...,C},j={1,...,C}}~,
 \end{cases}
\end{equation}
where, in summary, $TP$ is the diagonal elements, $FP$ is the sum of elements in column, except $TP$, $FN$ is the sum of elements in row, except $TP$, and $TN$ is the sum of all elements execpt $TP$, $FP$, and $FN$. Then, the accuracy and F1-score is finally calculated:
\begin{align}
 Accuracy &= \frac{TP + TN}{n}~,
\end{align}
where $n$ is the number of random samples used, 
\begin{align}
 Precision &= TP/(TP+FP)~,
\end{align}
\begin{align}
 Recall &= TP/(TP+FN)~,
\end{align}
finally:
\begin{align}
 F1 &= 2 * \frac{precision * recall}{precision + recall}~, 
\end{align}

The evaluation's source-code described here, is shared and can be found in Github (link available in Table \ref{tools-developed}).

\section{Practical example}
Using synthetic images (Figure \ref{cm-synthetic}), this practical example demonstrates the influence of boundary delineation over F1-score metric.
\begin{figure}[H]
    \centering    
    \caption{Evaluation over synthetic data. A practical example using (a) reference. (b) object segmented out of the reference boundary (outer). (c) object segmented out of the reference boundary (inner). (d) object partially segmented.}
    \vspace{6mm}
    \subfigure[]{\label{cm-synthetica}\includegraphics[width=0.242\textwidth]{\dropbox/phd/pics/cm/reference.png}}
    \subfigure[]{\label{cm-syntheticb}\includegraphics[width=0.242\textwidth]{\dropbox/phd/pics/cm/image-1.png}}      
    \subfigure[]{\label{cm-syntheticc}\includegraphics[width=0.242\textwidth]{\dropbox/phd/pics/cm/image-2.png}}       
    \subfigure[]{\label{cm-syntheticd}\includegraphics[width=0.242\textwidth]{\dropbox/phd/pics/cm/image-3.png}}
	\legenda{}
    \label{cm-synthetic}
    \FONTE{Author's production.}
\end{figure}

In the Table \ref{cm-synthetic-res} is shown the values for the synthetic evaluation. When test over the same reference image, the accuracy and F1-score seems correct, as there is no changing, but when the predicted object is bigger than in reference, then the recall is high but not the precision, leading to a low F1-score. When the predicted object is smaller it should be, it inverts, recall is lower than precision, also leading to a low F1-score. Finally, a predicted object intersecting only part of the reference, both values are low, equally to F1-score.

\begin{table}[H]
    \renewcommand{\arraystretch}{1.4}
    \caption{Demonstration of the Accuracy and F1-score changing according to the target boundary and location.}
    \scriptsize \centering
    \rowcolors{2}{white}{gray!25}
    \begin{tabular}{L{2cm}C{2cm}C{2cm}C{2cm}C{2cm}C{2cm}}
        \toprule
        \textbf{Reference} & \textbf{Predicted} & \textbf{Accuracy} & \textbf{Precision} & \textbf{Recall} & \textbf{F1-Score}\\ 
        \toprule
        \includegraphics[width=0.1\textwidth]{\dropbox/phd/pics/cm/reference.png} & \includegraphics[width=0.1\textwidth]{\dropbox/phd/pics/cm/reference.png} & 1.0 & 1.0 & 1.0 & 1.0 \\
        \includegraphics[width=0.1\textwidth]{\dropbox/phd/pics/cm/reference.png} & \includegraphics[width=0.1\textwidth]{\dropbox/phd/pics/cm/image-1.png} & 0.8903 & 0.8101 & 0.9345 & 0.8478 \\      
        \includegraphics[width=0.1\textwidth]{\dropbox/phd/pics/cm/reference.png} & \includegraphics[width=0.1\textwidth]{\dropbox/phd/pics/cm/image-2.png} & 0.8539 & 0.9252 & 0.5871 & 0.6080 \\      
        \includegraphics[width=0.1\textwidth]{\dropbox/phd/pics/cm/reference.png} & \includegraphics[width=0.1\textwidth]{\dropbox/phd/pics/cm/image-3.png} & 0.6162 & 0.5134 & 0.5212 & 0.4969 \\      
        \bottomrule
    \end{tabular}    
    \label{cm-synthetic-res}
    \FONTE{Author's production.}
\end{table}



    % -------------------------------------------------------------------------------------------
    
\end{document}
