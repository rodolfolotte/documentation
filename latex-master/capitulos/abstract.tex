%%%%%%%%%%%%%%%%%%%%%%%%%%%%%%%%%%%%%%%%%%%%%%%%%%%%%%%%%%%%%%%%%%%%%%%%%%%%%%%%
% ABSTRACT
\begin{abstract}
\hypertarget{estilo:abstract}{} %% uso para este Guia
      The research involving computational methods for the extraction of roads has intensified in the last two decades. The procedure is usually performed through the analysis of images acquired by optical sensors or radar (Radio Detection and Ranging). The great advantage in the usage of radar images is the possibility of survering areas often covered by clouds, since the imaging by active sensors is independent from atmospheric conditions in the region of interest. The cartographic mapping based on these images is often done manually, requiring considerable time and effort from the interpreter. There are currently many studies involving the extraction of roads by means of automatic or semi-automatic approaches, however, each of them has different solutions for different problems, making this task a scientific issue still open. Among the advantages of using radar images, one can mention the acquistion of images regardless of atmospheric and ilumination conditions, besides the possibility of surveying regions where the terrain is hidden by the vegetation canopy, among others. This work comprises the semi-automatic generation of seed points located close to the features of interest by means of Self-Organizing Maps (SOM). We then employ an active contour method (Snakes) for the extraction of the roads centre-axis in a synthetic aperture radar (SAR) airborne image. The obtained results were evaluated as to their quality with respect to perfection, correction, and redundancy.
\end{abstract}


